\chapter{Conditioning和Palm Theory}

\section{导论}

\subsection{Conditioning和典型的点}
点过程理论中的Palm概率或Palm测度是
给定点过程在某个位置包含点的事件。它还
将过程的“典型点”的概念形式化。非正式地,典型的
点源于一个选择过程,其中每个点都有相同的机会
选择。这个想法需要在数学上精确,尤其是在
无限点过程。例如,根据某些采样选择的点
程序,例如最接近原点的程序,不是典型的,因为它已经
以特定的、确定的方式选择。直观地看,Palm分布是
条件点过程分布,给定一个点(典型点)存在于
特定位置。

这种类型的conditioning有时指代interior conditioning,因为conditioning 作用在
$x\in \Phi$,问题是点过程在$x$之外表现如何?
相形之下, Papangelou conditional intensity是基于exterior conditioning的,
因为conditioning作用在$\mathbb{R}^d \backslash\{x\}$,问题是
有一个点在$x$的概率是多大。这两个概念相互对偶。

如果点过程是原子的,就像die process一样,在其中一个原子的位置上调节一个点不会造成任何困难。

\begin{example}
    用$P_o$表示Example 2.1中die process的分布做原点$o\in \mathbb{R}^2$的condition.
    让$E$是所有点$\Phi(\mathbb{R}^2)$的数目是奇数这一事件。我们有$P(E)=1/2$但是$P_o(E)=1$.
    相似地,用$\mathbb{E}_o$指代期望,我们有$\mathbb{E} \Phi(b((1,1), \in))=1 / 2\text{ and } \mathbb{E}_o \Phi(b((1,1), \in))=1 / 3 \text { for } 0<\in<1$
\end{example}

当我们考虑扩散点过程时,(??括号里的谁爱翻译谁翻译,我看不懂??),因为,
在这个case,我们做condition的事件具有概率$0$. 甚至,这样的conditioning可以是
非常直觉性的,正如下面这个例子所展示的。

\begin{example}
    考虑概率空间$\left(\Omega, \mathcal{B}^2, v_2\right)$
\end{example}

\section*{练习题}

\begin{exercise}
    让$\Phi$是一个随机旋转的die processs,来自例子2.1,
    让$B=[-1,1]^2$。确定$\mathbb{E} \Phi(B), \mathbb{E}_o \Phi(B), \text { and } \mathbb{E}_{(-1,1)} \Phi(B)$
\end{exercise}

\begin{exercise}
    the die cluster process.考虑一个parent intensity为$\lambda_p$的泊松聚簇过程,其中
    每个簇是一个独立且随机的旋转die process.
    再一次的,对于$B=[-1,1]^2$,找到$\mathbb{E} \Phi(B), \mathbb{E}_o \Phi(B), \text { and } \mathbb{E}_{(-1,1)} \Phi(B)$,
    画出K函数$K(r)$图像.
\end{exercise}

\begin{exercise}
    考虑$\mathbb{R}^2$上的混合泊松点过程,其中$\lambda$是一个均值$p$的伯努利随机变量。
    确定$\Lambda\left([0,1]^2\right)=\mathbb{E} \Phi\left([0,1]^2\right)$和
    $\mathbb{E}_o \Phi\left([0,1]^2\right)$
\end{exercise}

\begin{exercise}
    $\mathcal{K}(\mathrm{B})=\mathcal{K}(-\mathrm{B})$成立吗?给出证据或者给出反例。
\end{exercise}


\begin{exercise}
    证明,对于一般的PPP,$\lambda(x, \Phi) \equiv \lambda(x)$
\end{exercise}

\begin{exercise}
    画出式子(8.10) Georgii-Nguyen-Zessin formula的稳定的(形式/表格?),

    使用一个hard-core process.
    选择一个函数$h$,使得内部的左边的conditioning
    和Scaling the intensity有相同的效果
\end{exercise}

\begin{exercise}
    找到Neymann-Scott聚簇过程的$J$函数,使用聚簇Palm测度$\mathcal{C}_o$来表示。
\end{exercise}