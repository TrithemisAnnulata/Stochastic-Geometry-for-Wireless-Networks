\chapter{引言}

\section{什么是随机几何?}
\section{无线网络空间模型的一种——点过程}







\section*{练习题}
\begin{exercise}
    下面哪一个说法正确?
\end{exercise}


\begin{exercise}
    让$X_1,X_2,\cdots,X_k$是iid. 的随机变量,他们的CDF都是$F(x)$。
    最小值$\operatorname{min}_{i}\{ X_i \}$的cdf?
\end{exercise}


\begin{exercise}
    让$X$是一个参数为$Y$的随机变量,其中$Y$本身是一个参数为$\mu$的泊松随机变量。
    说明$X+Y$的生成函数是
    \begin{equation*}
        G_{X+Y}(x) \triangleq \mathbb{E}\left(x^{X+Y}\right)=\exp \left\{\mu\left(x e^{x-1}-1\right)\right\}
    \end{equation*}
\end{exercise}

\begin{exercise}
    让$X_1,\cdots,X_N$是$N$个iid.的随机变量,他们的cdf都是$F(x)$,其中
    $N$是均值为$\mu$的泊松随机变量。计算这$N$个随机变量的最大值的cdf$G(x)$
\end{exercise}


\begin{exercise}
    让$Z$是在半径为$a$的圆盘随机独立选取2个点的距离。
    说明$\mathbb{Z^2}=a^2$
\end{exercise}


\begin{exercise}
    让$a,b,c$是iid.的指数分布变量。说明,多项式$ax^2+bx+c$有实数根的概率是$1/3$
\end{exercise}

\begin{exercise}
    一只鹅下了$N$个蛋,其中$N$是均值为$\lambda$的泊松随机变量。每一个蛋以概率$p$被孵化,
    都是彼此独立的。让$K$是幼鹅的数量。
    计算$\mathbb{E}(K|N)$ ,$\mathbb{E}(K)$,$\mathbb{E}(N|K)$
\end{exercise}


