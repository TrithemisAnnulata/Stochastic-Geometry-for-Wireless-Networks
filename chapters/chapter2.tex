\chapter{点过程的描述}

\section{一维点过程的描述}






\section*{练习题}
\begin{exercise}
    对于$\mathbb{R}$上的强度为$\lambda$的均匀PPP,推导interarrival
    interval包括原点$o$的pdf。你观察了什么?
\end{exercise}


\begin{exercise}
    让$X_m$是一个均值为$m$的泊松随机变量。如果$M$是一个服从Gamma分布
    的随机变量,说明$X_M$的分布是negative binomial.
\end{exercise}

\begin{exercise}
    让$\Phi$是$\mathbb{R}$上一个强度为$\lambda$的均匀PPP,确定
    $\mathbb{P}( \Phi(B_1)=n_1,\Phi(B_2)=n_2)\text{ for }B_1=[0,2]\text{ and }B_2=[1,3]$
\end{exercise}

\begin{exercise}
    让$(X_i),i\in \mathbb{N}$,是一族独立的均值为$1/\lambda$的指数分布的随机变量.
    确定
    \begin{equation*}
        \mathbb{P}\left(X_1>t\right)+\sum_{k=1}^{\infty} \mathbb{P}\left(X_1+\cdots+X_k \leq s ; X_1+\cdots+X_{k+1}>t\right)
    \end{equation*}
    给出结果的解释,作为一个特定的点过程的void probability.
\end{exercise}

\begin{exercise}
    结果空间$\mathcal{N}$可以定义为如下公式吗?
    \begin{equation*}
        \mathcal{N} \triangleq\left\{\varphi \subset \mathbb{R}^d:|\varphi|=0\right\} ?
    \end{equation*}
\end{exercise}

\begin{exercise}
    径向仿真。考虑正实数线上,一个强度为$1$的一维PPP$\Psi=\left\{y_i\right\} \subset \mathbb{R}^{+}$。
    假定点是有序的,即,$y_1<y_2<\cdots$。考虑$\Psi$,
    我们想要产生$\mathbb{R}^2$上强度为$\lambda$的HPPP $\Phi=\left\{\left(r_i, \varnothing_i\right)\right\}$的一个实现,
    其中$(r_i,\phi_i)$是点的极坐标表示。振幅集合$\{ r_i\}$也应该是有序的。
    从$y_i$到$r_i$找到函数,可以将一维度的点映射到二维度的振幅,with the desired intensity.
\end{exercise}


\begin{exercise}
    生成函数。让$X_1,X_2,\cdots$是一系列iid的随机变量$\Omega \mathbb{N}_0$
    的生成函数$G_X$.让$N$是一个随机变量,和$X_i$独立,
    also taking values in $\mathbb{N}_0$. 展示
    \begin{equation*}
        S=X_1+X_2+\cdots+X_N
    \end{equation*}
    $S$具有生成函数$F_S(s)=G_N(G_X(s))$以及导数$G_S(s)$如果$N$是均值为$\lambda$的泊松随机变量。
\end{exercise}

\begin{exercise}
    类比(2.3),找到一个表示,来计数二维ranomly translated lattice的测度。
\end{exercise}

\begin{exercise}
    确定$\Lambda(\{ x\})$对于所有的$x\in \mathbb{R}^d$对于
    例子2.1定义的点过程。
\end{exercise}

\begin{exercise}
    写下die point process的显式分布
\end{exercise}

\begin{exercise}
    例子2.3中的点过程,是两个(甚至更多)的基础点过程的“叠加”superpositionn.
    找到他们,并且验证$P=P_1*P_2$,使用卷积的标准准则。
\end{exercise}

\begin{exercise}
    考虑Bernoulli die process,定义如下:从初始的die process开始,接着独立地以概率$q$移走每一个点.
    这样一个点过程也可以称为a die process with erasures,with erasure probability $q$.
    定义$\Lambda(\{ o \})$和$\Lambda(\mathbb{R}^2)$
\end{exercise}


\begin{exercise}
    让$s\Phi_{\text{die}}$是the die process $\Phi_{\text{die}}$ scaled by $s$.
    让
    \begin{equation*}
        \Phi^{\prime}=\bigcup\left\{x \in \mathbb{Z}^2: x+\frac{1}{3} \Phi_{\text {die }}\right\}
    \end{equation*}
    
    其中$x+\frac{1}{3}\Phi_{\text{die}}$是 a translation of the (scaled) die process by $x$,
    考虑静态点过程$\Phi \triangleq \Phi_Y^{\prime}$,其中
    $Y$是$[0,1]^2$上的均匀分布。找到$\Phi$的强度。
\end{exercise}


\begin{exercise}
    让$\Phi\subset \mathbb{R} $是一个强度函数为$\lambda(x)=\mathbf{1}(x \in[0,1])$的均匀PPP。
    如果我们独立地对每一个点做displacing by a random amount uniform on $[-1/2,1/2]$,
    得到的点过程的强度是多少?
\end{exercise}


\begin{exercise}
    让$\Phi=\{x_1,\cdots \}\subset \mathbb{R}$是一个强度为
    $\lambda(x)=\mathbf{1}(0\leq x<2\pi)$的均匀PPP。
    让$\Phi^\prime=\{\sin x_i \}$.  确定$\Phi^\prime$的强度$\lambda^\prime$
\end{exercise}


\begin{exercise}
    让$\Phi$是一个高斯泊松过程,$\sigma=1$,$n=1$,让$B^{-}$是$\mathbb{R}^2$
    的左半平面。对于$\Phi$以及对于$\Phi_{(1,0)}$确定
    $\Lambda(B^{-})$.
    提示,关联问题的第二部分和点对点问题的误比特率问题
\end{exercise}

\begin{exercise}
    图2.2展示了确定的non-HPPP的实现。确定这个过程的强度测度$\Lambda([0,10]^2)$
\end{exercise}

\begin{exercise}
    随机traslated lattice (看定义2.15)是遍历的吗?
\end{exercise}


\begin{exercise}
    让$\Phi$是$\mathbb{R}^2$上一个强度1的HPPP。计算满足
    $NN(NN(x))=x$的点的比例。这些是属于互相最近邻点对的节点。
    这个比例依赖强度吗?同时计算,对于lattice,这个比例是多少?
\end{exercise}


\begin{exercise}
    让$\Phi$指代randomly translated square integer lattice

    (a)展示强度$\lambda \equiv 1$
    (b)让$B_{\diamond}$是面积为$A,1<A<2$的方形区域,旋转了$\pi/4$,使得
    相对的角点,他们的x坐标(或者y坐标)相等。找到$P(\Phi(B_{\diamond})=0)$,
    以及$P(\Phi(B_{\diamond})>2)$,都写成$A$的函数的形式。
\end{exercise}


\begin{exercise}
    我们想要产生一个isotropic PPP $\Phi\subset \mathbb{R}^2$的实现,此PPP的强度是
    \begin{equation*}
        \lambda(x)=\frac{M}{2 \pi \sigma^2} \exp \left(-\frac{\|x\|^2}{2 \sigma^2}\right), \quad x \in \mathbb{R}^2 .
    \end{equation*}

    Simulation的区域是$W=[-5,5]^2$。
    (a)
    (b)
    (c)

\end{exercise}

\begin{exercise}
    有向空空间函数。让$\Phi\subset \mathbb{R}^2$是一个强度为1的PPP。
    让$\{ (r_i,\phi_i) \}$是点集的极坐标表示。让$n\in \mathbb{N}$,定义
    \begin{equation*}
        R_k=\min \left\{\left(r_i, \phi_i\right) \in \Phi, k \frac{2 \pi}{n}<\phi_i \leq(k+1) \frac{2 \pi}{n}: r_i\right\}, \quad k=0,1, \ldots, n-1
    \end{equation*}

    $R_k$是最近邻节点和原点的距离,在kth sector of width $2\pi/n$

    我们感兴趣$R_k$中最小值的CDF。不计算,你估计这个CDF有哪些性质?
    
    为了验证你算出来的CDF,使用Problem1.2中得到的结果。
\end{exercise}

\begin{exercise}
    描述具有图2.10所示Fry Plot的PPP。
\end{exercise}