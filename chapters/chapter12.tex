\chapter{连通性}

\section{导论}
在一些case中,percolation渗流并不是充分的,因为它仅仅保证无穷数量的节点在图或网络的某个地方连通。
事实上,连通节点的比例可能非常小;很容易会出现:大部分的节点和the infinite component不连通。
连通性是一个更加严格的条件;它要求所有节点都连接almost surely。

因为连通性参数,例如圆盘的半径,经常是无穷的(对于无穷图),我们首先聚焦在infinite graph,
比方说一个面积为$n$的方形区域,接着研究图在$n\rightarrow \infty$的连通性行为。

我们首先讨论的是随机梯格的连通性

\subsection{随机梯格的连通性}
我们也考虑网格$\mathbb{L}^2$上一个大小为$n\times n$个顶点的盒子$B(x)$。
正如bond percolation中的那样,每一条边以概率$p$发生。我们想要找到
$p$要满足什么条件,使得$B(n)$的所有顶点是联通的,在$n\rightarrow\infty$的渐进情况下。
我们建立了$\mathbb{L}^2$上bond percolation的临界概率$p_c=1/2$。
会议$\theta(p)$指代原点$o$属于the infinite component的概率。

如果$p>1/2$,在网格上存在一个唯一的infinite component,$B(n)$的每一个顶点以概率$\theta(p)$连接上它。
因此在$B(n)$存在一个component具备mean size$n^2\theta(p)$。
事实上,可以得到,在极限$n\rightarrow \infty$下,$\theta(p)$正好是
$B(n)$中连通节点的比例。

因此,对于full connectivity,必须$p\rightarrow 1$.问题是在$n$增长时,大概要多大的n就基本满足。
让$p$是$n$的函数,平均孤立节点数量的渐进是
\begin{equation*}
    \mathbb{E} N_{\text {isol }}(n) \sim n^2(1-p(n))^4, \quad n \rightarrow \infty
\end{equation*}

因为每个(内部的)节点是以概率$(1-p(n))^4$孤立的。边界节点是渐进可忽略的,因为他们的数量是
$\Theta(n) $.选择
\begin{equation*}
    p(n)=1-\frac{c(n)}{\sqrt{n}}
\end{equation*}

导致$\mathbb{E}N_{isol}(n)\sim c(n)^4$。所以对于连通性$c(n)\rightarrow 0$是必须的。
下面的定理给了一个更清晰的结果。

\begin{theorem}
    (随机梯格中的孤立节点) 让$\mathbb{E}N_{isol}(n)$是$B(n)$中孤立节点的数量。那么$\mathbb{E}N_{isol}(n)$收敛到
    一个均值为$M$的泊松随机变量,当且仅当
    \begin{equation*}
        n^2(1-p(n))^4 \rightarrow M, \quad n \rightarrow \infty
    \end{equation*}
\end{theorem}

\begin{proof}
    我们使用所谓的Chen-Stein方法来展示孤立节点的数量在给定条件下收敛到一个泊松分布。
    我们要应用的特定结果如下
\end{proof}

\begin{lemma}
    (总共variaiton distance的Chen-Stein界) 让$(A_i),i\in [n]$是
    指示随机变量的collection。让$S=\Sigma_{i=1}^n A_i$,其中$\mathbb{E}A_i=p_i$,以及
    \begin{equation*}
        M \triangleq \sum_{i=1}^n p_i
    \end{equation*}
    
    在$M<\infty$的假设下。因此$\mathbb{E}S=M$。那么,如果$A_i$是随机变量$(X_k)$的增函数
    \begin{equation*}
        \|S-\operatorname{Poi}(M)\|_{\mathrm{TV}} \leq \frac{1-\mathrm{e}^{-M}}{M}\left(\operatorname{var} S-M+2 \sum_{i=1}^n p_i^2\right)
    \end{equation*}
    其中$\operatorname{Poi}(M)$是均值为$M$的泊松随机变量,
    $\|\cdot\|_{\mathrm{TV}}$是total variation distance,定义是
    \begin{equation*}
        \|P-Q\|_{\mathrm{TV}} \triangleq \sup \{Y \in \mathcal{F}:|\mathbb{P}(Q \in Y)-\mathbb{P}(P \in Y)|\}
    \end{equation*}

    对2个随机变量$P$和$Q$定义相同的概率空间with $\sigma$- algebra $\mathcal{F}$
\end{lemma}

在我们的情况中,潜在的独立随机变量$(X_k)$是


todo todo 
todo todo
todo todo
todo todo


\section*{练习题}

\begin{exercise}
    不使用本章的任何结果,说明:对于圆盘图,
    \begin{equation*}
        \pi r^2(n)=5\pi \log n
    \end{equation*}
    是a.a.s. connectivity的充分条件。

    提示,面积为$n$的方形区域的分拆为一些面积为$\sqrt{\log n}$的方形区域。
    在每个方形上定义一个事件,说明所有事件发生的概率趋于1,在$n\rightarrow \infty$的条件下。
    接着证明这个联合事件表明:在给定的$r(n)$的条件下,可以得到连通性。
\end{exercise}

\begin{exercise}
    估计在定理12.8的下界的讨论中,"likely scenario for disconnectedness"事件的概率。
    说明是$k(n)>\log n/8$是必要的。
\end{exercise}

\begin{exercise}
    考虑1st最近邻图,顶点集为$\Phi$,其中$\Phi$是$\mathbb{R}^d$一个强度为1的PPP。
    对于一般的维度数$d$,描述它的成分的性质。
    接管关注$d=1$的case,陈述典型成分的大小的分布。
\end{exercise}

\begin{exercise}
    考虑一个$\sigma^2=1/2$的高斯PPP(见例子2.2),所以强度是$(n/\pi)e^{-r^2}$.在原点加上一个点,
    通过连接距离在$s$以内的2个点定义圆盘图。让$\rho$是最大的半径$u$使得所有的在$b(o,u)$内部的点和
    $o$连接。说明,假设$Rs<<1$,球$b(o,R)$内的孤立节点数量可以近似为
    \begin{equation*}
        \mathbb{E} N_{\mathrm{isol}}(R) \approx \frac{\exp \left(-n s^2 \mathrm{e}^{-R^2}\right)-\exp \left(-n s^2\right)}{s^2} .
    \end{equation*}
    接着,假设$n\rightarrow \infty$以及$n^{-1 / 2} \ll s \ll(\log n)^{-1 / 2}$,说明
    \begin{equation*}
        \rho \approx \log \left(n s^2 / \log \left(1 / s^2\right)\right)
    \end{equation*}
\end{exercise}

