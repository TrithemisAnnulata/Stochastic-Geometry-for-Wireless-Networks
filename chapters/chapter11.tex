\chapter{随机几何图和连续渗流}

\section{导论}
在这一章,我们介绍随机几何图(RGGs)作为之前章节考察过的lattice model的一般化。

RGG的顶点嵌入在$\mathbb{R}^d$,边依赖于顶点的空间位置。

位置经常使用点过程建模,考虑顶点的位置确定性地或概率性地建立边。
我们将会术语vertex 和 point 作为等价术语。
对于一个特定的图,我们将会限制渗流概率。因为顶点可以保证任意的位置在相同的空间(经常是$\mathbb{R}^d$,称为continuum percolation。


\begin{definition}
    (基础随机几何图RGG) 让$\Phi= \{ x_i \} \subset  \mathbb{R}^d $ 是一个(简单的)点过程。
    对于每一个(无序的)点对 $\{ x,y \} \subset \Phi$,独立地以概率$\xi(x-y)$加入一条边,其中$\xi : \mathbb{R}^d \to [0,1]$是(translation-invariantt)连通函数。 $\xi$的有效面积(or volume)是 
    \begin{equation*}
        \mathcal{A}_\xi  \triangleq \int \limits_{\mathbb{R}^d} \xi(y) \mathrm{d} y
    \end{equation*}
    得到的图$G=(\Phi, E)$是一个随机几何图。
\end{definition}

因此在RGG里有两个随机因素,点的位置,以及边是否存在。点过程可以是finite或者infinite。如果它是finite,一个经常使用的模型是(广义的)binomial点过程,经常 on  the unit cube $[0,1]^d$。如果它是infinite,典型的模型是PPP(泊松点过程)。








todo todo todo todo
todo todo todo todo
todo todo todo todo
todo todo todo todo

\section*{练习题}

\begin{exercise}
    利用和a Galton–Watson 分支过程的比较,找到Gilbert圆盘图的渗流概率$\theta(r,\lambda)$的数值的界
\end{exercise}   

\begin{exercise}
    对于Gilbert随机图,说明$p=n^{-1-1/k}$是出现$k+1$个节点的树的阈值函数
\end{exercise}

\begin{exercise}
    在Gilber随机图中,说明,如果$np<1$,包括顶点1的成分的期望大小的界是
    \begin{equation*}
        \mathbb{E}|C(1)| \leq \frac{1}{1-n p}
    \end{equation*}
\end{exercise}

\begin{exercise}
    对于空时SINR图,找到一个顶点入度的上界,此上界仅依赖门限$\theta$
\end{exercise}

\begin{exercise}
    确定空时圆盘干扰快照图中一个接收机的平均入度。一个发射机的出度由引理11.19确定。使相似的计算
    来找到一条“捷径”,这样你需要很少的计算来得到答案。
\end{exercise}

\begin{exercise}
    对于Poisson隐私图,说明,入度$N_{in}$有矩生成函数mgf
    \begin{equation*}
        \mathbb{E} \mathrm{e}^{t N_{\text {in }}}=\mathbb{E}\left(\exp \left(V\left(\mathrm{e}^t-1\right) / \lambda\right)\right)
    \end{equation*}
    其中$V$是和强度为1的PPP典型点关联的Voronoi cell的体积。
    对于一维的例子,给出mgf的具体的表达式,并且确定$\mathbb{P}(N_{in}=k)$
\end{exercise}

\begin{exercise}
    对于Possion隐私图上的out-percolation,说明
    \begin{equation*}
        \theta(\lambda) \leq \max \{0,1-\lambda\}
    \end{equation*}
    
    其中$\theta(\lambda)$是$\vec{G}\lambda$中原点属于一个无穷out-component的概率。
\end{exercise}


\begin{exercise}
    对于无向Poission隐私图$G_\lambda$,找到临界强度的上界。利用一个face percolation argument。
    原点处的component是finite如果它被circuit of closed faces包围。
    定义a face 是 closed,如果这个configuration包括监听者,满足s it impossible for an edge to
    form from the inside of the circuit to the outside。
    如果那个configuration以至少$1/2$概率发生,$o$处component是finite a.s. 
    注意得到的界是相对弱的。
\end{exercise}