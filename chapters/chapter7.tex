
%%%%%%%%%%%%%%%%%%%%%%%%%%%
\chapter{标记点过程}

默认中文字体
default english font

\section*{练习题}
\begin{exercise}
    (对于平稳过程)定义$K$函数,使用一般的记号$m_x^{(\Phi)}$
\end{exercise}

\begin{exercise}
    确定例子7.8中的标记点过程的the mark distrbution,
    如果$\Phi$是一个强度为$\lambda$的均匀PPP。
    Repeat for the case where Φis a
perturbed square lattice (see Definition 2.16) with iid perturbation vectors that are
uniform on b(o, 1/2)
\end{exercise}


\begin{exercise}
    让$\Phi$是一个强度为$\lambda$的PPP,用标记
    \begin{equation*}
        m_x=\Phi(b(x, R))-1, \quad x \in \Phi
    \end{equation*}
    
    其中$R>0$。来得到一个标记点过程。展示mark-correlation函数是
    \begin{equation*}
        k_{m m}(r)=1+\frac{|b(o, R) \cap b(r, R)|}{\pi r^2}
    \end{equation*}
\end{exercise}


\begin{exercise}
    对于式子7.11描述的标记过程,描述$\Phi_{[[0 p] \times\{1\}]}$的性质。
    
    如果基底过程$\Phi$是一个强度$\lambda$的PPP。
    什么MAC制式可以得到这个过程?
\end{exercise}

\begin{exercise}
    给出Campbell’s theorem的形式证明。
    这里考察的对象是(7.8)中的稳定标记点过程。
\end{exercise}


\begin{exercise}
    描述泊松双极网络模型中的发射机和接收机的叠加,是一个聚簇过程。
\end{exercise}

\begin{exercise}
    考虑一个无线网络,其中发射机来自一个稳定的强度为$\lambda$的点过程。todo  todo
\end{exercise}