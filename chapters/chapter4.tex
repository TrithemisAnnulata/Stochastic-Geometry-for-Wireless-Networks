\chapter{点过程上的和与积}

\section{引言}
\subsection{动机}
\subsection{记号}





\section*{练习题}
\begin{exercise}
    使用显式的计算矩生成函数的导数的方法,完成引理4.8的证明。
\end{exercise}

\begin{exercise}
    让$v(x)=\min \left\{1, \frac{1}{2}\|x\|^2\right\}$,
    从例子2.1中确定die process的the pgfl $G[v]$
\end{exercise}


\begin{exercise}
    比较$\mathbb{E}e^{-s[f]}$和$\exp (-\mathbb{E}S[f])$
\end{exercise}

\begin{exercise}
    展示泊松hole过程的pgfl。
    例子3.7中介绍的那个possion hole process,
    参数是$\lambda_1,\lambda_2,r$
    \begin{equation*}
        G[v]=\mathrm{E}\left[\exp \left(-\lambda_2 \int_{\mathrm{R}^2 \backslash \Xi_r}(1-v(x)) \mathrm{d} x\right)\right] .
    \end{equation*}

    如果$v=1-\mathbf{1}_B(x), \lambda_1=1, \text { and } r=1$,
    确定$G[v]$。

    这产生了void probability.
\end{exercise}



\begin{exercise}
    隐私图的定义如下。
\end{exercise}


\begin{exercise}
    让$\Phi$是一个泊松高斯过程,其中representative cluster是
    $\Phi_0=\{ o \}$ with probability $1-p$,$\Phi_0=\{ o,x \}$ with probability $p$.
    其中$x$以密度$f$分布。
    假定我们选择的密度正好可以让$\mathbb{P}(x=o)=0$。展示pgfl是
    \begin{equation*}
        G[v]=\exp \left(\lambda_{\mathrm{p}} \int_{\mathbf{R}^d}\left((1-p) v(x)+p v(x) \int_{\mathbf{R}^d} v(x+y) f(y) \mathrm{d} y-1\right) \mathrm{d} x\right) .
    \end{equation*}

    所以$v \in v$  意味着 $1-v(x)$在有界集外消失。
\end{exercise}
