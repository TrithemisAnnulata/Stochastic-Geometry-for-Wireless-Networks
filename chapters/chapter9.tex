
\chapter{渗流,连通性,覆盖的介绍}


\section{动机}

在这部分,我们关心一个网络或者一个图的特定的全局性质,例如是否存在
一个巨大的连通分量。我们将要用到的主要的新的数学工具是渗流理论。
渗流理论开始于大约50年前,当时是作为一个数学的框架去研究孔状媒质的行为。
它试图回答下面这些问题:
- 如果一个石头变得潮湿了,水会渗过石头吗?
- 如果一个材料包括2种成分,其中之一是完美绝缘体,resistance是有限的概率是?
- 一场森林大火蔓延到整个森林的概率?
- 一种病毒全球传播的概率?
- 大多数的网络连通的概率?

事实证明,当网络参数发生变化时,某些关键事件(如网络中存在巨大的连接组件)会突然出现。这种现象称为相变。
在图形或网络的上下文中,渗流与连通性和覆盖率有关。这是本部分的其他两个主题。



\section{什么是渗流?}
渗流的基础模型是:二维的网格,即,图$\mathbb{L}^2 \triangleq G(V,E)$,其中$V=\mathbb{Z}^2, E= \{ (x,y) \in V^2 : ||x-y||=1 \}$。所以$\text{deg} \nu =4 \; \nu \in V$。
现在,我们通过独立地以概率$1-p$移除每一条边,把格 转为 随机图。
等价地,我们每条边保留下来的概率是$p$。
这就是所谓的bond 渗流模型。
如果不是边而是顶点的情形,被称为site 渗流模型。
我们需要的其他的成分是 连通分量。对于一般的图来说,连通分量的定义如下:







todo todo todo todo
todo todo todo todo
todo todo todo todo
todo todo todo todo


