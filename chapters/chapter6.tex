
%%%%%%%%%%%%%%%%%%%%%%%%%%%
\chapter{点过程的矩测度}


\section*{练习题}
\begin{exercise}
    说明如下公式:
    \begin{equation*}
        \operatorname{var} \Phi(A)=\alpha^{(2)}(A \times A)+\Lambda(A)-(\Lambda(A))^2        
    \end{equation*}
\end{exercise}

\begin{exercise}
    使用R语言验证例子6.15中所使用近似的准确性
\end{exercise}

\begin{exercise}
    使用R中的`x=clickppp(30)`来手动进入30个点。尝试
    产生$[0,1]^2$上均匀PPP的一个典型实现。使用Kest函数和envelope函数来
    验证是否实际$K$函数和理论$K$函数匹配
\end{exercise}


\begin{exercise}
    找到Matérn聚簇过程的$K$函数
\end{exercise}


\begin{exercise}
    说明对于双泊松聚簇过程,
    \begin{equation*}
        K(r)=\pi r^2+\frac{F(r)}{\lambda_{\mathrm{p}}}
    \end{equation*}

    其中$F(r)$是簇中两个点的距离的分布
\end{exercise}

\begin{exercise}
    考察例子6.17中的Baddeley–Silverman process,设置区域为单位方,
    计算它的强度函数,方差$\text{var} \Phi(B)$对于任意的$B$,K函数。

\end{exercise}

\begin{exercise}
    解释2.8节引入的FryPlot和$K$函数的联系。
\end{exercise}


\begin{exercise}
    干扰。重复问题5.2,使用你自己的关于二阶积密度的知识。
    显式地写出和的平方:
    \begin{equation*}
        I^2=\left(\sum_{x \in \Phi} h_x \ell(x)\right)^2, \quad \ell(x)=\min \left\{\|x\|^{-\alpha}, 1\right\},
    \end{equation*}
\end{exercise}


\begin{exercise}
    一维聚簇过程的二阶矩测度。
    (a)
    (b)
    (c)
\end{exercise}