\chapter{点过程模型}

\section{引言}
\section{广义有限点过程}




\section*{练习题}
\begin{exercise}
    对于例子3.1的binomial点过程,确定Janossy测度和Janossy密度。
\end{exercise}


\begin{exercise}
    考虑有限点过程$\Phi\subset \mathbb{R}^2$,概率$p_1=1/2,p_2=1/2$
    ,以及
    \begin{equation*}
        f_1(x)=1_{W^+}(x)
    \end{equation*}
    \begin{equation*}
        f_2\left(x_1, x_2\right)=\left(\frac{1}{3} 1_{W-}\left(x_1\right)+\frac{2}{3} 1_{W+}\left(x_1\right)\right)\left(\frac{1}{3} 1_{W-}\left(x_2\right)+\frac{2}{3} 1_{W+\left(x_2\right)}\right)
    \end{equation*}

    对于$W^+=[0,1]^2$,$W^-=[-1,0]\times[0,1]$。
    找到Janossy密度$j_1,j_2$,计算$\mathbb{E}u(\Phi)$对于
    $u_1(\Phi)=\Phi(B)$(对于一般的$B\subset \mathbb{R}^2$),
    $u_2(\Phi)=2^{\Phi(\mathbb{R}^2)}$   ,使用公式(3.3)
\end{exercise}


\begin{exercise}
    找到一个广义的有限点过程的强度测度,写成$(p_k)$和$F_n$函数的形式,其中$(p_k)$和$F_n$的定义在Definition 3.1可以找到
\end{exercise}


\begin{exercise}
    考虑一个一维的Cox过程,它是这么得到的:通过使用例子3.6中的random thinning,
    其中基础的PPP是均匀的,强度为$\lambda_b$,随机场$T$由式子$T(x)=e^{-L|x|}$给出,
    $L$是一个随机变量。
    
    (a)让$\mathbb{P}(L=1)=3/4$以及$\mathbb{P}(L=2)=1/4$。确定$\Lambda([0,y))$,对于所有的
    $y\in \mathbb{R}$,$\lambda(\mathbb{R})$,以及强度函数在$0,\lambda(0)$处的值。

    (b)让$L$的pdf是$f_L(x)=2 x \mathbf{1}(0 \leq x \leq 1 )$。再一次,确定$\Lambda([0,y))$,对于所有的
    $y\in \mathbb{R}$,$\lambda(\mathbb{R})$,以及强度函数在$0,\lambda(0)$处的值。

\end{exercise}

\begin{exercise}
    对于例子3.7中的germ-grain模型,考虑这样的Cox过程,它的强度场如下:
    \begin{equation*}
        \zeta(x)=\lambda_a \mathbf{1}\left(x \in \Xi_r\right)+\lambda_b \mathbf{1}\left(x \notin \Xi_r\right) .v
    \end{equation*}

    确定过程的强度,展示$Matern$聚簇过程可以使用这个强度场建模。
\end{exercise}


\begin{exercise}
    考虑Gauss-Poisson过程,$p_2=1$(所有的簇正好有2个点),parent intensity是$\lambda$.
    我们想要计算零空间函数$F(r)$。如果我们认为聚簇过程是【一系列簇中心$\Phi_c$的过程】和【一系列其他点$\Phi_1$的过程】的superposition,
    都是强度为$\lambda$的均匀PPP。他们是依赖的,因为每一个$\Phi_c$中的点来说,都有$\Phi_1$中一个点和它距离1,随机朝向。
    说明零空间函数可以写成
    \begin{equation*}
        \begin{aligned}
            1-F(r) & =\mathbb{P}\left(\Phi_c(b(o, r))=0, \Phi_1(b(o, r))=0\right) \\
            & =\mathbb{P}\left(\Phi_c(b(o, r))=0\right) \mathbb{P}\left(\Phi_1(b(o, r))=0 \mid \Phi_c(b(o, r))=0\right)
            \end{aligned}
    \end{equation*}

    接着确定$F(r)$对于$r\leq 1/2$
    
    对于$r>1/2$,说明
    \begin{equation*}
        1-F(r)=\mathrm{e}^{-\lambda \pi r^2} \exp \left(-\lambda \int_r^{r+1} \frac{2 \arccos \left[\left(1+y^2-r^2\right) /(2 y)\right]}{2 \pi} 2 \pi y \mathrm{~d} y\right) .
    \end{equation*}

\end{exercise}

\begin{exercise}
    在R中仿真泊松hole process,验证强度。
\end{exercise}

\begin{exercise}
    对于Gibbs过程,通过对$P_{\Phi}(Y_k)$和$P_\lambda(Y_K)$的显式的计算(见3.9式),
    说明两个过程的void probability一致。
\end{exercise}


\begin{exercise}
    证明式子3.22。提示:见例子3.3
\end{exercise}

\begin{exercise}
    找到和Strauss hard-core过程(点的数量固定为$n$)对应的
    pair potential $\theta$,对于给定的$R$和$R^{\prime}$
\end{exercise}

\begin{exercise}
    在图3.8中,我们观察到所有的簇都有2个或者3个点。解释这个现象。
\end{exercise}

\begin{exercise}
    对于一般的finite Gibbs process,密度$f(\varphi)$可以表示为
    \begin{equation*}
        f\left(\varphi_n\right)=\mathrm{P}\left(Y_n\right) f_n\left(x_1, \ldots, x_n\right), n \in \mathbb{N}_0
    \end{equation*}
    吗?其中$ Y_n=\{\varphi \in \mathcal{N}: \varphi(W)=n\}$,
    $f_n$是已知过程有$n$个点,算条件密度。
\end{exercise}

\begin{exercise}
    描述和图3.11的FryPlot有关的点过程。
\end{exercise}