\chapter{覆盖}

\section{导论}
在几何的语境中和连通性相对的,是覆盖。
这里我们假设一个随机集被附属到点过程的每一个点上,
主要的问题是,是否所有这些随s机集的并覆盖一个特定的目标区域(可能$\mathbb{R}^d$)。


在无线网路的上下文中,这样的问题在蜂窝系统中自然而然地被提出,那里的目标是:
每个蜂窝基站或者传感器可以覆盖一定的区域(随机的),可能做到对我们感兴趣的区域的覆盖吗?

\section{Germ-grain模型和布尔模型}
\subsection{Germ-grain模型}
给出germ-grain model的定义。
点过程的点(germs)
iid的紧随机集(grains)
每个点(germ)都有一个附属的区域(grain)

\begin{definition}
    (Germ-grain模型)让$\Phi=\{ x_i \}$是$\mathbb{R}^d$上的点过程,称为germs.
    $(S_1,S_2,\cdots)$是一系列随机非空集的Collection,称为grains。那么并
    \begin{equation*}
        \Xi=\bigcup_{i \in \mathbb{N}} x_i+S_i
    \end{equation*}

    就是一个germ-grain model。$\Xi\subset \mathbb{R}^d$是占据区域,
    $\mathbb{R}^d\backslash\Xi$称为未占据区域,或者空区域。
\end{definition}

$(x_i+S_i)$被称为覆盖过程。一块区域$S$附属在一个点$x$上,所以下标也可以写$x$,像下面这样:
\begin{equation*}
    \Xi=\bigcup_{x \in \Phi} x+S_x \subset \mathbb{R}^d
\end{equation*}

一个点$y$被称为被覆盖,如果$y\in \Xi$.

如果集合$S_x$或者$S_i$是可数的,$\Xi$是聚簇点过程,正如3.4节描述的那样,
$|S|=v_d(S)=0$.希望覆盖一个不可数的$\mathbb{R}^d$的子集,我们需要
$|S|>0$来取得正概率。

\begin{definition}
    (Vacancy)vacancy$V(\mathcal{R})$是未被覆盖区域$\mathcal{R} \subseteq \mathbb{R}^d$的一部分的勒贝格测度
    \begin{equation*}
        V(\mathcal{R}) \triangleq|\mathcal{R} \backslash \Xi|=\int_{\mathcal{R}} \chi(y) \mathrm{d} y,
    \end{equation*}

    其他
    \begin{equation*}
        \chi(y)=1(y \notin \Xi)=\prod_i 1\left(y \notin S_i+x_i\right) .
    \end{equation*}

    每个$\Xi$的联通子集形成了component;在覆盖过程的语境中,components也可以说成clumps.

\end{definition}

\begin{definition}
    我们说,一个集$\mathcal{R}\subset \mathbb{R}^d$被区域$\Xi$覆盖,
    如果$\mathbb{P}(V=0)=1$
\end{definition}

\begin{note}
    条件$\mathbb{P}(V=0)=1$并不指示$\mathcal{R}\subset \Xi$,因为
    $\mathcal{R}$的一个低维度的子集可能被覆盖。所以,严格来说,$\mathbb{P}(V=0)=1$仅仅是$\mathcal{R}\subset \Xi$的充分条件。
    然而,如果grains $S$是闭集,区域$\mathcal{R}$是一个有界开集,这两者是等价的,即,
    $\mathbb{P}(V=0)=1\Leftrightarrow \mathcal{R}\backslash\Xi=\varnothing$
\end{note}

除了覆盖的问题,一个重要的问题是:是否要去覆盖的区域包括一个无穷大小的clump。
如果确实如此,覆盖过程被称为percolate。
connectivity,coverage和percolation的对应关系在Box13.1进行了总结。


大多数tractable的germ-grain模型是布尔模型,定义如下:
\begin{definition}
    (布尔模型)一个布尔模型首先得是一个germ-grain模型,germ点过程是uniform PPP,
    grains $S_i$是iid.的。
\end{definition}

有时,germ-grian模型中,germs是来自lattice的,那么我们称这个模型是
lattice Boolean model。一个典型布尔模型的case是,grains都是iid.的球。
对此,重要的结果是:每一个位置被覆盖的次数是泊松随机变量。

\begin{theorem}
    (布尔模型中的vacancy and coverage) 在PPP具有强度$\lambda$的布尔模型中,
    一个位置没有被覆盖的概率是$\exp (-\lambda \mathbb{E}|S|)$,它也是$\mathbb{R}^d$没有被占据的比例。
    紧接着,如果$\mathbb{E}|S|<\infty$,那么$\Xi\neq \mathbb{E}^d$. 进一步,每一个location覆盖的次数是均值为
    $\lambda\mathbb{E}|S|$的泊松随机变量。
\end{theorem}

\begin{proof}
    让$M$是原点被覆盖的次数
    \begin{equation}
        M=\sum_{x \in \Phi} 1\left(o \in x+S_x\right) .
    \end{equation}
    
    让$F(y)=\mathbb{P}(y\in S)$,有$\mathbb{E}|S|=\int_{\mathbb{R}^d} F(y) \mathrm{d} y .$。取期望,有
    \begin{equation*}
        \begin{aligned}
            \mathbb{E} M &=\mathbb{E}\left[\sum_{x \in \Phi} \mathbb{E}\left(1\left(o \in x+S_x\right) \mid \Phi\right)\right] \\
            &=\mathbb{E}\left[\sum_{x \in \Phi} F(-x)\right] \\
            & \stackrel{\text { (a) }}{=} \lambda \int_{\mathbb{R}^d} F(-y) \mathrm{d} y \\
            &=\lambda \mathbb{E}|S|
            \end{aligned}
    \end{equation*}

    其中(a)是因为Campbell定理,定理4.6;(b)是因为如下的公式:
    \begin{equation*}
        \mathrm{Ee}^{t 1\left(o \in x+S_x\right)}=1-F(-x)+e^t F(-x)
    \end{equation*}

    所以$M$的矩生成函数和泊松随机变量的矩生成函数形式一致,所以$M$服从泊松分布。
\end{proof}

这个定理的可以产生the 1st and 2nd moments of the vacancy.

\begin{corollary}
    (1st and 2nd moments of the vacancy) 对于布尔模型,vacancy的前2个矩可以由下面给出
    \begin{equation*}
        \mathbb{E}(V(\mathcal{R}))=|\mathcal{R}| \exp (-\lambda \mathbb{E}|S|)
    \end{equation*}
    
    以及
    \begin{equation*}
        \mathbf{E}\left(V(\mathcal{R})^2\right)=\iint_{\mathcal{R}^2} \exp \left(-2 \lambda \mathbb{E}|S|+\lambda \mathbb{E}\left(\left|\left(y_1-y_2+S\right) \cap S\right|\right)\right) \mathrm{d} y_1 \mathrm{~d} y_2 .
    \end{equation*}
\end{corollary}


\begin{proof}
    均值的推导是
    \begin{equation*}
        \mathbb{E} V=\int_{\mathcal{R}} \mathbb{P}(x \notin \Xi) \mathrm{d} x=\int_{\mathcal{R}} \mathbb{E} \chi(x) \mathrm{d} x=|\mathcal{R}| \exp (-\lambda \mathbb{E}|S|)
    \end{equation*}

    对于2nd moment,推导是
    \begin{equation*}
        \begin{aligned}
            \mathbb{E}\left(V^2\right) &=\iint_{\mathcal{R}^2} \mathbb{P}\left(y_1 \notin \Xi, y_2 \notin \Xi\right) \mathrm{d} y_1 \mathrm{~d} y_2 \\
            &=\iint_{\mathcal{R}^2} \mathbb{E}\left(\chi\left(y_1\right) \chi\left(y_2\right)\right) \mathrm{d} y_1 \mathrm{~d} y_2
            \end{aligned}
    \end{equation*}

    其中,
    \begin{equation}
        \begin{aligned}
            \mathbb{E}\left(\chi\left(y_1\right) \chi\left(y_2\right)\right) &=\mathbb{P}\left(\forall i: y_1 \notin x_i+S_i, y_2 \notin x_i+S_i\right) \\
            & \stackrel{(\mathrm{a})}{=} \mathbb{P}\left(\forall i: x_i \notin y_1-y_2+S_i, x_i \notin S_i\right) \\
            &=\mathbb{P}\left(\forall i: x_i \notin\left(y_1-y_2+S_i\right) \cup S_i\right) \\
            & \stackrel{\text { (b) }}{=} \exp \left(-\lambda \mathbb{E}\left(\left|\left(y_1-y_2+S\right) \cup S\right|\right)\right) \\
            &=\exp \left(-2 \lambda \mathbb{E}|S|+\lambda \mathbb{E}\left(\left|\left(y_1-y_2+S\right) \cap S\right|\right)\right)
            \end{aligned}
    \end{equation}

    其中(a)是因为平稳性,(b)是因为定理13.5
\end{proof}

如果$E(|S|)=\infty$,那么:如果$\lambda>0$,即,没有临界的密度。,那么每个component的germ数量是无穷的 a.s. 。
否则,主要问题是:是否$\Xi$中包括原点的component,无限的概率是正的。
(假设原点是PPP的一部分);以及,$\mathbb{R}^d$是否被覆盖。对于圆盘图来说,可以说明,the infinite component是唯一的a.s. 。

\section{具备固定圆盘的布尔模型}
\subsection{单个覆盖}
研究最多的是:把固定半径$r$的圆盘(或者球)当作grain的布尔模型。
在这个case中,我们写作$\Xi=\Phi \oplus S$,这个覆盖过程和Gilbert圆盘图等价。
事实上,从圆盘图,我们立马可以得到coverage percolation problem:如果$r>r_c/2$,其中$r_c$是
圆盘图上渗流的临界半径,布尔模型有一个infinite component,since then覆盖过程的2个圆盘重叠如果他们的顶点在圆盘图中连通。

这个模型的一个基础的应用是,传感器网络。
如果传感器能够检测半径$r$以内的事件,$\Xi$是传感器网络覆盖的区域。
Figure 13.1展示了这些布尔模型的2个例子。

我们关心于:在半径为$r$的圆盘上,面积为$a$的方形区域,在$n\rightarrow\infty$的时候被覆盖a.a.s. ,
需要的条件是什么。我们通过三步来解决这个问题。首先我们找到一个仅仅依赖mean vacancy的必需的条件。
第二步,我们使用vacancy的方差来在必需条件的基础上进行改善。第三步,我们展示这个第二必要条件也是必要的。

平面覆盖的第1必要条件

考虑2维的例子,最基础的问题是,$r$需要多大来为了保证整个平面被覆盖,因为我们可能总是放缩$\lambda$和$r$同时保持$\lambda r^2$为常数,
不失一般性,我们固定$\lambda=1$。

原点没被覆盖的概率是PPP中没有点在距离$r$以内,即,
\begin{equation*}
    \mathbb{E}_\chi(o)=\exp \left(-\pi r^2\right)
\end{equation*}

这对$\mathbb{R}^2$中的所有点都成立,所以面积$n$的方形区域的期望vacancy是
\begin{equation*}
    \mathbb{E} \mathrm{V}\left([0, \sqrt{n}]^2\right)=\mathbb{E}\left|[0, \sqrt{n}]^2 \backslash \Xi\right|=n \exp \left(-\pi r^2\right)
\end{equation*}

这也可以从Corollary 13.6得到。

覆盖的一个必要条件是:当$n\rightarrow\infty$的时候,$\mathbb{E}V([0,\sqrt{n}]^2)\rightarrow 0$. 这由下面这个式子保证:
\begin{equation}
    \pi r^2=\log n+\omega(1), \quad n \rightarrow \infty,
\end{equation}

它也是连通性的相同的条件。然而,$\mathbb{E}V\rightarrow 0$对于覆盖来说并不充分,因为
它并没有保证$\mathbb{P}(V=0)\rightarrow 1$. 
这个non-implication $\mathbb{E}V(n) \not \Rightarrow \mathbb{P}(V=0)\rightarrow 1$为这个事实提供了例子:
均值的收敛并不能表明a.s. 的收敛。

平面覆盖的第2必要条件

$\mathbb{P}(V=0)$的边界可以得到,如果vacancy的2nd moment(或者方差)也成为已知的。
写$V=V1(V>0)$,应用Cauchy-Schwartz不等式,我们有
\begin{equation*}
    \mathbb{E} V=\mathbb{E}(V 1(V>0)) \leq\left(\mathbb{E}\left(V^2\right) \mathbb{P}(V>0)\right)^{1 / 2},
\end{equation*}

因此
\begin{equation}
    \mathbb{P}(V>0) \geq \frac{(\mathbb{E} V)^2}{\mathbb{E}\left(V^2\right)}=1-\frac{\operatorname{var} V}{\mathbb{E}\left(V^2\right)} \quad \Longrightarrow \quad \mathbb{P}(V=0) \leq \frac{\operatorname{var} V}{\mathbb{E}\left(V^2\right)}
\end{equation}

因此我们需要去计算方差,我们可以使用Corollary 13.6 来计算方差,
vacancy的协方差可以从(13.2)算出,为
\begin{equation*}
    \operatorname{cov}\left(\chi\left(y_1\right) \chi\left(y_2\right)\right)=\exp (-2 \lambda \mathbb{E}|S|)\left(\exp \left(\lambda \mathbb{E}\left(\left|\left(y_1-y_2+S\right) \cap S\right|\right)\right)-1\right)
\end{equation*}

因此,
\begin{equation*}
    \begin{aligned}
        \operatorname{var} V(\mathcal{R}) &=\iint_{\mathcal{R}^2} \operatorname{cov}\left(\chi\left(x_1\right), \chi\left(x_2\right)\right) \mathrm{d} x_1 \mathrm{~d} x_2 \\
        &=\exp \left(-2 \pi r^2\right) \iint_{\mathcal{R}^2}\left(\exp \left(\left|b(o, r) \cap b\left(y_1-y_2, r\right)\right|\right)-1\right) \mathrm{d} y_1 \mathrm{~d} y_2 \\
        & \sim|\mathcal{R}| \mathrm{e}^{-2 \pi r^2} \int_{\mathbb{R}^2}(\exp (|b(o, r) \cap b(y, r)|)-1) \mathrm{d} y \quad \text { as }|\mathcal{R}| \rightarrow \infty
        \end{aligned}
\end{equation*}

因为渐进地,边界影响消失了。2个半径为$r$的圆盘在距离$h$的交区域,
\begin{equation*}
    A(h)=2 r^2 \arccos \left(\frac{h}{2 r}\right)-\frac{h}{2} \sqrt{4 r^2-h^2}, \quad h<2 r .
\end{equation*}

因此
\begin{equation*}
    \operatorname{var} V(\mathcal{R}) \sim|\mathcal{R}| \mathrm{e}^{-2 \pi r^2} 2 \pi \int_0^{2 r} h\left(\mathrm{e}^{A(h)}-1\right) \mathrm{d} h
\end{equation*}

因为$A(h)\geq \pi r^2-2rh$,积分是紧的下界,
\begin{equation*}
    \int_0^{2 r} h\left(e^{A(h)}-1\right) \mathrm{d} h>\int_0^{\pi r / 2} h\left(e^{\pi h^2-2 h r}-1\right) \mathrm{d} h
\end{equation*}

因此,
\begin{equation*}
    \begin{aligned}
        \operatorname{var} V(\mathcal{R}) & \gtrsim|\mathcal{R}| \mathrm{e}^{-2 \pi r^2} 2 \pi \int_0^{2 r} h\left(\mathrm{e}^{\pi h^2-2 h r}-1\right) \mathrm{d} h \\
        &=|\mathcal{R}| \pi \mathrm{e}^{-\pi r^2} \frac{2-\mathrm{e}^{-\pi r^2}\left(\pi^2 r^4+2 \pi r^2+2\right)}{4 r^2},
        \end{aligned}
\end{equation*}

其中$\gtrsim$表示渐进意义下的不等关系。
为了得到上界,我们注意到$A(h)\leq \pi r^2-\pi r h/2$(for $h\leq 2r$),这导致
\begin{equation*}
    \operatorname{var} V(\mathcal{R}) \lesssim|\mathcal{R}| 4 \mathrm{e}^{-\pi r^2} \frac{2-\mathrm{e}^{-\pi r^2}\left(\pi^2 r^4+2 \pi r^2+2\right)}{\pi r^2}
\end{equation*}

所以有一个常数因子$16/\pi^2$在两个渐进的界之间,我们有
\begin{equation*}
    \operatorname{var} V\left([0, \sqrt{n}]^2\right)=\Theta\left(n \frac{\mathrm{e}^{-\pi r^2}}{r^2}\right), \quad \text { as } n \rightarrow \infty
\end{equation*}

当$n\rightarrow\infty$的时候,必须地$r\rightarrow\infty$来期望得到覆盖,因此我们已经丢弃了
方差的界的所有小项。

结果是,(13.4)中$\mathbb{P}(V>0)$的下界需要为0,来保证收敛,即,
$(\mathbb{E} V)^2 / \mathbb{E}\left(V^2\right) \rightarrow 0$.
因为$(\mathbb{E} V)^2=n^2e^{-2\pi r^2}$,条件是
\begin{equation*}
    \frac{(\mathbb{E} V)^2}{\operatorname{var} V+(\mathbb{E} V)^2} \rightarrow 0 \quad \Longleftrightarrow \quad n r^2 \mathrm{e}^{-\pi r^2} \rightarrow 0 \quad \text { as } n \rightarrow \infty
\end{equation*}

由于因子$r^2$的存在,有一个比(13.3)更加限制性的必要的条件.
令$\pi r^2=\log n+\log\log n+c$,我们有$nr^2e^{-\pi r^2}\rightarrow e^{-c/\pi}$,
所以我们需要将常数$c$替换为一个满足$\omega(1)$的函数,即,对于平面被覆盖,我们的新的必须条件是
\begin{equation}
    \pi r^2=\log n+\log \log n+\omega(1)
\end{equation}

平面被覆盖的充要条件

为了展示这个更加紧的必要条件(1.35)也是充分的,我们使用Gilbert对open disk这个case的一个观察:
面积为$n$的方形区域被覆盖,如果圆盘边界的所有的交集被覆盖,
磁盘边界和面积$n$的正方形区域边界之间的所有交点也是如此.

如果圆盘是closed,结果是相同的,但是Gilbert条件的方程是很不优雅的,因为圆盘的所有交,现在
不得不被覆盖3次(他们已经被2个形成交的圆盘覆盖过了,但是还要被another disk覆盖),
而盘边界和边界之间的交点需要被覆盖两次(它们已经被引起交点的盘覆盖了一次,并且需要被另一个盘覆盖)。图13.2显示了购买力平价在单位正方形和半径为r的圆及其交点上的两种实现。


使用Gilbert的观察,我们得到结论如下:
\begin{itemize}
    \item 1. 对于一个半径为$r$的圆盘,交的密度是$4\pi r^2$,因为盘边界在距离2r内与每个盘边界相交
    \item 2. 在面积$n$的方形区域的交的期望次数是$4\pi r^2 n$
    \item 3. 为了覆盖,每一个交需要存在有一个PPP的点距离小于$r$
    \item 4. 交没有被覆盖的平均次数,是$4\pi r^2n e^{-\pi r^2}$,如果(13.5)成立,它将会变成0。所以这个条件也是充分的。
\end{itemize}

现在我们考虑多覆盖的case。

\subsection{多覆盖}
这里我们需要平面的每个位置都被覆盖至少$k$次。Gilbert的条件以一种直接的方式广义化到$k$-coverage:
每一个圆盘边界的交必须被覆盖$k$次。

正如单覆盖的case里讨论的那样,充要条件是:交被覆盖$k-1$(或者更少)次的个数的平均$I_{k-1}$,
变为0:
\begin{equation*}
    \mathbb{E}\left(I_{k-1}\right) \sim 4 \pi r^2 n \exp \left(-\pi r^2\right) \frac{\left(\pi r^2\right)^{k-1}}{\Gamma(k)} \rightarrow 0, \quad \text { as } n \rightarrow \infty
\end{equation*}

\begin{theorem}
    (固定圆盘们的k-converage的条件)  对于$\mathbb{P}\left(V\left([0, \sqrt{n}]^2\right)=0\right) \rightarrow 1 \text { as } n \rightarrow \infty$,即,k-coverage,它的充要条件是:
    \begin{equation*}
        \pi r^2=\log n+k \log \log n+\omega
    \end{equation*}
\end{theorem}

\begin{proof}
    对于$\pi r^2=\log n+k\log\log n+c,\mathbb{E}(I_{k-1})\rightarrow e^{-c}\pi^{-k}$

    正如连通性问题那样,k-coverage需要一个比Single coverage稍大的半径。

    通过更多的努力,可以得到,对于$\pi r^2=\log n+k\log \log n+c$
    \begin{equation*}
        \mathbb{P}\left(V\left([0, \sqrt{n}]^2\right)=0\right) \rightarrow \exp \left(-\frac{\mathrm{e}^{-c}}{\Gamma(k)}\right), \quad n \rightarrow \infty
    \end{equation*}
\end{proof}

这个结果的证明背后的想法是:观察到,
未覆盖区域基本上形成了它们自己的均值$e^{-c}/\Gamma(\kappa)$泊松过程。
为了使得这些启发性的结果(heuristics)更加严格,我们需要用到the Chen-Stein方法。

\section{应用}
覆盖过程在无线网络中的经典应用当然是传感器网络。
对于这个问题,之前节得到的结果是直接可应用的。
这里我们讨论远在布尔模型之上的3个应用。

\subsection{无线网络中的SINR 覆盖}
让$\Phi=\{ x_i \}\subset \mathbb{R}^2$是一个点过程,
用$I^{!x}(y)$表示位置$y$处的干扰,如果$x\in \Phi$是关心的发射机,即,
\begin{equation*}
    I^{\mid x}(y)=\sum_{z \in \Phi \backslash\{x\}} h_{z y} \ell(\|z-y\|)
\end{equation*}

其中$h_{zy}$是$z$和$y$之间的衰落。我们定义位置$y$处的干扰:
\begin{equation*}
    \operatorname{SINR}(x, y)=\frac{h_{x y} \ell(\|x-y\|)}{W+I^{1 x}(y)}
\end{equation*}

其中$W$是热力学噪声功率。发射机$x$的覆盖蜂窝包括所有这样的位置,
在其中$x$为关心发射机的SINR等于或
大于阈值$\theta$:
\begin{equation*}
    S_x=\left\{y \in \mathbb{R}^2: \operatorname{SINR}(x, y) \geq \theta\right\}
\end{equation*}

由于衰落,蜂窝非常可能是disconnected的。SINR覆盖过程是
\begin{equation*}
    \Xi_{\mathrm{SINR}}=\bigcup_{x \in \Phi} S_x
\end{equation*}

如果$\theta\geq 1$,每一个位置不能被超过一个发射机覆盖,当然并不能是最近的那个。

无噪声覆盖

如果$\Phi$是一个强度为$\lambda$的uniform PPP,衰落是iid.的瑞利衰落,路径损失指数是$\alpha$,
$W=0$,那么位置$y$被发射机$x$覆盖的概率$p_x(y)$,可以从
Section 5.2中成功概率推导出:
\begin{equation*}
    p_x(y)=\exp \left(-\lambda \pi \theta^\delta \Gamma(1+\delta) \Gamma(1-\delta)\|x-y\|^2\right)
\end{equation*}

其中$\delta=2/\alpha$.如果$\theta\geq 1$,$p_x(y)$是被覆盖的区域部分,我们得到了平均的覆盖蜂窝大小
\begin{equation*}
    \mathbb{E}|S|=\int_{\mathbb{R}^2} p_o(y) \mathrm{d} y=\frac{1}{\lambda \theta^\delta \Gamma(1+\delta) \Gamma(1-\delta)} .
\end{equation*}

没有干扰和噪声,当路径损失指数$\alpha$增加的时候,覆盖蜂窝收敛到Voronoi蜂窝图

无干扰的覆盖
没有interference,所以现在就是SNR 覆盖过程,
它是一个布尔模型(如果$\Phi$是一个PPP的话)。没有衰落,SNR cell仅仅是
一个半径为$(\theta W)^{-1/\alpha}$的圆盘,因为
\begin{equation*}
    \frac{\|x-y\|^{-\alpha}}{W} \geq \theta \quad \Longleftrightarrow \quad\|x-y\|<\frac{1}{(\theta W)^{1 / \alpha}} .
\end{equation*}

所以$|S|=\pi(\theta W)^{-\delta}$. 噪声功率和SNR门限应该被单个参数替代。
使用发射功率$P$和【将噪声功率从$W$改成$W^{\prime}=W/P$】等价。在这个case中,我们可以使用
Section 13.3的结果,来找到关于$P$的条件,来使一个大的区域被覆盖。由于瑞利衰落,我们有
\begin{equation*}
    \mathbb{P}(\mathrm{SNR} \geq \theta)=\mathbb{P}\left(\frac{h\|x-y\|^{-\alpha}}{W}\right)=\exp \left(-\theta W\|x-y\|^\alpha\right)
\end{equation*}

以及
\begin{equation*}
    \mathbb{E}|S|=\frac{\pi \Gamma(1+\delta)}{(\theta W)^\delta}
\end{equation*}

在非衰落情况下按小区大小归一化:
\begin{equation*}
    \frac{\mathbb{E}|S|}{\pi(\theta W)^{-\delta}}=\Gamma(1+\delta)
\end{equation*}

这表明了瑞利衰落会减少平均蜂窝大小,如果$\alpha>2$,因为:
$\Gamma(1+\delta)<1$如果$\delta<1$.

噪声和干扰都考虑,计算覆盖

瑞利衰落下,噪声和衰落都考虑,覆盖概率是
\begin{equation*}
    \left.p_o(y)=\exp \left(-\theta W\|y\|^\alpha-\lambda \pi \theta^\sigma \Gamma(1+\delta) \Gamma(1-\delta)\|y\|^2\right)\right) .
\end{equation*}

对于$\delta=1/2$这是可积的,但是得到的结果是很丑陋的unwieldy。

\subsection{最近邻发射机的SIR覆盖}
这里我们考虑一个之前模型的轻微改动版本,其中关心的发射机并没有确定性给出,而是选择为
最近的一个发射机,这里假设在距离$R$处。
假设接收机在原点$o$,因为现在圆盘$b(o,R)$被保证没有干扰器,这将导致干扰引起很大的改变。
所以,对于iid的随机变量$h_x$,干扰是:
\begin{equation*}
    I=\sum_{x \in \Phi \backslash b(o, R)} h_x \ell(x)
\end{equation*}

再一次我们关注这样的case,强度为$\lambda$的PPP的发射机在平面上,对于关心的信号和干扰,衰落都是瑞利衰落。
噪声被忽略。

为了找到干扰$R$的拉普拉斯变换,我们可以遵循Section 5.1.7的步骤,
只不过现在$b(o,R)$是空的。
\begin{equation*}
    \begin{aligned}
        \mathcal{L}_I(s \mid R) &=\mathbb{E} \prod_{x \in \Phi \backslash b(o, R)} \mathbb{E}_h\left(\mathrm{e}^{-s h \ell(x)}\right) \\
        &=\mathbb{E} \prod_{x \in \Phi \backslash b(o, R)} \frac{s \ell(x)}{1+s \ell(x)} \\
        & \stackrel{(\mathrm{a})}{=} \exp \left(-\lambda \int_{\mathbb{R}^2 \backslash b(o, R)} \frac{1}{1+s \ell(x)} \mathrm{d} x\right) \\
        &=\exp \left(-2 \pi \lambda \int_R^{\infty} \frac{r s}{s+r^\alpha} \mathrm{d} r\right) \\
        &=\exp \left(\pi s^\delta \Gamma(1+\delta) \Gamma(1-\delta)-\pi R^2 H_\delta\left(R^\alpha / s\right)\right),
        \end{aligned}
\end{equation*}

其中(a)是因为pgfl,$delta=2/\alpha$,$H_\delta$是高斯双曲函数
\begin{equation*}
    H_\delta(x) \triangleq{ }_2 F_1(1, \delta ; 1+\delta ;-x)
\end{equation*}

利用$H_\delta$,从$0$到$R$的 “complementary integral”可以表示成
\begin{equation*}
    \int_0^R \frac{r s}{s+r^\alpha} \mathrm{d} r=\frac{R^2}{2} H_\delta\left(R^\alpha / s\right) .
\end{equation*}

$o$被最近邻发射机(距离$R$处)覆盖的概率因此是
\begin{equation*}
    \begin{aligned}
        p_o(R) &=\mathbb{P}\left(h R^{-\alpha}>\theta I\right) \\
        &=\mathcal{L}_I\left(\theta R^\alpha\right) \\
        &=\exp \left(-\lambda \pi R^2\left(\theta^\delta \Gamma(1+\delta) \Gamma(1-\delta)-H_\delta(1 / \theta)\right)\right) .
        \end{aligned}
\end{equation*}

因为$R$是PPP中最近邻的距离,它的概率分布是$f_R(r)=2\lambda \pi r e^{-\lambda \pi r^2}$,
覆盖的概率是
\begin{equation*}
    \begin{aligned}
        p_o &=\mathbb{E}\left(p_o(R)\right) \\
        &=\frac{1}{1+\theta^\delta \Gamma(1+\delta) \Gamma(1-\delta)-H_\delta(1 / \theta)}
        \end{aligned}
\end{equation*}

对于$\alpha=4$,因为$H_{1/2}(1/\theta)=\sqrt{\theta}$,
我们得到覆盖概率$p_o$的简单的闭形式
\begin{equation}
    p_o=\frac{1}{1+\sqrt{\theta}(\pi / 2-\arctan (1 / \sqrt{\theta}))} .
\end{equation}



\subsection{隐私覆盖?隐秘覆盖?}
隐私图,(包括一些边,在那些边上安全通信是可行的),首次在11.4.3节提出。
这里我们研究对应的覆盖问题。

基站和窃听器在平面上随机分布,基站可以覆盖蜂窝区域,并且基站可以覆盖半径由到最近的窃听者的距离确定的圆形区域。
虽然这些假设导致了分析上的可处理性
模型,它们非常逼真。事实上,蜂窝网络正在经历
从精心规划的基站部署到不定期部署的重大转变
包括微型基站和毫微微小区的异构基础设施。即使
如果没有这样小的基站,可以认为从
随机模型的精度与基于晶格的模型相同或更好
基站模型,见图。7.1和7.2。

虽然这些假设导致了分析上的可处理性
模型,它们非常逼真。事实上,蜂窝网络正在经历
从精心规划的基站部署到不定期部署的重大转变
包括微型基站和毫微微小区的异构基础设施。即使
如果没有这样小的基站,可以认为从
随机模型的精度与基于晶格的模型相同或更好
基站模型,见图。7.1和7.2。

我们假定基站和窃听器来自独立的强度为$1$和$\lambda$的PPP,
对应地,在$\mathbb{R}^d$中,它们分别是$\Phi$和$\Psi $。
一个$\lambda=0.1$的例子在图13.3中展示。

该覆盖率模型的行为与相应的(具有相同分布的独立盘半径)的模型完全不同,
如果radii是
独立的,我们从式子(13.5)中知道
\begin{equation*}
    \mathbb{E}\left(\pi R^2\right)=(1+\in) \log n
\end{equation*}

是【渐进覆盖一个面积为$n$的方形】充分条件(对于任意$\in >0$)。
在隐私覆盖的case中,$\mathbb{E}(\pi R^2)=1/\lambda$,所以the radii是独立的,
\begin{equation*}
    \lambda=[(1+\in) \log n]^{-1}
\end{equation*}

是充分的,为了覆盖,$\lambda$随着$n$对数趋势减少。
由于依赖性(一个监听者可能确定一些近邻的覆盖圆盘的半径),
$\lambda$下降得如此之快,大概是一个$n^{-1/d}$的趋势。

一维中的隐私覆盖

我们确定线段的覆盖比例$\mathbb{P}(o \in \Xi)=1-\mathbb{E} \chi(o)$

\begin{theorem}
    
    (一维中的隐私覆盖)覆盖的线段的比例是
    \begin{equation*}
        \mathbb{P}(o \in \Xi)=\frac{1+4 \lambda}{(1+2 \lambda)^2}
    \end{equation*}
\end{theorem}

\begin{proof}
    让$L$是【$o$被一个$o$左侧的点覆盖】事件。让$R$是【$o$被一个$o$右侧的点覆盖】事件。
    事件$L$和$R$是独立的,
    \begin{equation*}
        \mathbb{P}(o \in \Xi)=1-(1-\mathbb{P}(L))(1-\mathbb{P}(R))=1-(1-\mathbb{P}(R))^2=2 \mathbb{P}(R)-\mathbb{P}(R)^2
    \end{equation*}

    具有对称性。事件$R$发生,如果,对于某个$t>0$,$\Phi$中最近的点
    (a legitimate node)在距离$t$,没有$\Psi $(监听者)中的点在$[0,2t]$。因此
    \begin{equation*}
        \mathbb{P}(R)=\int_0^{\infty} \mathrm{e}^{-t} \mathrm{e}^{-2 \lambda t} \mathrm{~d} t=\frac{1}{2 \lambda+1}
    \end{equation*}
\end{proof}

所以一个长度为$n$的区间的期望的vacancy,是
\begin{equation*}
    \mathbb{E} V([0, n])=n \mathbb{E} \chi(o)=n(1-\mathbb{P}(o \in \Xi))=\frac{4 \lambda^2 n}{(1+2 \lambda)^2}
\end{equation*}

为了达成对实线的覆盖($n\rightarrow \infty$),我们因此需要$\lambda^2 n\rightarrow 0$。
一个更加细节化的分析展示出,如果$n\rightarrow \infty$,$\mathbb{P}(V=0)\sim e^{-4 n \lambda^2}$。
所以,对于覆盖,尺度缩放条件事实上是$\lambda =\mathcal{o}(n^{-1/2})$。
如果$n\lambda^2\rightarrow \infty$,$\mathbb{P}(V=0) \rightarrow 0$;
如果$n\lambda^2\rightarrow 0$,$\mathbb{P}(V=0)\rightarrow 1$。

2维中的隐私覆盖

2维的case要更加复杂。自然的推论是:
关键缩放是$\lambda=\mathcal{o}(n^{-1/3})$.
事实上,可以发现,如果$n\lambda^3\rightarrow \infty$,
$\mathbb{P}(V=0)\rightarrow 0$。
传统上,最著名的结果是
\begin{equation*}
    n(\log n)^3 \lambda^3 \rightarrow \quad \Longrightarrow \quad \mathbb{P}(V=0) \rightarrow 1
\end{equation*}

和$\lambda=o\left(n^{-1 / 3}(\log n)^{-1}\right)$,
所以有一个对数gap.


\subsection{哨兵选择}
哨兵选择问题是从能量有效传感器网络中一个实用issue抽象出来的。
一个感兴趣的区域$\mathcal{R}$被认为是k-covered(每个$k\in \mathcal{R}$被至少$k$个传感器覆盖),
这$k$个传感器可以被$k$钟颜色以某种方式染色,使得
$\mathcal{R}$被指定颜色的传感器子集所覆盖吗?
如果是这种情况,那么可以将时间划分为帧,使得在帧i中,只有颜色1+mod(i,k)的节点处于活动状态,而其他节点可以休眠以节省能量,同时仍然保证始终覆盖。

因为k-coverage是可以以一些小代价达成的(传感器感知半径$r$的略微增长——看定理13.7),
可以预期显著的能量增益$\&$前提是提供k覆盖的传感器可以被划分为k个单个覆盖。

明显地,这一般是不可能的,很容易构造一个k-cover的反例,其中
k-cover并不是可分的。

\begin{example}
    让$n=4$,$k=2$.$\{1,2,3,4\}$的大小为2的子集构成的集合有6个元素,
    \begin{equation*}
        \mathcal{S}=\{\{1,2\},\{1,3\},\{1,4\},\{2,3\},\{2,4\},\{3,4\}\} .
    \end{equation*}

    那么让
    \begin{equation*}
        \begin{aligned}
            &S_1=\{\{1,2\},\{1,3\},\{1,4\}\}, \\
            &S_2=\{\{1,2\},\{2,3\},\{2,4\}\}, \\
            &S_3=\{\{1,3\},\{2,3\},\{3,4\}\}, \\
            &S_4=\{\{1,4\},\{2,4\},\{3,4\}\} .
            \end{aligned}
    \end{equation*}

    $S$的每一个元素在$S_1,S_2,S_3,S_4$只出现2次,
    但是不能将$S_i$分割成2个单独的覆盖。

\end{example}


具体的设置是:一个强度为1的传感器节点的PPP,一个感兴趣的区域$\mathcal{R}=[0,\sqrt{n}]^2$。
假设每个传感器可以覆盖一个半径为$r$的圆盘,让$\Xi$是得到的布尔模型。
让$C_k$标示$\mathcal{R}$被k-覆盖的事件。我们知道,
k-覆盖可以渐进得到,如果$\pi r^2=\log k\log \log n+\omega(1)$.
让$P_k$是$\Xi$是k-partitionable的事件。那么

\begin{equation*}
    \mathbb{P}\left(C_k \backslash P_k\right) \leq \frac{c_k}{\log n}
\end{equation*}

因此,如果$\mathbb{P}\left(C_k\right)=\Theta(1)$,那么
当$n\rightarrow \infty$的时候,条件概率$\mathbb{P}\left(P_k \mid C_k\right)=1-o(1)$。
严格的证明是很长的。
已经被知道:k-partitionability的“障碍”,正如图13.4中展示的那样,当$n\rightarrow \infty$的时候
以接近0的概率发生。所以渐进上,k-覆盖可以被分拆almost surely.提出的哨兵选择方案是可以work的。

一个更加简单的方式是:验证一个随机染色是否可以以正概率达到parition.
这是"概率方法"的一个应用。这个方法,和Lovasz局部定理一样,产生
想要的结果如果每处覆盖的等级是$3k \log \log n$的。

这个证明方法并不是构造的,即,如果找到这样的染色仍然是开放问题。事实上,
传感器节点应该通过使用单独局部知识来选择一个颜色。对于覆盖过程更具体的介绍可以
在Hall(1985)的著作中找到。在第4章Stoyan et al.(1995)的部分,以及第3章到第5章的Meester\&Roy(1996)的部分,我们介绍了他们是怎么研究布尔模型的。
概率方法是Alon\&Spencer(2008)的研究对象。

SINR覆盖在第二部分的Baccelli \& Blaszczyszyn (2009)被讨论。
最近邻发射机覆盖的case在Andrews et al. (2011)得到了详细分析。
拓展到multi-tier异构网络的case在Dhillon et al. (2012)被研究。

在Sarkar \& Haenggi
(2012)中,隐私覆盖的问题被引入和研究
关于哨兵选择的主要的结果由Balister et al.
(2010).给出。

\section*{练习题}

\begin{exercise}
    让$\Phi_1,\Phi_2,\cdots$是$\mathbb{R}$上的独立的强度为$\lambda$的均匀PPP,
    让$1\geq t_1\geq t_2,\cdots$是收敛到0的正数序列。
    \begin{equation*}
        \Xi=\bigcup_{i \in \mathbb{N}} \bigcup_{x \in \Phi_i}\left(x, x+t_i\right),
    \end{equation*}
    其中$(x,x+t_i)\subset \mathbb{R}$是从$x$到$x+t_i$的开区间,
    让$V=V(\mathbb{R})=|\mathbb{R}\backslash \Xi \mid $是这个模型的vacancy.
    未覆盖的集合$\mathbb{R}\backslash\Xi$被称作随机康托集。
    找到位置$y\in \mathbb{R}$没有被$\Xi$覆盖的概率,给出要使$\mathbb{P}(V=0)$成立,关于序列$(t_i)$的充分条件和必要条件。
\end{exercise}

\begin{exercise}
    对于具有半径$r$的固定圆盘的布尔模型的单个覆盖,找到关于$r(n)$需要满足的条件,可以使得,
    vacancy$V=V([0,\sqrt{n}]^2)$,平均趋于0,但是$V=0$的概率并不是1。即,
    \begin{equation*}
        \mathbb{E}(V) \rightarrow 0  \text{but}  \mathbb{P}(V=0) \nrightarrow 1
    \end{equation*}
    有一个$r(n)$可以使得$mathbb{E}(V)\rightarrow 0$并且$\mathbb{P}(V=0)=0$吗?
\end{exercise}

\begin{exercise}
    Stienen模型是一个二维的覆盖过程,其中germs来自均匀PPP,grains是半径是【最近邻距离一般】的圆盘。
    给出Stienen模型的一个在$[0,5]^2$的实现,其中PPP的强度是1。确定模型的覆盖面积比例。
\end{exercise}

\begin{exercise}
    在标准布尔模型(germs来自PPP)中,巨大成分是唯一的如果它存在。通过反例表明,在顶点形成固定晶格的模型中,这并不一定正确。
    
\end{exercise}

\begin{exercise}
    对于在13.4.2节讨论的最近邻发射机覆盖问题,找到$\alpha=4$时的覆盖概率。
    正如(13.6)那样,但是考虑噪声。
\end{exercise}

% 13.6
\begin{exercise}
    一维中的隐私覆盖。考虑两个窃听者之间的典型到达间隔。根据窃听者PPP的强度,找出该间隔被覆盖的概率。
    提示:首先说明间隔的覆盖范围仅取决于间隔中点附近两点的位置。
\end{exercise}




todo
todo
todo
todo
todo
todo
todo

