\chapter{Bond percolation和site percolation}

\section{随机树和分支过程}
在涉及到图和它们的性质的问题中,经常是有用的去首先考虑无环的图,即,树。

\subsection{正则分支树的渗流}






todo todo todo todo
todo todo todo todo
todo todo todo todo
todo todo todo todo

\section*{练习题}
\begin{exercise}
    考虑2-branching树(也被称为二叉树),具有二项分布的后代分布,确定Percolation的概率,并且画出来,
    找到(10.4),(10.5),(10.6)的临界指数$\beta,\gamma,\delta$
\end{exercise}

\begin{exercise}
    展示二叉树的每一代中$(Z_n)$的成员数量的序列,从马尔可夫链条的角度去考虑,给出转移概率。
\end{exercise}


\begin{exercise}
    对于【后代概率分布是均值为$\lambda$的Possion分布】的分支过程,找到它灭绝的概率$\eta$
\end{exercise}

\begin{exercise}
    让$T$是分支过程的全体后代(total progency),即,
    \begin{equation*}
        T \triangleq \sum_{n=0}^{\infty} Z_n
    \end{equation*}

    让它的生成函数是$G_T(s)=\mathbb{E}(s^T)$.说明
    \begin{equation*}
        G_T(s)=s G\left(G_T(s)\right)
    \end{equation*}

    其中,$G_{(s)}=\mathbb{E}(s^X)$是后代分布的生成函数。
    
    提示。首先建立,如果$Z_1=i$,
    \begin{equation*}
        T \stackrel{\mathrm{d}}{=} 1+\sum_{k=1}^i T_k
    \end{equation*}
    
    其中,$T_k$是iid.的随机变量,和$T$有相同的分布。
\end{exercise}

\begin{exercise}
    对于一个平均后代为$\mu<1$的分支过程,说明,期望的总体后代是
    \begin{equation*}
        \mathbb{E} T=\frac{1}{1-\mu} .
    \end{equation*}
\end{exercise}


\begin{exercise}
    使用随机游走解释递归公式(10.3)
\end{exercise}


\begin{exercise}
    确定(渐进情形下),图10.17中lozenge lattice的渗流阈值
\end{exercise}


\begin{exercise}
    中枢边。考虑$\mathbb{L}^2$上一个bond percolation模型,事件$A=\{ o\leftrightarrow \partial B(1) \}$.
    让四个边是$e_1,\cdots,e_4$。考虑一个显式的表达式,对于事件$W_i$使得:边$e_i,i\in[4]$是中枢的,
    确定$\mathbb{P}_p(W_i)$。接着找到
    $\mathbb{E}_p(N(A))$和$\mathbb{E}_p(N(A) \mid A)$。
    当$p\rightarrow 0$的时候,这两个期望表现如何?这个行为是我们所预料到的吗?
    解释这个现象。
\end{exercise}


\begin{exercise}
    Russo公式。
\end{exercise}


\begin{exercise}
    证明,对于$\mathbb{L}^2$上的bond percolation,$p_c=p_c^{\prime}$成立。
    其中,
    \begin{equation*}
        p_{\mathrm{c}}^{\prime}=\sup \{p: \chi(p)<\infty\}
    \end{equation*}

    $\chi(p)$是原点处component的大小的期望值(正如10.1.4节定义的那样)
\end{exercise}

% 10.11
\begin{exercise}
    当证明infinite components数量不可能是无限的时候,我们使用了trifurcations(三叉点)的概念。
    对具有three trifurcations的$n$,在diamond $S(n)$中画出一个configuration.
    指明这样的边界点,满足它们被假定认为和一个infinite open cluster连通。
    解释为什么bifurcations(二叉)不能得到这个结论。
\end{exercise}





