\documentclass[lang=cn,10pt]{elegantbook}

\title{基于ElegantBook\LaTeX{} 模板的电子笔记}
\subtitle{\text{Stochastic Geometry for Wireless Networks}}

\author{authorPlaceholder}
\institute{authorPlaceholder \LaTeX{} Program}
\date{\today}
\version{3.141}
\bioinfo{邮箱}{\email{placeholder@example.com}}

\extrainfo{ $\alpha \beta \gamma \delta \epsilon \zeta \eta \theta \iota \kappa \lambda \mu \nu  o  \pi \rho \sigma \tau \upsilon \phi \chi \psi \omega $}

\setcounter{tocdepth}{3}

\logo{logo-blue.png}
% \cover{scatter.jpg}  好像路径不需要写全就能找到。。。
\cover{cover.jpg} 


% 本文档命令
\usepackage{array}
\newcommand{\ccr}[1]{\makecell{{\color{#1}\rule{1cm}{1cm}}}}

% 修改标题页的橙色带
% \definecolor{customcolor}{RGB}{32,178,170}
% \colorlet{coverlinecolor}{customcolor}

\begin{document}

\maketitle
\frontmatter

\tableofcontents  %toc

\mainmatter

%%%%%%%%%%%%%%%%%%%%%%%%%%%
\chapter*{Introduction}
\markboth{Introduction}{Introduction}
this is introduction.

我为什么想写这本书:

\begin{itemize}
  \item 1.我想进一步学习\LaTeX{}的语法。
  \item 2.这本书的部分内容对于我来说实在是太难了,我转而投向汉化而不是做我自己乱七八糟的笔记。
  \item 3.我一直有写电子\LaTeX{}笔记的想法和翻译电子书的想法。
  \item 4.我尝试使用\lstinline{\input}命令来把一个长的tex文件分开
  \item 5.我尝试使用\lstinline{\lstinline}命令来处理代码风格化
  \item 6.我希望这个模板最好是很容易跨平台的
  \item 7.更新到最新版的模板后,不再缺少字体
\end{itemize}

\chapter{引言}

\section{什么是随机几何?}
\section{无线网络空间模型的一种——点过程}







\section*{练习题}
\begin{exercise}
    下面哪一个说法正确?
\end{exercise}


\begin{exercise}
    让$X_1,X_2,\cdots,X_k$是iid. 的随机变量,他们的CDF都是$F(x)$。
    最小值$\operatorname{min}_{i}\{ X_i \}$的cdf?
\end{exercise}


\begin{exercise}
    让$X$是一个参数为$Y$的随机变量,其中$Y$本身是一个参数为$\mu$的泊松随机变量。
    说明$X+Y$的生成函数是
    \begin{equation*}
        G_{X+Y}(x) \triangleq \mathbb{E}\left(x^{X+Y}\right)=\exp \left\{\mu\left(x e^{x-1}-1\right)\right\}
    \end{equation*}
\end{exercise}

\begin{exercise}
    让$X_1,\cdots,X_N$是$N$个iid.的随机变量,他们的cdf都是$F(x)$,其中
    $N$是均值为$\mu$的泊松随机变量。计算这$N$个随机变量的最大值的cdf$G(x)$
\end{exercise}


\begin{exercise}
    让$Z$是在半径为$a$的圆盘随机独立选取2个点的距离。
    说明$\mathbb{Z^2}=a^2$
\end{exercise}


\begin{exercise}
    让$a,b,c$是iid.的指数分布变量。说明,多项式$ax^2+bx+c$有实数根的概率是$1/3$
\end{exercise}

\begin{exercise}
    一只鹅下了$N$个蛋,其中$N$是均值为$\lambda$的泊松随机变量。每一个蛋以概率$p$被孵化,
    都是彼此独立的。让$K$是幼鹅的数量。
    计算$\mathbb{E}(K|N)$ ,$\mathbb{E}(K)$,$\mathbb{E}(N|K)$
\end{exercise}



\chapter{点过程的描述}

\section{一维点过程的描述}






\section*{练习题}
\begin{exercise}
    对于$\mathbb{R}$上的强度为$\lambda$的均匀PPP,推导interarrival
    interval包括原点$o$的pdf。你观察了什么?
\end{exercise}


\begin{exercise}
    让$X_m$是一个均值为$m$的泊松随机变量。如果$M$是一个服从Gamma分布
    的随机变量,说明$X_M$的分布是negative binomial.
\end{exercise}

\begin{exercise}
    让$\Phi$是$\mathbb{R}$上一个强度为$\lambda$的均匀PPP,确定
    $\mathbb{P}( \Phi(B_1)=n_1,\Phi(B_2)=n_2)\text{ for }B_1=[0,2]\text{ and }B_2=[1,3]$
\end{exercise}

\begin{exercise}
    让$(X_i),i\in \mathbb{N}$,是一族独立的均值为$1/\lambda$的指数分布的随机变量.
    确定
    \begin{equation*}
        \mathbb{P}\left(X_1>t\right)+\sum_{k=1}^{\infty} \mathbb{P}\left(X_1+\cdots+X_k \leq s ; X_1+\cdots+X_{k+1}>t\right)
    \end{equation*}
    给出结果的解释,作为一个特定的点过程的void probability.
\end{exercise}

\begin{exercise}
    结果空间$\mathcal{N}$可以定义为如下公式吗?
    \begin{equation*}
        \mathcal{N} \triangleq\left\{\varphi \subset \mathbb{R}^d:|\varphi|=0\right\} ?
    \end{equation*}
\end{exercise}

\begin{exercise}
    径向仿真。考虑正实数线上,一个强度为$1$的一维PPP$\Psi=\left\{y_i\right\} \subset \mathbb{R}^{+}$。
    假定点是有序的,即,$y_1<y_2<\cdots$。考虑$\Psi$,
    我们想要产生$\mathbb{R}^2$上强度为$\lambda$的HPPP $\Phi=\left\{\left(r_i, \varnothing_i\right)\right\}$的一个实现,
    其中$(r_i,\phi_i)$是点的极坐标表示。振幅集合$\{ r_i\}$也应该是有序的。
    从$y_i$到$r_i$找到函数,可以将一维度的点映射到二维度的振幅,with the desired intensity.
\end{exercise}


\begin{exercise}
    生成函数。让$X_1,X_2,\cdots$是一系列iid的随机变量$\Omega \mathbb{N}_0$
    的生成函数$G_X$.让$N$是一个随机变量,和$X_i$独立,
    also taking values in $\mathbb{N}_0$. 展示
    \begin{equation*}
        S=X_1+X_2+\cdots+X_N
    \end{equation*}
    $S$具有生成函数$F_S(s)=G_N(G_X(s))$以及导数$G_S(s)$如果$N$是均值为$\lambda$的泊松随机变量。
\end{exercise}

\begin{exercise}
    类比(2.3),找到一个表示,来计数二维ranomly translated lattice的测度。
\end{exercise}

\begin{exercise}
    确定$\Lambda(\{ x\})$对于所有的$x\in \mathbb{R}^d$对于
    例子2.1定义的点过程。
\end{exercise}

\begin{exercise}
    写下die point process的显式分布
\end{exercise}

\begin{exercise}
    例子2.3中的点过程,是两个(甚至更多)的基础点过程的“叠加”superpositionn.
    找到他们,并且验证$P=P_1*P_2$,使用卷积的标准准则。
\end{exercise}

\begin{exercise}
    考虑Bernoulli die process,定义如下:从初始的die process开始,接着独立地以概率$q$移走每一个点.
    这样一个点过程也可以称为a die process with erasures,with erasure probability $q$.
    定义$\Lambda(\{ o \})$和$\Lambda(\mathbb{R}^2)$
\end{exercise}


\begin{exercise}
    让$s\Phi_{\text{die}}$是the die process $\Phi_{\text{die}}$ scaled by $s$.
    让
    \begin{equation*}
        \Phi^{\prime}=\bigcup\left\{x \in \mathbb{Z}^2: x+\frac{1}{3} \Phi_{\text {die }}\right\}
    \end{equation*}
    
    其中$x+\frac{1}{3}\Phi_{\text{die}}$是 a translation of the (scaled) die process by $x$,
    考虑静态点过程$\Phi \triangleq \Phi_Y^{\prime}$,其中
    $Y$是$[0,1]^2$上的均匀分布。找到$\Phi$的强度。
\end{exercise}


\begin{exercise}
    让$\Phi\subset \mathbb{R} $是一个强度函数为$\lambda(x)=\mathbf{1}(x \in[0,1])$的均匀PPP。
    如果我们独立地对每一个点做displacing by a random amount uniform on $[-1/2,1/2]$,
    得到的点过程的强度是多少?
\end{exercise}


\begin{exercise}
    让$\Phi=\{x_1,\cdots \}\subset \mathbb{R}$是一个强度为
    $\lambda(x)=\mathbf{1}(0\leq x<2\pi)$的均匀PPP。
    让$\Phi^\prime=\{\sin x_i \}$.  确定$\Phi^\prime$的强度$\lambda^\prime$
\end{exercise}


\begin{exercise}
    让$\Phi$是一个高斯泊松过程,$\sigma=1$,$n=1$,让$B^{-}$是$\mathbb{R}^2$
    的左半平面。对于$\Phi$以及对于$\Phi_{(1,0)}$确定
    $\Lambda(B^{-})$.
    提示,关联问题的第二部分和点对点问题的误比特率问题
\end{exercise}

\begin{exercise}
    图2.2展示了确定的non-HPPP的实现。确定这个过程的强度测度$\Lambda([0,10]^2)$
\end{exercise}

\begin{exercise}
    随机traslated lattice (看定义2.15)是遍历的吗?
\end{exercise}


\begin{exercise}
    让$\Phi$是$\mathbb{R}^2$上一个强度1的HPPP。计算满足
    $NN(NN(x))=x$的点的比例。这些是属于互相最近邻点对的节点。
    这个比例依赖强度吗?同时计算,对于lattice,这个比例是多少?
\end{exercise}


\begin{exercise}
    让$\Phi$指代randomly translated square integer lattice

    (a)展示强度$\lambda \equiv 1$
    (b)让$B_{\diamond}$是面积为$A,1<A<2$的方形区域,旋转了$\pi/4$,使得
    相对的角点,他们的x坐标(或者y坐标)相等。找到$P(\Phi(B_{\diamond})=0)$,
    以及$P(\Phi(B_{\diamond})>2)$,都写成$A$的函数的形式。
\end{exercise}


\begin{exercise}
    我们想要产生一个isotropic PPP $\Phi\subset \mathbb{R}^2$的实现,此PPP的强度是
    \begin{equation*}
        \lambda(x)=\frac{M}{2 \pi \sigma^2} \exp \left(-\frac{\|x\|^2}{2 \sigma^2}\right), \quad x \in \mathbb{R}^2 .
    \end{equation*}

    Simulation的区域是$W=[-5,5]^2$。
    (a)
    (b)
    (c)

\end{exercise}

\begin{exercise}
    有向空空间函数。让$\Phi\subset \mathbb{R}^2$是一个强度为1的PPP。
    让$\{ (r_i,\phi_i) \}$是点集的极坐标表示。让$n\in \mathbb{N}$,定义
    \begin{equation*}
        R_k=\min \left\{\left(r_i, \phi_i\right) \in \Phi, k \frac{2 \pi}{n}<\phi_i \leq(k+1) \frac{2 \pi}{n}: r_i\right\}, \quad k=0,1, \ldots, n-1
    \end{equation*}

    $R_k$是最近邻节点和原点的距离,在kth sector of width $2\pi/n$

    我们感兴趣$R_k$中最小值的CDF。不计算,你估计这个CDF有哪些性质?
    
    为了验证你算出来的CDF,使用Problem1.2中得到的结果。
\end{exercise}

\begin{exercise}
    描述具有图2.10所示Fry Plot的PPP。
\end{exercise}
\chapter{点过程模型}

\section{引言}
\section{广义有限点过程}




\section*{练习题}
\begin{exercise}
    对于例子3.1的binomial点过程,确定Janossy测度和Janossy密度。
\end{exercise}


\begin{exercise}
    考虑有限点过程$\Phi\subset \mathbb{R}^2$,概率$p_1=1/2,p_2=1/2$
    ,以及
    \begin{equation*}
        f_1(x)=1_{W^+}(x)
    \end{equation*}
    \begin{equation*}
        f_2\left(x_1, x_2\right)=\left(\frac{1}{3} 1_{W-}\left(x_1\right)+\frac{2}{3} 1_{W+}\left(x_1\right)\right)\left(\frac{1}{3} 1_{W-}\left(x_2\right)+\frac{2}{3} 1_{W+\left(x_2\right)}\right)
    \end{equation*}

    对于$W^+=[0,1]^2$,$W^-=[-1,0]\times[0,1]$。
    找到Janossy密度$j_1,j_2$,计算$\mathbb{E}u(\Phi)$对于
    $u_1(\Phi)=\Phi(B)$(对于一般的$B\subset \mathbb{R}^2$),
    $u_2(\Phi)=2^{\Phi(\mathbb{R}^2)}$   ,使用公式(3.3)
\end{exercise}


\begin{exercise}
    找到一个广义的有限点过程的强度测度,写成$(p_k)$和$F_n$函数的形式,其中$(p_k)$和$F_n$的定义在Definition 3.1可以找到
\end{exercise}


\begin{exercise}
    考虑一个一维的Cox过程,它是这么得到的:通过使用例子3.6中的random thinning,
    其中基础的PPP是均匀的,强度为$\lambda_b$,随机场$T$由式子$T(x)=e^{-L|x|}$给出,
    $L$是一个随机变量。
    
    (a)让$\mathbb{P}(L=1)=3/4$以及$\mathbb{P}(L=2)=1/4$。确定$\Lambda([0,y))$,对于所有的
    $y\in \mathbb{R}$,$\lambda(\mathbb{R})$,以及强度函数在$0,\lambda(0)$处的值。

    (b)让$L$的pdf是$f_L(x)=2 x \mathbf{1}(0 \leq x \leq 1 )$。再一次,确定$\Lambda([0,y))$,对于所有的
    $y\in \mathbb{R}$,$\lambda(\mathbb{R})$,以及强度函数在$0,\lambda(0)$处的值。

\end{exercise}

\begin{exercise}
    对于例子3.7中的germ-grain模型,考虑这样的Cox过程,它的强度场如下:
    \begin{equation*}
        \zeta(x)=\lambda_a \mathbf{1}\left(x \in \Xi_r\right)+\lambda_b \mathbf{1}\left(x \notin \Xi_r\right) .v
    \end{equation*}

    确定过程的强度,展示$Matern$聚簇过程可以使用这个强度场建模。
\end{exercise}


\begin{exercise}
    考虑Gauss-Poisson过程,$p_2=1$(所有的簇正好有2个点),parent intensity是$\lambda$.
    我们想要计算零空间函数$F(r)$。如果我们认为聚簇过程是【一系列簇中心$\Phi_c$的过程】和【一系列其他点$\Phi_1$的过程】的superposition,
    都是强度为$\lambda$的均匀PPP。他们是依赖的,因为每一个$\Phi_c$中的点来说,都有$\Phi_1$中一个点和它距离1,随机朝向。
    说明零空间函数可以写成
    \begin{equation*}
        \begin{aligned}
            1-F(r) & =\mathbb{P}\left(\Phi_c(b(o, r))=0, \Phi_1(b(o, r))=0\right) \\
            & =\mathbb{P}\left(\Phi_c(b(o, r))=0\right) \mathbb{P}\left(\Phi_1(b(o, r))=0 \mid \Phi_c(b(o, r))=0\right)
            \end{aligned}
    \end{equation*}

    接着确定$F(r)$对于$r\leq 1/2$
    
    对于$r>1/2$,说明
    \begin{equation*}
        1-F(r)=\mathrm{e}^{-\lambda \pi r^2} \exp \left(-\lambda \int_r^{r+1} \frac{2 \arccos \left[\left(1+y^2-r^2\right) /(2 y)\right]}{2 \pi} 2 \pi y \mathrm{~d} y\right) .
    \end{equation*}

\end{exercise}

\begin{exercise}
    在R中仿真泊松hole process,验证强度。
\end{exercise}

\begin{exercise}
    对于Gibbs过程,通过对$P_{\Phi}(Y_k)$和$P_\lambda(Y_K)$的显式的计算(见3.9式),
    说明两个过程的void probability一致。
\end{exercise}


\begin{exercise}
    证明式子3.22。提示:见例子3.3
\end{exercise}

\begin{exercise}
    找到和Strauss hard-core过程(点的数量固定为$n$)对应的
    pair potential $\theta$,对于给定的$R$和$R^{\prime}$
\end{exercise}

\begin{exercise}
    在图3.8中,我们观察到所有的簇都有2个或者3个点。解释这个现象。
\end{exercise}

\begin{exercise}
    对于一般的finite Gibbs process,密度$f(\varphi)$可以表示为
    \begin{equation*}
        f\left(\varphi_n\right)=\mathrm{P}\left(Y_n\right) f_n\left(x_1, \ldots, x_n\right), n \in \mathbb{N}_0
    \end{equation*}
    吗?其中$ Y_n=\{\varphi \in \mathcal{N}: \varphi(W)=n\}$,
    $f_n$是已知过程有$n$个点,算条件密度。
\end{exercise}

\begin{exercise}
    描述和图3.11的FryPlot有关的点过程。
\end{exercise}
\chapter{点过程上的和与积}

\section{引言}
\subsection{动机}
\subsection{记号}





\section*{练习题}
\begin{exercise}
    使用显式的计算矩生成函数的导数的方法,完成引理4.8的证明。
\end{exercise}

\begin{exercise}
    让$v(x)=\min \left\{1, \frac{1}{2}\|x\|^2\right\}$,
    从例子2.1中确定die process的the pgfl $G[v]$
\end{exercise}


\begin{exercise}
    比较$\mathbb{E}e^{-s[f]}$和$\exp (-\mathbb{E}S[f])$
\end{exercise}

\begin{exercise}
    展示泊松hole过程的pgfl。
    例子3.7中介绍的那个possion hole process,
    参数是$\lambda_1,\lambda_2,r$
    \begin{equation*}
        G[v]=\mathrm{E}\left[\exp \left(-\lambda_2 \int_{\mathrm{R}^2 \backslash \Xi_r}(1-v(x)) \mathrm{d} x\right)\right] .
    \end{equation*}

    如果$v=1-\mathbf{1}_B(x), \lambda_1=1, \text { and } r=1$,
    确定$G[v]$。

    这产生了void probability.
\end{exercise}



\begin{exercise}
    隐私图的定义如下。
\end{exercise}


\begin{exercise}
    让$\Phi$是一个泊松高斯过程,其中representative cluster是
    $\Phi_0=\{ o \}$ with probability $1-p$,$\Phi_0=\{ o,x \}$ with probability $p$.
    其中$x$以密度$f$分布。
    假定我们选择的密度正好可以让$\mathbb{P}(x=o)=0$。展示pgfl是
    \begin{equation*}
        G[v]=\exp \left(\lambda_{\mathrm{p}} \int_{\mathbf{R}^d}\left((1-p) v(x)+p v(x) \int_{\mathbf{R}^d} v(x+y) f(y) \mathrm{d} y-1\right) \mathrm{d} x\right) .
    \end{equation*}

    所以$v \in v$  意味着 $1-v(x)$在有界集外消失。
\end{exercise}

\chapter{无线网络中的干扰和中断}
在这一章,我们讨论无线网络中点过程的和以及其他函数的一些应用。
其中点过程理论最著名的一个应用就是对干扰进行特征化。
这里的干扰,我们通常指干扰功率。

\section{对干扰进行特征化}

\subsection{将干扰看作一个散粒噪声}
让$\ell: \mathbb{R}^d \mapsto \mathbb{R}^{+}$是一个路径损耗函数。
如果我们用一个点过程$\Phi$对发送机的位置进行建模,所有的发送机都以单位功率发送,
在位置$y$处测量到的总和功率是
\begin{equation}
    I(y)=\sum_{x \in \Phi} \ell(y-x), \quad y \in \mathbb{R}^d
\end{equation}

假设信道中没有衰落。通过和式子(3.24)进行比较,干扰$sI(y)$
是一个散粒噪声随机场。更精确地说,因为路径损失规律是典型的形式
$\ell(x)=\|x\|-\alpha$(或者它的一个有界的版本),它是一个功率律的散粒
噪声随机场。一个例子在Fig. 5.1中展现,其中从一个PPP引起的干扰用平面上的一条线表示。

\subsection{静态点过程的平均干扰}
首先,我们对功率律$\ell(x)=\|x\|-\alpha$情形下,一个强度为$\lambda$的
平稳的发射机点过程$\Phi \subset \mathbb{R}^d$的干扰的平均值,进行推导。
因为期望是对所有的点过程进行考虑的,遵循平稳性$\mathbb{E}I(y)=\mathbb{E}I(o)$
(事实上,对所有的的$y\in \mathbb{R}^d$都有 $I(y)\stackrel{\mathrm{d}}{=} I(o)$ ) ,
我们可以聚焦于原点,那里
\begin{equation}
    \mathbb{E} I=\mathbb{E}\left(\sum_{x \in \Phi}\|x\|^{-\alpha}\right)
\end{equation}

利用Cambell的求和公式(定理4.1)
\begin{equation}
    \mathrm{E} I=\lambda \int_{\mathbf{R}^d}\|x\|^{-\alpha} \mathrm{d} x .
\end{equation}

在2维的case,
\begin{equation}
    \lambda \int_{\mathbb{R}^2}\|x\|^{-\alpha} \mathrm{d} x=\lambda \int_0^{2 \pi} \int_0^{\infty} r^{-\alpha} r \mathrm{~d} r \mathrm{~d} \varphi= \begin{cases}\left.2 \pi \frac{\lambda}{2-\alpha} r^{2-\alpha}\right|_0 ^{\infty}, & \alpha \neq 2 . \\ \left.2 \pi \lambda \log r\right|_0 ^{\infty}, & \alpha=2 .\end{cases}
\end{equation}

当$\alpha\neq2$的时候,事实上积分并不收敛,
当$\alpha=2$的时候,积分也并不收敛。
所以对于满足功率路径损耗律的情形,2维空间中的平均干扰是无穷的,
对于所有的路径损失指数$\alpha$都是如此

在3维中,我们发现
\begin{equation}
    \mathbb{E} I=\lambda \int_{\mathbb{R}^3}\|x\|^{-\alpha} \mathrm{d} x=\left.4 \pi \frac{\lambda}{3-\alpha} r^{3-\alpha}\right|_0 ^{\infty}, \quad \alpha \neq 3 .
\end{equation}

再一次,没有$\alpha$取值可以让积分收敛。我们可能推断这个结论对任意的维度都成立。
为了深入研究,我们关注均匀PPP,应用映射定理(Theorem 2.34)将它投射到一维。

1.使用函数$f(x)=\| x \|$映射到一维。新的点过程$\Phi^{\prime}$,具有intensity measure为 
$\Lambda^{\prime}([0, r))=\lambda c_d r^d$和intensity function为$\lambda^{prime}(r)=\lambda_d d r^{d-1}$,
其中$c_d$是$d$维单位球的体积。
2.应用Campbell定理得到【和的平均】
\begin{equation}
    \begin{aligned}
        \mathbb{E} I=\mathbb{E}\left(\sum_{r \in \Phi^{\prime}} r^{-\alpha}\right) &=\int_0^{\infty} r^{-\alpha} \lambda^{\prime}(r) \mathrm{d} r \\
        &=\left.\lambda c_d \frac{d}{d-\alpha} r^{d-\alpha}\right|_0 ^{\infty}, \quad \alpha \neq d .
        \end{aligned}
\end{equation}
所以,对于所有的$d$,没有$d$可以让这个平均存在。如果$\alpha\leq d$,
原函数在$\alpha=\infty$处为无穷,那意味着远处的所有的干扰器对干扰贡献了绝大部分。
换言之,平均干扰变得无穷,因为节点的数量是无穷的。
如果$\alpha\geq d$,原函数在$\alpha=0$处为无穷,表明和原点非常近的那些节点对干扰贡献了大部分,
因为路径损失律的奇异性。

尽管平均干扰在$\alpha \leq d$和$\alpha \geq d$的时候都发散,这两种分散在定性上是非常不同的。
如果$\alpha >d$,条件(4.5)满足,所以干扰是有限的,a.s. almost surely.
另一方面,如果$\alpha\leq d$,条件(4.5)并不满足,表明干扰是无限的,a.s.
所以在第一个case里,我们可以期望干扰具有一个定义良好的分布(尽管长尾导致均值发散),
而在第二种情况下,其分布不存在,因为我们不仅有
$\mathbb{E}I=\infty$,也有$I=\infty$ a.s.

这个均值的发散性,明显是因为附近的干扰器明显是一个modeling artifact,
因为没有一个接收机获得比发射的功率更多的功率。
如果路径损失函数被一个更加精确的函数替代,比如
\begin{equation}
    \ell(x)=\min \left\{1,\|x\|^{-\alpha}\right\} \quad \text { or } \quad \ell(x)=(1+\|x\|)^{-\alpha} \text {, }
\end{equation}

这样在$r\rightarrow 0$的modeling artifact就被改善了,条件(4.5)总是满足,
只要$\alpha >d$均值干扰就是有限的。\\


\begin{remark}
    \begin{itemize}
        \item 为什么我们不能使用$\|x \|^{-\alpha}$直接作为映射函数?原因是
        ,如果这样,得到的点过程将不会是局部有限的:
        我们将有$\Lambda^{\prime}([0,1])=\Lambda([1,\infty])=\infty$,
        这违反了均值测度的有限性。
        \item 对于相同intensity的所有点过程,均值是完全相等的。以及,由于平稳性,
        干扰也是相同的,无论我们在$\mathbb{R}^d$的哪里测量。我们将会在后续的章节讨论:
        如何去确定过程中一个点或邻近一个点的干扰,而不是任意的位置。
      \end{itemize}
\end{remark}


在无线网络中的语境中,均匀PPP经常是一个泊松网络。

\subsection{泊松网络中干扰的方差}
在泊松网络的case中,我们使用Corollarys 4.8来找到干扰的方差。
让$\ell(x)=\min \left\{r_0^{-\alpha},\|x\|^{-\alpha}\right\} \text { for } r_0>0$。
对于一个$\mathbb{R}^d$上的homogeneous PPP,我们有
\begin{equation}
    \begin{aligned}
        \operatorname{var} I=\lambda \int_{\mathbb{R}^d} \ell^2(x) \mathrm{d} x &=\lambda c_d r_0^{d-2 \alpha}+\left.\frac{\lambda c_d d}{d-2 \alpha} r^{d-2 \alpha}\right|_{r_0} ^{\infty} \\
        &=\lambda c_d r_0^{d-2 \alpha}\left(\frac{2 \alpha}{2 \alpha-d}\right), \quad \text { for } 2 \alpha>d
        \end{aligned}
\end{equation}

条件$2\alpha>d$经常是满足的。如果网络是有限的,我们让$r_0\rightarrow 0$,
对于有限方差,我们需要$\alpha<d/2$,$\alpha$的范围是非常不可能的.

\subsection{泊松网络中附近发射机的干扰}
在一些case中,我们可能对从仅仅单个附近发射机发出的干扰感兴趣。
我们首先关注距离原点最近的发射机,标记它的干扰为$I_1$。事实上,
这个信号功率可能事实上表示我们关心的信号,因为它来自一个邻近的节点;
在这个case中,$I_1$可能是关心的功率。

使用例子2.11中计算的距原点最近点的距离,我们发现
\begin{equation}
    \mathbb{P}\left(I_1 \leq x\right)=\mathbb{P}\left(R^{-\alpha} \leq x\right)=\mathbb{P}\left(R>x^{-1 / \alpha}\right)=\exp \left(-\lambda c_d x^{-\delta}\right)
\end{equation}

其中$\delta \triangleq d / \alpha$。均值是
\begin{equation}
    \mathbb{E} I_1=c^{1 / \delta} \Gamma(1-1 / \delta)
\end{equation}

如果$\delta<1$,那么这根本不存在。所以,如果是奇异路径损失律的情形,
来自唯一最近节点的平均干扰是无限的。这是由于对于小$\delta$分布的长尾:
\begin{equation}
    \mathbb{P}\left(I_1>x\right) \sim \lambda c_d x^{-\delta}, x \rightarrow \infty .
\end{equation}

如果$\delta<1$,$\int \mathbb{P}\left(I_1>x\right) \mathrm{d} x $发散,
因此,平均值并不存在。广义上,$\mathbb{E}(I_1^p)$对于$p<\delta$存在。

类比地,我们有pdf概率分布函数
\begin{equation}
    f_{I_1}(x) \sim \lambda c_d \delta x^{-\delta-1}, x \rightarrow \infty
\end{equation}

接下来我们把这个表达式推广到nth近邻干扰器的干扰$I_n$的情形。
nth近邻$R_n$的距离的the survivor function(和cdf互补)可以从式子(2.12)得到
\begin{equation}
    \mathbb{P}\left(R_n>r\right)=\frac{\Gamma_{\mathrm{ic}}\left(n, \lambda c_d r^d\right)}{\Gamma(n)}
\end{equation}

所以,对于$n=2$,
\begin{equation}
    \mathbb{P}\left(I_2<x\right)=\exp \left(-\lambda c_d x^{-\delta}\right)\left(1+\lambda c_d x^{-\delta}\right)
\end{equation}

以及
\begin{equation}
    \mathbb{P}\left(I_2>x\right) \sim \frac{1}{2}\left(\lambda c_d\right)^2 x^{-2 \delta} .
\end{equation}

所以我们需要$2\delta>1$来让$\mathbb{E}I_2$存在。对于一般的$n$,
\begin{equation}
    \mathbb{P}\left(I_n<x\right)=\exp \left(-\lambda c_d x^{-\delta}\right) \sum_{i=0}^{n-1} \frac{\left(\lambda c_d x^{-\delta}\right)^i}{i !}
\end{equation}

对于长尾概率,我们需要从$n$加到$\infty$,所以主导项将会是当$x\rightarrow \infty$ 有$i=n$的那一个。
因此
\begin{equation}
    \mathbb{P}\left(I_n>x\right) \sim \frac{1}{n !}\left(\lambda c_d\right)^n x^{-n \delta}
\end{equation}

这意味着$\mathbb{E}(I_n^p)$存在,如果$p<n\delta$。因此,如果
干扰消除技术被使用来达到一个有限的2nd moment,$k>\alpha$干扰器需要被
消除,在二维的网络中。
尽管我们可以找到对所有的$n$,$I_n$的分布,通过这种方式很难得到
总共干扰的分布,因为$I_n$既不是独立也不是同分布的。
我们换一种思路进行处理,分为考虑衰落和不考虑衰落两种情形来进行讨论。


\subsection{没有衰落的泊松网络中,干扰的分布}
在本小节中,我们关注二维网络的情况,并假设没有衰落。因为$\ell(x)$假设是各向同性的,我们也使用它的一维
版本$\tilde{\ell}: \mathbb{R}^{+} \mapsto \mathbb{R}^{+}$,
所以$\tilde{\ell}(\|x\|) \equiv \ell(x) .$
这里我们假定$\tilde{\ell(x)}$是严格单调减的(invertible),以及$\lim _{r \rightarrow \infty} \tilde{\ell}(r)=0$

我们的目标是找到干扰的特征函数,进一步,如果可能,找到它的
分布。我们遵循一种基本但强大的技术,使用Sousa\&Silvester(1990)用到的技术,
它包括以下2步:
\begin{itemize}
    \item 1. 首先考虑一个有限的网络,假定半径为$a$的圆盘,中心在原点,
    对此有限区域中具有固定数量节点做condition。节点的位置是i.i.d.的事件。
    \item 2. 接着de-condition作用在(Poisson)数量的节点,让圆盘半径趋向无穷。
\end{itemize}

\text{Step 1. }考虑和原点相距$a$的节点发出的干扰:
\begin{equation}
    I_a=\sum_{x \in \Phi \cap b(o, a)} \tilde{\ell}(\|x\|) .
\end{equation}

在$a\rightarrow \infty$的极限下,$I_a\rightarrow I$.让$\mathcal{F}_{I_a}$是随机变量$I_a$的
特征函数,即
\begin{equation}
    \mathcal{F}_{I_a}(\omega) \triangleq \mathbb{E}\left(\mathrm{e}^{\mathrm{j} \omega I_a}\right), \mathrm{j}=\sqrt{-1} .
\end{equation}

对半径$a$的具有$k$个节点的圆盘做Condition,
\begin{equation}
    \mathcal{F}_{I_a}(\omega)=\mathbb{E}\left(\mathbb { E } \left(\mathrm{e}^{\left.\mathrm{j} \omega I_a \mid \Phi(b(o, a))=k\right)}\right.\right.
\end{equation}

考虑到有$k$个点在$b(o,a)$,这些圆盘上的点是i.i.d.的,具有径向密度
\begin{equation}
    f_R(r)= \begin{cases}2 r / a^2 & \text { if } 0 \leq r \leq a \\ 0 & \text { otherwise }\end{cases}
\end{equation}

特征函数是$k$个独立特征函数的乘积:
\begin{equation}
    \mathrm{E}\left(\mathrm{e}^{\mathrm{j} \omega I_a} \mid \Phi(b(o, a))=k\right)=\left(\int_0^a \frac{2 r}{a^2} \exp (\mathrm{j} \omega \tilde{\ell}(r)) \mathrm{d} r\right)^k
\end{equation}

\text{Step 2. }在$b(o,a)$找到$k$个节点的概率由泊松分布给出,因此
\begin{equation}
    \mathcal{F}_{I_a}(\omega)=\sum_{k=0}^{\infty} \frac{\exp \left(-\lambda \pi a^2\right)\left(\lambda \pi a^2\right)^k}{k !} \mathrm{E}\left(\mathrm{e}^{\mathrm{j} \omega I_a} \mid \Phi(b(o, a))=k\right)
\end{equation}

将(5.2)式带入上式,对$k$求和,将和理解为指数函数的泰勒展开,我们得到
\begin{equation}
    \mathcal{F}_{I_a}(\omega)=\exp \left(\lambda \pi a^2\left(-1+\int_0^a \frac{2 r}{a^2} \exp (\mathrm{j} \omega \tilde{\ell}(r)) \mathrm{d} r\right)\right) .
\end{equation}

做替换$r\rightarrow \ell^{-1}(x)$,其中${\tilde{\ell}}^{-1}$是$\tilde{\ell}$的逆,
让$a\rightarrow \infty$,产生
\begin{equation}
    \lim _{a \rightarrow \infty} a^2\left(-1+\int_0^a \frac{2 r}{a^2} \exp (\mathrm{j} \omega \tilde{\ell}(r)) \mathrm{d} r\right)=\int_0^{\infty}\left(\tilde{\ell}^{-1}(x)\right)^2 \mathrm{j} \omega \mathrm{e}^{\mathrm{j} \omega x} \mathrm{~d} x
\end{equation}


所以
\begin{equation}
    \mathcal{F}_I(\omega)=\exp \left(\mathrm{j} \lambda \pi \omega \int_0^{\infty}(\tilde{\ell}-1(x))^2 \mathrm{e}^{\mathrm{j} \omega x} \mathrm{~d} x\right) .
\end{equation}

为了得到更多正确的结果,我们需要指定路径损失律。对于标准功率律
$\tilde{\ell}(r)=r^{-\alpha}$,我们得到
\begin{equation}
    \mathcal{F}_I(\omega)=\exp \left(j \lambda \pi \omega \int_0^{\infty} x^{-2 / \alpha} \mathrm{e}^{\mathrm{j} \omega x} \mathrm{~d} x\right)
\end{equation}

对于$\alpha\leq 2$,积分发散,表明干扰是无穷的a.s. 对于$\alpha >2$,
\begin{equation}
    \mathcal{F}_I(\omega)=\exp \left(-\lambda \pi \Gamma(1-2 / \alpha) \omega^{2 / \alpha} \mathrm{e}^{-\mathrm{j} \pi / \alpha}\right), \quad \omega \geq 0 .
\end{equation}

对于负的$\omega$,上式由对称条件$\mathcal{F}_I^*(-\omega)=\mathcal{F}_I(\omega)$确定。
对于$\alpha=4$,
\begin{equation}
    \mathcal{F}_l(\omega)=\exp \left(-\lambda \pi^{3 / 2} \exp (-j \pi / 4) \sqrt{\omega}\right) .
\end{equation}

这个case引起了我们的兴趣,因为它是唯一的case使得密度以闭形式的形式存在:
\begin{equation}
    f_I(x)=\frac{\pi \lambda}{2 x^{3 / 2}} \exp \left(-\frac{\pi^3 \lambda^2}{4 x}\right) .
\end{equation}

这就是所谓的 Lévy distribution,也可以看成一个逆的gamma distribution,或者看成无限均值高斯分布的逆。
对于其他的$\alpha$值,密度可能得用无穷级数的形式表示(Sousa\&Silvester 1990,Eqn.(22)).

\subsection{稳定分布}
为了解释式子(5.5)的特征函数,我们简要地介绍stable distributions类族.

\begin{definition}
    (稳定分布) 一个随机变量$X$被称为具有稳定分布,
    如果对于所有的$a,b>0$,存在$c>0,d$,满足
    \begin{equation}
        a X_1+b X_2 \stackrel{\mathrm{d}}{=} c X+d,
    \end{equation}
    其中$X_1$和$X_2$是和$X$具有相同分布的i.i.d.的r.v. 如果式子(5.7)对于
    $d=0$成立,那么分布是严格稳定的。
\end{definition}

\begin{theorem}
    对于任意的稳定的随机变量$X$,有一个在$0<\delta\leq 2$范围的参数$\delta$,
    让定义中的$c$满足$c^\delta=a^\delta+b^\delta$.
\end{theorem}

参数$\delta$是【指数特征】,也称为【稳定性指数】。对于高斯随机变量,
$\delta=2$,因为$aX_1+bX_2$是均值$(a+b)\mu$且方差$(a^2+b^2)\sigma^2$的高斯随机变量,即,
分布对于$c=\sqrt{a^2+b^2}$和$d=(a+b-c)\mu$成立。一般地,一个稳定分布的特征函数是:

\begin{equation}
    \mathbb{E}\left(\mathrm{e}^{\mathrm{j} t X}\right)= \begin{cases}\exp \left(\mathrm{j} t \mu-\gamma|t|^\delta(1-\mathrm{j} \beta \operatorname{sgn}(t) \tan (\pi \delta / 2))\right) & \delta \neq 1 \\ \exp (\mathrm{j} t \mu-\gamma|t|(1+\mathrm{j}(2 \beta / \pi) \operatorname{sgn}(t) \log (|t|))) & \delta=1,\end{cases}
\end{equation}

其中$\operatorname{sgn}(t)$是$t$的符号,$\beta\in [-1,1]$是偏度参数,
$\mu$是漂移,$\gamma$是分散参数。比较式子(5.8)和式子(5.5),我们观察到式子(5.5)是
一个随机变量的特征函数,其中$\delta=2/\alpha,\beta=1,\mu=0$,
\begin{equation*}
    \gamma=\frac{\lambda \pi \Gamma(1-2 / \alpha)}{\cos (\pi / \alpha)}=\frac{\lambda \pi \Gamma(1-\delta)}{\cos (\pi \delta / 2)}
\end{equation*}

尤其是,对于$\alpha=4$,$\gamma=\lambda\pi^{3/2}/\sqrt{2}$. 正如之前提到的,这个case中,
其中$\delta=1/2$,是唯一的case满足干扰功率的pdf存在。

偏度$\beta=1$意味着随机变量的支撑集限定为$\mathbb{R}^{+}$。

一个偏度$\beta=1$且漂移$\mu=0$的稳定随机变量$X$的拉普拉斯变换,具有如下的紧凑型:
\begin{equation}
    E\left(e^{-s X}\right)=e^{-\kappa s^\delta}
\end{equation}

参数$\kappa $和$k$相关,关系是$k=\kappa\cos(\pi/\alpha)$,我们
将$\kappa$称为dispersion参数或者scale参数。
这并不违反公共术语,因为不幸的是,对于稳定分布,存在非常多的不同的,关于dispersion和scale的定义。
在表格(5.9)中,高斯随机变量的dispersion等于方差的一半。对于其他的稳定分布
($\delta<2$),方差并不存在,但是dispersion表演一个相似的角色,因为它表明
此分布延展得多宽。

稳定分布的理论包括结果,可以将PPP和稳定分布联系起来。

\begin{theorem}
    (级数表示) 让$\{\tau_i\subset\mathbb{R} \}$表示强度$\lambda$的PPP的到达时间,
    让$(h_i)$是和$\tau_i$独立的一族随机变量,如果有限和
    \begin{equation}
        \sum_{i=1}^{\infty} \tau_i^{-1 / \delta} h_i
    \end{equation}

    收敛a.s. ,那么这个级数和会收敛到一个稳定随机变量。
\end{theorem}

式子(5.10)被称为LePage级数表示/LePage series representation  

这和我们已经发现的结果一致,因为,在我们的case中,距离增长到$d$th power,
$r_i^d=\| x_i \|^d$,包括一个强度为$\lambda_{c_d}$的homogeneous PPP.
所以我们的和变成
\begin{equation*}
    \sum_{i=1}^{\infty}\left(r_i^{-d}\right)^{-\alpha / d}
\end{equation*}

以及$\delta=d/\alpha$.

定理5.3的一个结果如下所示

\begin{corollary}
    (干扰的尺度变换) 让$I(\lambda)$是强度为$\lambda$的单位PPP $\Phi\subset \mathbb{R}^d$,
    路径损失函数为$\ell(x)=\| x \|^{-\alpha}$.那么
    \begin{equation*}
        a I(\lambda) \stackrel{\mathrm{d}}{=} I\left(a^\delta \lambda\right), \quad \forall a>0 \text {, where } \delta=d / \alpha
    \end{equation*}
\end{corollary}

这看起来可能有些反直觉,但是它和(5.7)其实是一致的。因为$I(\lambda_1+\lambda_2)\stackrel{\mathrm{d}}{=} I(\lambda_1)+I(\lambda_2)$,
(利用PPP的superposition性质),我们可以从推论中得到
\begin{equation*}
    a I_1(\lambda)+b I_2(\lambda) \stackrel{\mathrm{d}}{=} I_1\left(a^\delta \lambda\right)+I_2\left(b^\delta \lambda\right) \stackrel{\mathrm{d}}{=} I\left(\left(a^\delta+b^\delta\right) \lambda\right) \stackrel{\mathrm{d}}{=} c I(\lambda)
\end{equation*}

其中$c=(a^\delta+b^\delta)^{1/\delta}$,$I,I_1,I_2$来源于3个有各自强度的独立的PPP。

从上面的推论中我们可以直接获得干扰的拉普拉斯变换,如果功率放缩$P$倍。
因为$I \propto P$,我们有$PI(\lambda)=I(P^\delta \lambda)$,即,功率放缩$P$倍
和密度放缩$P^\delta$倍有着相同的效果。所以,可能反直觉的,如果稳定分布,干扰和密度$\lambda$并不成比例。

$\delta<1$的稳定分布变量具有性质:iid的$X_n$的算术平均$(X_1+\cdots+X_n)/n$
随着$n$增长,和$X_k n^{1/\delta-1}$相等。
n个这样稳定的随机变量的总和和最大值以相同的方式缩放,
i、 例如,总和由偶尔的大值控制。

Figure 5.2 展示了对powe-law路径损失的PPP的干扰建模的稳定随机变量的实现。
对应的surrvicvor funciton在Figure 5.3中给出。

对应的拉普拉斯变换是
\begin{equation}
    \mathcal{L}_I(s)=\exp \left(-\lambda \pi \Gamma(1-2 / \alpha) s^{2 / \alpha}\right)
\end{equation}

特征指数小于1的稳定分布没有任何有限矩。特别是,平均干扰发散,这是由于原点处路径损耗定律的奇异性。
这也可以从$\mathbb{E}(I)=-\left.(\mathrm{d} / \mathrm{d} s) \log \left(\mathcal{L}_I(s)\right)\right|_{s=0}=\lim _{s \rightarrow 0} c s^{2 / \alpha-1}=\infty$
很容易得出。

使用节点位置的iid性质,对固定数量的节点进行conditioning处理,并对泊松分布进行条件处理的方法适用于许多其他问题。


\subsection{有衰落情形下的干扰分布}
有衰落的情形下,从每个发射机$x$发出的功率,会被乘上一个衰落系数$h_x$,假设i.i.d. .
因此干扰由下面的和式给出:
\begin{equation*}
    I=\sum_{x \in \Phi} h_x \ell(x),
\end{equation*}

我们的目标是计算拉普拉斯变换
\begin{equation*}
    \mathcal{L}(s)=\mathbb{E} \mathrm{e}^{-s I}=\mathrm{E}\left(\prod_{x \in \Phi} \mathrm{e}^{-s h \ell(x)}\right)
\end{equation*}

因为衰落是iid的,所以
\begin{equation*}
    \mathcal{L}(s)=\mathbb{E}_{\Phi}\left(\prod_{x \in \Phi} \mathbb{E}_h\left(\mathrm{e}^{-s h \ell(x)}\right)\right)
\end{equation*}

将此PPP映射到一维,我们知道$\lambda(r)=\lambda c_d r^{d-1}$。再一次让
$\tilde{\ell}(\| x \|)\equiv \ell(x)$。现在我们对$v(r)=\mathbb{E}_h\left(\mathrm{e}^{-\operatorname{sh} \ell(r)}\right)$
使用pgfl,得到
\begin{equation*}
    \mathcal{L}(s)=\exp \left\{-\int_0^{\infty} \mathrm{E}_h\left[1-\mathrm{e}^{-s h \bar{\ell}(r)}\right] \lambda(r) \mathrm{d} r\right\}
\end{equation*}

对于1维的PPP,对$h$做condition,我们有
\begin{equation*}
    \begin{aligned}
        \int_0^{\infty}(1-\exp (-s h \tilde{\ell}(r))) \lambda(r) \mathrm{d} r &=\lambda c_d \int_0^{\infty}\left(1-\exp \left(-s h r^{-\alpha}\right)\right) d r^{d-1} \mathrm{~d} r \\
        & \stackrel{(a)}{=} \lambda c_d \int_0^{\infty}(1-\exp (-s h / y)) \delta y^{\delta-1} \mathrm{~d} y \\
        & \stackrel{\text { bb }}{=} \lambda c_d \int_0^{\infty}(1-\exp (-s h x)) \delta x^{-\delta-1} \mathrm{~d} x \\
        & \stackrel{(c)}{=} \lambda c_d \int_0^{\infty} x^{-\delta} s h \exp (-s h x) \mathrm{d} x \\
        &=\lambda c_d(h s)^\delta \Gamma(1-\delta), \quad 0<\delta<1
        \end{aligned}
\end{equation*}

其中(a)是因为做了替换$y\leftarrow r^{-1/\alpha}$,
(b)是因为$x\leftarrow y^{-1}$,(c)是因为积分改写。
聚焦于对于$h$随机性求期望,我们得到
\begin{equation*}
    \mathcal{L}(s)=\exp \left(-\lambda c_d \mathbb{E}\left(h^\delta\right) \Gamma(1-\delta) s^\delta\right)
\end{equation*}

所以,有衰落的情形下,干扰也必须是一个指数特征为$\delta$的稳定分布。
分散系数$\kappa$从$\mathbb{E}(h^\delta)$到$\lambda c_d \mathbb{E}(h^\delta)\Gamma(1-\delta)$改变。

在瑞利衰落的case中,其中$h$是指数的,$\mathbb{E}(h^\delta)=\Gamma(1+\delta)$,
所以
\begin{equation*}
    \begin{aligned}
        \mathcal{L}(s) &=\exp \left(-\lambda c_d \Gamma(1+\delta) \Gamma(1-\delta) s^\delta\right) \\
        &=\exp \left(-\lambda c_d \frac{\pi \delta}{\sin (\pi \delta)} s^\delta\right) \\
        &=\exp \left(-\frac{\lambda c_d}{\operatorname{sinc} \delta} s^\delta\right)
        \end{aligned}
\end{equation*}

当$\delta \rightarrow 1$,$\sin(\pi\delta)\sim \pi (1-\delta)$,所以极限情形下我们有
\begin{equation*}
    \mathcal{L}(s) \approx \exp \left(-\lambda c_d s^\delta \frac{\delta}{1-\delta}\right)
\end{equation*}

这表明:当路径损失指数接近网络维度时,分散锐利地增长。


\section{泊松网络中中断的概率}
\begin{definition}
    (定义SINR) SINR的定义是
    \begin{equation}
        \mathrm{SINR} \triangleq \frac{S}{W+I}
    \end{equation}
    其中$S$是我们关心的信号的功率,$W$是噪声的功率,$I$是干扰的功率。
    如果噪声被忽略($W=0$),SINR退化为SIR。
\end{definition}

对于固定调制,固定编码制式,干扰被看作噪声对待,
例如,使用一个简单线性接收机,一个公认的打包模型是:
如果SINR超过某个阈值$\theta$,则传输成功。所以我们
如下定义成功概率。

\begin{definition}
    (定义传输成功概率) 传输成功概率$p_s(\theta)$的定义是:
    \begin{equation}
        p_{\mathrm{s}}(\theta) \triangleq \mathbb{P}(\text { SINR }>\theta)
    \end{equation}
    
\end{definition}
而$1-p_s$就是中断概率。

阈值$\theta$和 (物理层)(比特?)传输率$R$相关。
利用香农信道容量公式,他们由$\theta=2^R-1$相关联;
由于实际限制,阈值应该选取得略大,来给$R$提供容限。



\section{泊松双极网络中的空间吞吐}
\begin{definition}
    (泊松双极网络)\quad 一个泊松双极网络包括由系列发射机$\{ x_i \}\subset \mathbb{R}^2$组成的PPP,
    ,还包括系列接收机$\{ y_i \}$,
    每个接收机均匀随机选择朝向进行接收。
    每一对接收机发射机满足$\| x_i-y_i \|=r\text{ for all }i$.

    使用displacement theorem(Theorem 2.33),接收机$\{ y_i \}$的点过程本身也是PPP。
    所以泊松双极网络包括两个【有依赖关系的】PPP,即一个发射机PPP和一个接收机PPP。


\end{definition}


\subsection{空间吞吐}
成功概率$p_s$可以堪称对特定链路的一个吞吐度量。
为了量化整个网络的性能,吞吐量需要在所有链路上平均。
假设泊松双极网络的每一个发射机决定以概率$p$发射,以概率$1-p$静默。
这是所谓的ALOHA信道接入制式。在一个强度为$\lambda$的泊松双极网络,
空间吞吐可能定义为
\begin{equation}
    T(p) \triangleq p \lambda p p_s(p, r)
\end{equation}

其中$p_s(p,r)$是发射机在距离$r$上,以概率$p$选择发射,通信的成功概率。
对于一个给定的$r$,它在
\begin{equation}
    p_{\mathrm{opt}}=\frac{1}{\lambda \theta^\delta \pi \Gamma(1-\delta) \Gamma(1+\delta) r^2}
\end{equation}

处取得最大值。当然,$p_{\mathrm{opt}}$不能超过1.
如果右侧的表达式产生一个大于1的值,那就表明网络密度$\lambda$可以增加
来达到一个更高的吞吐。类似地,链路距离$r$可以包括在度量中,
这将导致运输能力并且优化。

\subsection{香农吞吐}
空间吞吐量是基于中断的度量,因为某些传输不会
成功。相反,如果发射机能够对SINR条件做出快速反应
并调整其传输速率,另一个量可能是相关的,称为
香农吞吐(Baccelli \& Blaszczyszyn (2009)提出)。
考虑$p_s(\theta)$作为门限$\theta$的函数,香农吞吐(以nats的单位)是:
\begin{equation}
    \begin{aligned}
        \mathbb{E} \log (1+\mathrm{SINR}) &=-\int_0^{\infty} \log (1+\theta) \mathrm{d} p_{\mathrm{s}}(\theta) \\
        & \stackrel{\text { (a) }}{=} \int_0^{\infty} p_{\mathrm{s}}\left(\mathrm{e}^x-1\right) \mathrm{d} x
        \end{aligned}
\end{equation}

这里(a)是因为一个随机变量$X$的期望可以用它的survivor funciton来表示。
乘以当前发射机的密度,得到另一种类型的
空间吞吐量——我们把干扰当做噪声可以期待的最好的吞吐。






\section{传输容量}
Weber等人(2005)中引入的传输容量是无线网络的性能指标,用于测量成功传输的空间强度,受允许中断概率的限制(当接收器处的SINR低于阈值时发生中断)。限制停机概率的优点是
这在许多应用中是必要的。相反,如(5.15)所示,最大化空间吞吐量可能会导致如下成功概率
低为e–1,这意味着每个数据包会多次重传,进而导致不可接受的延迟。此外,与其他指标相比,传输容量相对可控;在某些情况下,它会导致简单的闭式表达式,而在许多其他情况下,可以找到紧边界。


\begin{definition}
    (传输容量)\quad 对于一个强度为$\lambda$的泊松双极网络,在原点添加参考接收机,在距离$r$处添加所需发射机,
    让$p_s(\lambda)$指代:
    在存在来自双极网络中所有发射机的干扰的情况下,该接收机的成功概率。
    考虑到一个目标中断概率$\epsilon\in (0,1)$,传输容量$\lambda_{tc}$定义为
    \begin{equation}
        \lambda_{\mathrm{tc}}(\in) \triangleq(1-\in) \max \left\{\lambda: p_s(\lambda) \geq 1-\in\right\}
    \end{equation}
    
    因为成功概率$p_s(\lambda)$对于$\lambda$是单调减的,传输容量可以写成
    \begin{equation}
        \lambda_{\mathrm{tc}}(\in)=(1-\in) p_s^{-1}(1-\in)
    \end{equation}

    其中$p_s^{-1}(1-\epsilon)$产生【导致成功概率$1-\epsilon$的】强度$\lambda$。
\end{definition}

\begin{example}
    如果所有发射机都使用单位功率,所有链路都受到瑞利衰落影响,并且噪声被忽略,那么求出二维网络的传输容量。
\end{example}

\begin{solution}
    从式子(5.14),我们知道
    \begin{equation}
        p_s(\lambda)=\exp \left(-\pi \lambda \theta^\delta r^2 \Gamma(1+\delta) \Gamma(1-\delta)\right)
    \end{equation}

    其中$\delta=2/\alpha$.定义$\gamma \triangleq \pi \theta^\delta r^2 \Gamma(1+\delta) \Gamma(1-\delta)$,
    通过inverting $p_s(\lambda)$,我们得到
    \begin{equation}
        \lambda_{\mathrm{tc}}(\epsilon)=-\frac{1}{\gamma}(1-\epsilon) \log (1-\epsilon)
    \end{equation}

    因为$\epsilon$经常是非常小的,我们可能使用对数函数的Taylor展开,来得到
    \begin{equation}
        \begin{aligned}
            \lambda_{\mathrm{tc}}(\epsilon) &=\frac{\epsilon-\epsilon^2 / 2}{\gamma}+O\left(\epsilon^3\right) \\
            &=\frac{\epsilon}{\gamma}+O\left(\epsilon^2\right),
            \end{aligned}
    \end{equation}

    这给出了$\epsilon\rightarrow 0$时的上界。
\end{solution}

上面solution中定义的参数$\gamma$被Haenggi(2009)称为空间竞争参数spatial contention parameter.
它通常被定义为作为发射器强度$\lambda$函数的成功概率$p_s(\lambda)$在$\lambda=0$位置的(负)斜率
\begin{equation}
    \gamma \triangleq-\left.\frac{\mathrm{d} p_{\mathrm{s}}(\lambda)}{\mathrm{d} \lambda}\right|_{\lambda=0}
\end{equation}

这个参数描述了:随着越来越多的链路加载在网络中,成功概率下降得有多快。
所以,一般来说,传输容量的渐进行为,是
\begin{equation}
    \lambda_{t c}(\epsilon) \sim \frac{\epsilon}{\gamma}, \quad \epsilon \rightarrow 0
\end{equation}

对于大$\epsilon$,传输容量并不是随着$\epsilon$单调增的,这是由于因子$1-\epsilon$的存在;
它在$\epsilon=1-1-/e$的时候达到最大值$1/(e\gamma)$.

作为链路距离$r$的一个函数,很明显看到$\lambda_{\mathrm{tc}} \propto r^{-2}$,
这表明一个事实:
如果遵循中断限制,在$r$距离上传输,需要和$r^2$成比例的面积。




\subsection{具有衰落的网络}
对于独立瑞利衰落的case,我们发现Example 5.2的闭形式的传输容量。
对于一般的衰落,我们通过再一次聚焦在一个主导的干扰器上,取得了一个界。
在这个case中,
\begin{equation}
    \Phi_{\text {dom }}=\left\{x \in \Phi: h r^{-\alpha} / h_x\|x\|^{-\alpha}<\theta\right\}
\end{equation}

其中,$h$是关心的链路的衰落系数,$h_x$是干扰器$x$的衰落系数。
正如2.9.2节推导的那样,有衰落的路径损失过程$\Xi $的强度函数是:
\begin{equation}
    \mu(x)=\lambda \pi \delta x^{\delta-1} \mathbb{E}\left(h^\delta\right) .
\end{equation}


因此,给定$h$,缺少任意主导干扰器的概率是【$\Xi$中没有点和原点距离在$r^\alpha \theta /h$】的概率。
即
\begin{equation}
    \mathbb{P}\left(\Phi_{\text {dom }}=\emptyset \mid h\right)=\exp \left(-\int_0^{r^\alpha \theta / h} \mu(x) \mathrm{d} x\right)=\exp \left(-\lambda \pi \mathbb{E}\left(h^\delta\right) r^2 \theta^\delta h^{-\delta}\right)
\end{equation}

因此,
\begin{equation}
    p_s(\lambda)<\mathbb{E}_h\left(\exp \left(-\lambda \pi \mathbb{E}\left(h^\delta\right) r^2 \theta^\delta h^{-\delta}\right)\right)
\end{equation}

应用如下形式的Jensen不等式:
\begin{equation}
    \mathbb{E}_h\left(\exp \left(-\lambda \pi \mathbb{E}\left(h^\delta\right) r^2 \theta^\delta h^{-\delta}\right)\right)>\exp \left(-\lambda \pi \mathbb{E}\left(h^\delta\right) r^2 \theta^\delta \mathbb{E}\left(h^{-\delta}\right)\right)
\end{equation}

并不产生想要的上界,因为不等式的方向错了。但是这表明,近似的,有
\begin{equation}
    p_s(\lambda) \approx \exp \left(-\lambda \pi r^2 \theta^\delta \mathbb{E}\left(h^\delta\right) \mathbb{E}\left(h^{-\delta}\right)\right)
\end{equation}

结果是,
\begin{equation}
    \lambda_{\mathrm{tc}}(\epsilon) \approx \frac{-(1-\epsilon) \log (1-\epsilon)}{\pi r^2 \theta^\delta \mathbb{E}\left(h^\delta\right) \mathbb{E}\left(h^{-\delta}\right)} \sim \frac{\epsilon}{\pi r^2 \theta^\delta \mathbb{E}\left(h^\delta\right) \mathbb{E}\left(h^{-\delta}\right)}
\end{equation}

强度,这是精确的表达式,对于瑞利衰落来说,即,在这种情况下,使用主导干扰的成功概率上限和使用Jensen不等式的成功概率下限的效果精确地相互抵消。

因为$E(h^\delta)\mathbb{E}(h^{-\delta})>1$(除了没有衰落的case),
衰落似乎对传输容量具有负面影响。


\section{干扰的时间相关性}
到目前为止,我们对干扰做特征化,仅仅是在平面上的单个位置,
没有考虑时间这一要素。虽然干扰的分布不取决于平稳点过程中的位置,
但在两个附近位置测量的干扰是相关的,
因为它们取决于同一点过程,因此受到共同随机性的影响。
所以,干扰是空间相关的。这在Fig. 5.4进行了展示,展现了
ALOHA传输概率为$p=1/3$的泊松网络中的干扰场的3个快照。

以及,在一个静态网络中,信道接入或者MAC制式选择一变化的顶点子集作为发射机。
例如,在时隙$k$,$\Phi_k\subset \Phi$是发射机的集合,让$I_k$表示原点处在时刻$k$测量到的干扰。
那么$I_k$和$I_j$是相关的,同样是由于共同的随机性$\Phi$。
因此在干扰场中也有时间相关性。

这两种类型的相关性(空间相关性和时间相关性)都值得探索,
因为它们影响了多跳通信和基于重复的传输方案的联合成功概率,
其中发射机重新发送未接收的消息。

我们关注在$\mathbb{R}^2$上由强度$\lambda$的潜在发射机的PPP引起的干扰场,
这些发射机采用ALOHA作为MAC方案。
路径损耗定律$\ell(x)$ 假定具有以下性质。
\begin{itemize}
    \item 它仅仅依赖$\| x \|$.
    \item 它随着$\| x \|$增加而单调减.
    \item 它是可积的,即,$\int_{\mathbb{R}^2} \ell(x) \mathrm{d} x<\infty$
\end{itemize}

例如,一个有效的路径损失模型是
\begin{equation}
    \ell_\epsilon(x)=\frac{1}{\epsilon+\|x\|^\alpha}, \epsilon \in(0, \infty), \alpha>2
\end{equation}

标准奇异路径损失模型$\ell(x)=\| x \|^{-\alpha}$可以由上式取极限$\lim _{\epsilon \rightarrow 0} \ell_\epsilon(x)$得到。
因为是在ALOHA中,每一个节点在每一个时隙进行一个独立决策来发送,
在一个时隙发射机的集合是原始PPP$\Phi$的thinned版本,
导致在时隙$k$的得到的发射机过程$\Phi_k$,也是一个PPP。
这很容易通过考虑void概率来进行验证。
对于一个有界的集合$B\subset \mathbb{R}^2$,我们有
\begin{equation}
    \begin{aligned}
        \mathbb{P}\left(\Phi_k(B)=0\right) &=\mathbb{E} \prod_{x \in \Phi \cap B} 1(x \text { not a transmitter at time } k) \\
        & \stackrel{(\mathrm{a})}{=} \prod_{x \in \Phi \cap B} \mathbb{P}(x \text { not a transmitter at time } k) \\
        &=\mathbb{E}\left[(1-p)^{\Phi(B)}\right] \\
        & \stackrel{(b)}{=} \exp (-\lambda p|B|)
        \end{aligned}
\end{equation}

这里(a)是因为:节点彼此独立地决定是否传输,由于$\Phi$是PPP,
$\Phi(B)$是均值为$\lambda |B|$的泊松随机变量;
这里(b)是因为:泊松随机变量的MGF矩生成函数。

从上面我们可以看到,发射机在时刻$k$处理的void probability是$\exp(-p\lambda |B|)$,
其对应于PPP。我们还观察到$\Phi_k$具有密度$p\lambda$,这是直观的。
在时刻$k$在位置$z$的干扰是:
\begin{equation}
    I_k(z)=\sum_{x \in \Phi} 1\left(x \in \Phi_k\right) h_{x z}(k) \ell(x-z)
\end{equation}

其中,$h_{xz}$是位置$x$和位置$z$之间的衰落,此衰落被假定时间上和空间上都是独立的。

针对具有传输概率$p$的ALOHA,我们的目标是计算随机变量$I_k(u),I_l(v)$的相关系数\\







\begin{remark}
    对于ALOHA,一个有限的collection或者向量$\mathrm{v}=\left(I_{k_1}(z), I_{k_2}(z)\right.  \left.\ldots, I_{k_n}(z)\right) \text { with } n \in \mathbb{N}, k_1 \neq k_2 \neq \ldots \neq k_n, z \in \mathbb{R}^2$
    ,是可交换的,或者对称性依赖的,因为$v$的联合分布并不随着它的成分做permutation而改变。
    可交换性的使用在Kingman(1978)中被研究。
\end{remark}


为了计算随机变量的联合矩,我们使用联合拉普拉斯变换,作用于随机变量$I_k(u)$和$I_l(v)$

\begin{theorem}
    (干扰的联合拉普拉斯变换) \quad $I_k(u),I_l(v),k\neq l$的联合拉普拉斯变换,是
    \begin{equation}
        \mathcal{L}\left(s_1, s_2\right)=\exp \left(-\lambda \int_{\mathbb{R}^2}\left[1-\xi\left(s_1, u-x\right) \xi\left(s_2, v-x\right)\right] \mathrm{d} x\right)
    \end{equation}
    
    其中
    \begin{equation}
        \xi(s, x)=1-p+p \mathcal{L}_h(s \ell(x))
    \end{equation}

    $\mathcal{L}_h$表示对衰落分布做拉普拉斯变换
\end{theorem}

\begin{proof}
    在时刻$k$位置$u\in \mathbb{R}^2$的干扰是
    \begin{equation}
        I_k(u)=\sum_{x \in \Phi_k} h_{x u}(k) \ell(x-u) .
    \end{equation}

    在时刻$l$位置$v\in \mathbb{R}^2$的干扰是
    \begin{equation}
        I_l(v)=\sum_{y \in \Phi_l} h_{y v}(l) \ell(y-v) .
    \end{equation}

    所以,联合拉普拉斯变换是
    \begin{equation}
        \mathcal{L}\left(s_1, s_2\right)=\mathbb{E} \exp \left[-s_1 \sum_{x \in \Phi_k} h_{x u}(k) \ell(x-u)-s_2 \sum_{y \in \Phi_l} h_{y v}(l) \ell(y-v)\right] .
    \end{equation}

    将上式重写为乘积的形式
    \begin{equation}
        \begin{aligned}
            &\mathcal{L}\left(s_1, s_2\right) \\
            &\quad=\mathbb{E} \prod \exp \left(-s_1 1\left(x \in \Phi_k\right) h_{x u}(k) \ell(x-u)\right) \exp \left(-s_2 1\left(x \in \Phi_l\right) h_{x v}(l) \ell(x-v)\right)
            \end{aligned}
    \end{equation}

    因为在ALOHA中,每一个节点都是独立决策是否发送,考虑ALOHA情形,上式变为
    \begin{equation}
        \begin{aligned}
            &\mathcal{L}\left(s_1, s_2\right) \\
            &\quad=\mathbb{E} \prod_{x \in \Phi}\left[1-p+p \exp \left(-s_1 h_{x u}(k) \ell(x-u)\right)\right]\left[1-p+p \exp \left(-s_2 h_{x v}(l) \ell(x-v)\right)\right]
            \end{aligned}
    \end{equation}

    因为衰落被假定是空间时间独立的,考虑到这一点,上式变为
    \begin{equation}
        \begin{aligned}
            \mathcal{L}\left(s_1, s_2\right) &=\mathbb{E} \prod_{x \in \Phi}\left[1-p+p \mathcal{L}_h\left(s_1 \ell(x-u)\right)\right]\left[1-p+p \mathcal{L}_h\left(s_2 \ell(x-v)\right)\right] \\
            &=\mathbb{E} \prod_{x \in \Phi} \xi\left(s_1, x-u\right) \xi\left(s_2, x-v\right) \\
            &=\exp \left(-\lambda \int_{\mathbb{R}^2}\left[1-\xi\left(s_1, x-u\right) \xi\left(s_2, x-v\right)\right] \mathrm{d} x\right) .
            \end{aligned}
    \end{equation}

    最后一步是使用PPP的pgfl。
\end{proof}

使用上面定理中相似的方法,相同时间$k$,不同位置$u,v$,随机变量$I_k(u),I_k(v)$的联合拉普拉斯变换
是
\begin{equation}
    \mathcal{L}_{I(u) I(v)}\left(s_1, s_2\right)=\exp \left(-\lambda \int_{\mathbb{R}^2}\left[1-\mathcal{L}_h\left(s_1 \ell(x-u)\right) \mathcal{L}_h\left(s_2 \ell(x-v)\right)\right] \mathrm{d} x\right)
\end{equation}

要计算$I_k(u)$和$I_l(v)$的乘积的均值,需要计算相关系数
\begin{equation}
    \begin{aligned}
        \mathbb{E}\left[I_k(u) I_l(v)\right] &=\left.\frac{\partial^2}{\partial s_2 \partial s_1} \mathcal{L}\left(s_1, s_2\right)\right|_{\left(s_1, s_2\right)=(0,0)} \\
        &=p^2 \lambda \int_{\mathbb{R}^2} \ell(x-u) \ell(x-v) \mathrm{d} x+\lambda^2 p^2\left(\int_{\mathbb{R}^2} \ell(x) \mathrm{d} x\right)^2 .
        \end{aligned}
\end{equation}

\begin{lemma}
    ALOHA情形,路径损失函数取式子(5.17),干扰$I_k(u)$和干扰$I_l(v)$的空时相关系数是
    \begin{equation}
        \zeta(u, v)=\frac{p \int_{\mathbb{R}^2} \ell(x) \ell(x-\|u-v\|) \mathrm{d} x}{\mathbb{E}\left(h^2\right) \int_{\mathbb{R}^2} \ell^2(x) \mathrm{d} x} .
    \end{equation}
\end{lemma}

\begin{proof}
    因为$I_k(u)$和$I_k(v)$是独立分布的,我们有
    \begin{equation}
        \zeta(u, v)=\frac{\mathbb{E}\left[I_k(u) I_l(v)\right]-\mathbb{E}\left[I_k(u)\right]^2}{\mathbb{E}\left[I_k(u)^2\right]-\mathbb{E}\left[I_k(u)\right]^2}
    \end{equation}

    使用上面的定量关系替换,我们有
    \begin{equation}
        \begin{aligned}
            \zeta(u, v) &=\frac{p \int_{\mathbb{R}^2} \ell(x-u) \ell(x-v) \mathrm{d} x}{\mathbb{E}\left[h^2\right] \int_{\mathbb{R}^2} \ell^2(x) \mathrm{d} x} \\
            & \stackrel{\text { (a) }}{=} \frac{p \int_{\mathbb{R}^2} \ell(x) \ell(x-\|u-v\|) \mathrm{d} x}{\mathbb{E}\left[h^2\right] \int_{\mathbb{R}^2} \ell^2(x) \mathrm{d} x},
            \end{aligned}
    \end{equation}

    其中(a)是因为替换$y=x-u$以及$\ell(x)$仅仅依赖$\| x \|$

    
\end{proof}

我们观察到相关系数并不依赖于时刻$k$和时刻$l$,从式子(5.20)我们得到了如下关于空间相关性的lemma。

\begin{lemma}
    干扰的空间相关系数,即,$I_k(u),I_l(v)$的相关系数,是
    \begin{equation}
        \zeta_{\mathrm{s}}(u, v)=\frac{\int_{\mathbb{R}^2} \ell(x) \ell(x-\|u-v\|) \mathrm{d} x}{\mathbb{E}\left[h^2\right] \int_{\mathbb{R}^2} \ell^2(x) \mathrm{d} x} .
    \end{equation}
\end{lemma}

通过在lemma 5.11中设置$\| u-v \|=0$,我们得到了时间相关系数。

\begin{lemma}
    对于ALOHA,时间相关系数,是
    \begin{equation}
        \zeta_{\mathrm{t}}=\frac{p}{\mathbb{E}\left(h^2\right)}
    \end{equation}

    和$\ell(x)$独立。当衰落是Nakagami-m,相关系数是$\zeta_{\mathrm{t}}=pm/(m+1)$.
    尤其是,对于$m=1$(瑞利衰落),时间相关系数是$p/2$,对于$m\rightarrow \infty$(无衰落),
    时间相关系数是$p$
\end{lemma}

因此相关系数随着$m$增加而增加,即,衰落可以减小相关系数,这是符合我们直觉的。注意到
上面的推导$\int_{\mathbb{R}^2} \ell^2(x) \mathrm{d} x$对于$\ell(x)=\|x\|^{-\alpha}$没有很好的定义。
我们可以使用这样的路径损耗来替代:$\epsilon\rightarrow 0,\ell_{\epsilon}(x)$


\section{中断概率的时间相关性}
我们使用和之前小节的同样的设置,额外地,我们假设
在原点的发射机具备一个目的地$z\in \mathbb{R}^2$。
让$A_k$表示原点可以在时刻$k$连接到目的地$z$的事件。
即,
\begin{equation}
    \mathrm{SIR}=\frac{h_{o z}(k) l(z)}{I_k(z)}>\theta .
\end{equation}

我们假设所有的链路都服从瑞利衰落。我们想要找到
$\mathrm{P}\left(A_k, A_l\right), k \neq l$(这表明两个不同时隙中成功概率的相关性)
成功的联合概率。
推导是基于【一个pgfl的一个良好的应用】进行的。
让$\theta_z=\theta/\ell(z)$,我们有
\begin{equation}
    \begin{aligned}
        \mathbb{P}\left(A_k, A_l\right) &=\mathbb{P}\left(h_{o z}(k)>\theta_z I_k(z), h_{o z}(l)>\theta_z I_l(z)\right) \\
        & \stackrel{(\mathrm{a})}{=} \mathbb{E}\left[\exp \left(-\theta_z I_k(z)\right) \exp \left(-\theta_z I_l(z)\right)\right] \\
        &=\mathbb{E}\left[\exp \left(-\theta_z \sum_{x \in \Phi} l(x)\left[1\left(x \in \Phi_k\right) h_{x z}(k)+1\left(x \in \Phi_l\right) h_{x z}(l)\right]\right)\right] \\
        & \stackrel{(\mathrm{b})}{=} \mathbb{E}\left[\prod_{x \in \Phi}\left(\frac{p}{1+\theta_z \ell(x)}+1-p\right)^2\right] \\
        & \stackrel{(\mathrm{c})}{=} \exp \left(-\lambda \int_{\mathbb{R}^2} 1-\left(\frac{p}{1+\theta_z \ell(x)}+1-p\right)^2 \mathrm{~d} x\right) .
        \end{aligned}
\end{equation}

其中(a)是因为$h_{oz}(k),h_{oz}(l),k\neq l$的独立性;
(b)是因为考虑$h_{xz}(k),h_{xz}(l)$和ALOHA后取的平均;
(c)是因为PPP的pgfl。
根据联合概率,条件概率(假设之前的传输成功或失败)可被得到。

\section*{供拓展阅读的参考书目}
随机几何在无线网络分析中的应用始于Musa\&
Wasylkiwskyj(1978)和Takagi\&Kleinrock(1984)
Sousa和Silvester(1990)、Sousa(1990)和Sousa等人的PPP中的干扰特征,
和Ilow\&Hatzinakos(1998)。Mathar中考虑了格点和购买力平价\&
Mattfeldt(1995)和Haenggi(2009)。二项式点过程中的干扰
在Srinivasa和Haenggi(2010)中进行了分析。

瑞利衰落中,干扰的拉普拉斯变换和成功概率的联系,第一次由Linnartz (1992) and Zorzi
\& Pupolin (1995)给出。
这个结果被Baccelli et al. (2006)推广,Baccelli 他们也给出了干扰和中断对路由的影响的结果。

Baccelli \& Blaszczyszyn (2009) 的两卷书,
卷1中对随机几何给出了介绍,卷2中介绍了MAC设计自组网中路由的详细介绍。

Haenggi\&Gant(2008)的专著中详细讨论了作为点过程建模的无线网络中的干扰以及由此产生的中断概率,
该专著也包含了泊松模型之外的扩展。

传输容量在Weber等人(2005年)中介绍,是Weber\&Andrews(2012年)专著的主题,Ganti\&Haenggi(2009年b)分析了干扰相关性。

关于稳定分布的细节可以在Samorodnitsky \& Taqqu (1994)找到。



\section*{练习题}
\begin{exercise}
    让$\Phi$是一个均匀PPP,让$I_r$表示在原点$o$处测得的干扰,   
    此干扰源自干扰器$\Phi \cap b(o, r)$,证明如果路径损耗律是
    $\bar{\ell}(r)=r^{-3}$,那么
    \begin{equation}
        \mathbb{E}\left(I_{2 r}-I_r\right)=\mathbb{E}\left(I_{\infty}-I_{2 r}\right), \quad \forall r>0
    \end{equation}
\end{exercise}


\begin{exercise}
    让路径损失函数是$\ell(x)=\min \left\{\|x\|^{-\alpha}, 1\right\}$,
    让$\Phi$是$\mathbb{R}^2$上一个强度为$\lambda$的稳定的PPP。干扰是
    \begin{equation}
        I=\sum_{x \in \Phi} h_x \ell(x)
    \end{equation}
    对于 iid.的$h_x$有分布函数$F_h$。
    
    确定$\operatorname{var}(I)$和$\mathbb{E}(I^2)$对于
    $F_h(x)=\mathbf{1}\{x \geq 1\}$以及$F_h(x)=1-\exp (-x)$.
    这里的$\alpha$要满足什么条件才能使方差是有限的?

    使用涉及1st moment和2nd moment的不等式,你可以对$I$的分布给出什么结论?
\end{exercise}



\begin{exercise}
    考虑平面上强度为独立的PPP。考虑路径损耗律
    为$\ell(x)=\|x \|^{-4}$的情形,将来自原点$o$的四个PPP的干扰
    记为$I,I_1,I_2,I_3$。找到常数$c$使满足
    \begin{equation}
        I_1+I_2+I_3 \stackrel{\mathrm{d}}{=} \mathrm{cI} .
    \end{equation}
    试着从$3$的情形向$n$的情形做推广。
\end{exercise}


\begin{exercise}
    设置一个模拟,以验证$\lambda=1$的前一个问题,
    对于相同的路径损耗函数但没有衰落。
    绘制$\alpha \in\{2.1,2.2, \ldots, 5.0\}$
    情形下的平均值和方差,理论曲线也要包括?
    在这个过程中你遇到了什么困难?
\end{exercise}


\begin{exercise}
    理论分析计算吞吐量。考虑一个密度为$\lambda$的PPP$\Phi\subset \mathbb{R}^2$。
    假设瑞利衰落,概率为p的时隙ALOHA作为信道接入协议。
    定义吞吐量是
    \begin{equation}
        T=p(1-p)p_s(p)
    \end{equation}
    
    其中$p_s(p)$是距离r上传输的成功概率,当
    仅考虑干扰:
    \begin{equation}
        p_s=\mathbb{P}(\text{SIR}>\theta)
    \end{equation}
    
    为什么这是一个有意义的吞吐指标?
    
    找到最优的$\theta,\alpha,r$的函数形式的传输概率,
    以及得到的$p_s,T$
    
    如果无线电具有全双工能力,吞吐量定义将如何改变?
    
    导出相应的最佳p并进行比较。
    
    绘制$r=1,\lambda=1,\theta=1,\alpha=4$情形的吞吐$T$-概率$p$
    函数,半双工和全双工的情形都要给出。

\end{exercise}


\begin{exercise}
    模拟来计算吞吐量。对相同的场景写一个模拟,其中
    $r=1,\lambda=1,\theta=1,\alpha=4$。绘制模拟得到的吞吐$T$-概率$p$曲线,
    和理论分析的结果进行比较。
\end{exercise}


\begin{exercise}
    再一次对于相同的场景,假定干扰器并不会衰落(但是来自关心的发射机的信道会)
    你期待成功概率会怎么改变?使用模拟仿真来进行验证。
\end{exercise}


\begin{exercise}
    对于泊松网络,瑞利衰落,ALOHA,Lemma 5.13已经表示干扰的
    时间相关和传输概率$p$成比例。利用Section 5.1.5的方法推导同样的结果:
    首先考虑一个有限的网络,并以节点的数量为条件,然后对节点的(泊松)数量解除条件,让网络面积增长到
    无穷。
    从一个有限的$\epsilon$开始,让$\epsilon\rightarrow 0$,观察路径损失函数$\ell_\epsilon$的
    表现。
\end{exercise}


\begin{exercise}
    对于Section 5.6考虑的成功概率的时间相关,证明如下结论:
    \begin{equation}
        \frac{\mathbb{P}\left(A_k \mid A_l\right)}{\mathbb{P}\left(A_k\right)}>1
    \end{equation}
    这表明成功事件是正相关的。
\end{exercise}






%%%%%%%%%%%%%%%%%%%%%%%%%%%
\chapter{点过程的矩测度}


\section*{练习题}
\begin{exercise}
    说明如下公式:
    \begin{equation*}
        \operatorname{var} \Phi(A)=\alpha^{(2)}(A \times A)+\Lambda(A)-(\Lambda(A))^2        
    \end{equation*}
\end{exercise}

\begin{exercise}
    使用R语言验证例子6.15中所使用近似的准确性
\end{exercise}

\begin{exercise}
    使用R中的`x=clickppp(30)`来手动进入30个点。尝试
    产生$[0,1]^2$上均匀PPP的一个典型实现。使用Kest函数和envelope函数来
    验证是否实际$K$函数和理论$K$函数匹配
\end{exercise}


\begin{exercise}
    找到Matérn聚簇过程的$K$函数
\end{exercise}


\begin{exercise}
    说明对于双泊松聚簇过程,
    \begin{equation*}
        K(r)=\pi r^2+\frac{F(r)}{\lambda_{\mathrm{p}}}
    \end{equation*}

    其中$F(r)$是簇中两个点的距离的分布
\end{exercise}

\begin{exercise}
    考察例子6.17中的Baddeley–Silverman process,设置区域为单位方,
    计算它的强度函数,方差$\text{var} \Phi(B)$对于任意的$B$,K函数。

\end{exercise}

\begin{exercise}
    解释2.8节引入的FryPlot和$K$函数的联系。
\end{exercise}


\begin{exercise}
    干扰。重复问题5.2,使用你自己的关于二阶积密度的知识。
    显式地写出和的平方:
    \begin{equation*}
        I^2=\left(\sum_{x \in \Phi} h_x \ell(x)\right)^2, \quad \ell(x)=\min \left\{\|x\|^{-\alpha}, 1\right\},
    \end{equation*}
\end{exercise}


\begin{exercise}
    一维聚簇过程的二阶矩测度。
    (a)
    (b)
    (c)
\end{exercise}

%%%%%%%%%%%%%%%%%%%%%%%%%%%
\chapter{标记点过程}

默认中文字体
default english font

\section*{练习题}
\begin{exercise}
    (对于平稳过程)定义$K$函数,使用一般的记号$m_x^{(\Phi)}$
\end{exercise}

\begin{exercise}
    确定例子7.8中的标记点过程的the mark distrbution,
    如果$\Phi$是一个强度为$\lambda$的均匀PPP。
    Repeat for the case where Φis a
perturbed square lattice (see Definition 2.16) with iid perturbation vectors that are
uniform on b(o, 1/2)
\end{exercise}


\begin{exercise}
    让$\Phi$是一个强度为$\lambda$的PPP,用标记
    \begin{equation*}
        m_x=\Phi(b(x, R))-1, \quad x \in \Phi
    \end{equation*}
    
    其中$R>0$。来得到一个标记点过程。展示mark-correlation函数是
    \begin{equation*}
        k_{m m}(r)=1+\frac{|b(o, R) \cap b(r, R)|}{\pi r^2}
    \end{equation*}
\end{exercise}


\begin{exercise}
    对于式子7.11描述的标记过程,描述$\Phi_{[[0 p] \times\{1\}]}$的性质。
    
    如果基底过程$\Phi$是一个强度$\lambda$的PPP。
    什么MAC制式可以得到这个过程?
\end{exercise}

\begin{exercise}
    给出Campbell’s theorem的形式证明。
    这里考察的对象是(7.8)中的稳定标记点过程。
\end{exercise}


\begin{exercise}
    描述泊松双极网络模型中的发射机和接收机的叠加,是一个聚簇过程。
\end{exercise}

\begin{exercise}
    考虑一个无线网络,其中发射机来自一个稳定的强度为$\lambda$的点过程。todo  todo
\end{exercise}
\chapter{Conditioning和Palm Theory}

\section{导论}

\subsection{Conditioning和典型的点}
点过程理论中的Palm概率或Palm测度是
给定点过程在某个位置包含点的事件。它还
将过程的“典型点”的概念形式化。非正式地,典型的
点源于一个选择过程,其中每个点都有相同的机会
选择。这个想法需要在数学上精确,尤其是在
无限点过程。例如,根据某些采样选择的点
程序,例如最接近原点的程序,不是典型的,因为它已经
以特定的、确定的方式选择。直观地看,Palm分布是
条件点过程分布,给定一个点(典型点)存在于
特定位置。

这种类型的conditioning有时指代interior conditioning,因为conditioning 作用在
$x\in \Phi$,问题是点过程在$x$之外表现如何?
相形之下, Papangelou conditional intensity是基于exterior conditioning的,
因为conditioning作用在$\mathbb{R}^d \backslash\{x\}$,问题是
有一个点在$x$的概率是多大。这两个概念相互对偶。

如果点过程是原子的,就像die process一样,在其中一个原子的位置上调节一个点不会造成任何困难。

\begin{example}
    用$P_o$表示Example 2.1中die process的分布做原点$o\in \mathbb{R}^2$的condition.
    让$E$是所有点$\Phi(\mathbb{R}^2)$的数目是奇数这一事件。我们有$P(E)=1/2$但是$P_o(E)=1$.
    相似地,用$\mathbb{E}_o$指代期望,我们有$\mathbb{E} \Phi(b((1,1), \in))=1 / 2\text{ and } \mathbb{E}_o \Phi(b((1,1), \in))=1 / 3 \text { for } 0<\in<1$
\end{example}

当我们考虑扩散点过程时,(??括号里的谁爱翻译谁翻译,我看不懂??),因为,
在这个case,我们做condition的事件具有概率$0$. 甚至,这样的conditioning可以是
非常直觉性的,正如下面这个例子所展示的。

\begin{example}
    考虑概率空间$\left(\Omega, \mathcal{B}^2, v_2\right)$
\end{example}

\section*{练习题}

\begin{exercise}
    让$\Phi$是一个随机旋转的die processs,来自例子2.1,
    让$B=[-1,1]^2$。确定$\mathbb{E} \Phi(B), \mathbb{E}_o \Phi(B), \text { and } \mathbb{E}_{(-1,1)} \Phi(B)$
\end{exercise}

\begin{exercise}
    the die cluster process.考虑一个parent intensity为$\lambda_p$的泊松聚簇过程,其中
    每个簇是一个独立且随机的旋转die process.
    再一次的,对于$B=[-1,1]^2$,找到$\mathbb{E} \Phi(B), \mathbb{E}_o \Phi(B), \text { and } \mathbb{E}_{(-1,1)} \Phi(B)$,
    画出K函数$K(r)$图像.
\end{exercise}

\begin{exercise}
    考虑$\mathbb{R}^2$上的混合泊松点过程,其中$\lambda$是一个均值$p$的伯努利随机变量。
    确定$\Lambda\left([0,1]^2\right)=\mathbb{E} \Phi\left([0,1]^2\right)$和
    $\mathbb{E}_o \Phi\left([0,1]^2\right)$
\end{exercise}

\begin{exercise}
    $\mathcal{K}(\mathrm{B})=\mathcal{K}(-\mathrm{B})$成立吗?给出证据或者给出反例。
\end{exercise}


\begin{exercise}
    证明,对于一般的PPP,$\lambda(x, \Phi) \equiv \lambda(x)$
\end{exercise}

\begin{exercise}
    画出式子(8.10) Georgii-Nguyen-Zessin formula的稳定的(形式/表格?),

    使用一个hard-core process.
    选择一个函数$h$,使得内部的左边的conditioning
    和Scaling the intensity有相同的效果
\end{exercise}

\begin{exercise}
    找到Neymann-Scott聚簇过程的$J$函数,使用聚簇Palm测度$\mathcal{C}_o$来表示。
\end{exercise}

\chapter{渗流,连通性,覆盖的介绍}


\section{动机}

在这部分,我们关心一个网络或者一个图的特定的全局性质,例如是否存在
一个巨大的连通分量。我们将要用到的主要的新的数学工具是渗流理论。
渗流理论开始于大约50年前,当时是作为一个数学的框架去研究孔状媒质的行为。
它试图回答下面这些问题:
- 如果一个石头变得潮湿了,水会渗过石头吗?
- 如果一个材料包括2种成分,其中之一是完美绝缘体,resistance是有限的概率是?
- 一场森林大火蔓延到整个森林的概率?
- 一种病毒全球传播的概率?
- 大多数的网络连通的概率?

事实证明,当网络参数发生变化时,某些关键事件(如网络中存在巨大的连接组件)会突然出现。这种现象称为相变。
在图形或网络的上下文中,渗流与连通性和覆盖率有关。这是本部分的其他两个主题。



\section{什么是渗流?}
渗流的基础模型是:二维的网格,即,图$\mathbb{L}^2 \triangleq G(V,E)$,其中$V=\mathbb{Z}^2, E= \{ (x,y) \in V^2 : ||x-y||=1 \}$。所以$\text{deg} \nu =4 \; \nu \in V$。
现在,我们通过独立地以概率$1-p$移除每一条边,把格 转为 随机图。
等价地,我们每条边保留下来的概率是$p$。
这就是所谓的bond 渗流模型。
如果不是边而是顶点的情形,被称为site 渗流模型。
我们需要的其他的成分是 连通分量。对于一般的图来说,连通分量的定义如下:







todo todo todo todo
todo todo todo todo
todo todo todo todo
todo todo todo todo



\chapter{Bond percolation和site percolation}

\section{随机树和分支过程}
在涉及到图和它们的性质的问题中,经常是有用的去首先考虑无环的图,即,树。

\subsection{正则分支树的渗流}






todo todo todo todo
todo todo todo todo
todo todo todo todo
todo todo todo todo

\section*{练习题}
\begin{exercise}
    考虑2-branching树(也被称为二叉树),具有二项分布的后代分布,确定Percolation的概率,并且画出来,
    找到(10.4),(10.5),(10.6)的临界指数$\beta,\gamma,\delta$
\end{exercise}

\begin{exercise}
    展示二叉树的每一代中$(Z_n)$的成员数量的序列,从马尔可夫链条的角度去考虑,给出转移概率。
\end{exercise}


\begin{exercise}
    对于【后代概率分布是均值为$\lambda$的Possion分布】的分支过程,找到它灭绝的概率$\eta$
\end{exercise}

\begin{exercise}
    让$T$是分支过程的全体后代(total progency),即,
    \begin{equation*}
        T \triangleq \sum_{n=0}^{\infty} Z_n
    \end{equation*}

    让它的生成函数是$G_T(s)=\mathbb{E}(s^T)$.说明
    \begin{equation*}
        G_T(s)=s G\left(G_T(s)\right)
    \end{equation*}

    其中,$G_{(s)}=\mathbb{E}(s^X)$是后代分布的生成函数。
    
    提示。首先建立,如果$Z_1=i$,
    \begin{equation*}
        T \stackrel{\mathrm{d}}{=} 1+\sum_{k=1}^i T_k
    \end{equation*}
    
    其中,$T_k$是iid.的随机变量,和$T$有相同的分布。
\end{exercise}

\begin{exercise}
    对于一个平均后代为$\mu<1$的分支过程,说明,期望的总体后代是
    \begin{equation*}
        \mathbb{E} T=\frac{1}{1-\mu} .
    \end{equation*}
\end{exercise}


\begin{exercise}
    使用随机游走解释递归公式(10.3)
\end{exercise}


\begin{exercise}
    确定(渐进情形下),图10.17中lozenge lattice的渗流阈值
\end{exercise}


\begin{exercise}
    中枢边。考虑$\mathbb{L}^2$上一个bond percolation模型,事件$A=\{ o\leftrightarrow \partial B(1) \}$.
    让四个边是$e_1,\cdots,e_4$。考虑一个显式的表达式,对于事件$W_i$使得:边$e_i,i\in[4]$是中枢的,
    确定$\mathbb{P}_p(W_i)$。接着找到
    $\mathbb{E}_p(N(A))$和$\mathbb{E}_p(N(A) \mid A)$。
    当$p\rightarrow 0$的时候,这两个期望表现如何?这个行为是我们所预料到的吗?
    解释这个现象。
\end{exercise}


\begin{exercise}
    Russo公式。
\end{exercise}


\begin{exercise}
    证明,对于$\mathbb{L}^2$上的bond percolation,$p_c=p_c^{\prime}$成立。
    其中,
    \begin{equation*}
        p_{\mathrm{c}}^{\prime}=\sup \{p: \chi(p)<\infty\}
    \end{equation*}

    $\chi(p)$是原点处component的大小的期望值(正如10.1.4节定义的那样)
\end{exercise}

% 10.11
\begin{exercise}
    当证明infinite components数量不可能是无限的时候,我们使用了trifurcations(三叉点)的概念。
    对具有three trifurcations的$n$,在diamond $S(n)$中画出一个configuration.
    指明这样的边界点,满足它们被假定认为和一个infinite open cluster连通。
    解释为什么bifurcations(二叉)不能得到这个结论。
\end{exercise}






\chapter{随机几何图和连续渗流}

\section{导论}
在这一章,我们介绍随机几何图(RGGs)作为之前章节考察过的lattice model的一般化。

RGG的顶点嵌入在$\mathbb{R}^d$,边依赖于顶点的空间位置。

位置经常使用点过程建模,考虑顶点的位置确定性地或概率性地建立边。
我们将会术语vertex 和 point 作为等价术语。
对于一个特定的图,我们将会限制渗流概率。因为顶点可以保证任意的位置在相同的空间(经常是$\mathbb{R}^d$,称为continuum percolation。


\begin{definition}
    (基础随机几何图RGG) 让$\Phi= \{ x_i \} \subset  \mathbb{R}^d $ 是一个(简单的)点过程。
    对于每一个(无序的)点对 $\{ x,y \} \subset \Phi$,独立地以概率$\xi(x-y)$加入一条边,其中$\xi : \mathbb{R}^d \to [0,1]$是(translation-invariantt)连通函数。 $\xi$的有效面积(or volume)是 
    \begin{equation*}
        \mathcal{A}_\xi  \triangleq \int \limits_{\mathbb{R}^d} \xi(y) \mathrm{d} y
    \end{equation*}
    得到的图$G=(\Phi, E)$是一个随机几何图。
\end{definition}

因此在RGG里有两个随机因素,点的位置,以及边是否存在。点过程可以是finite或者infinite。如果它是finite,一个经常使用的模型是(广义的)binomial点过程,经常 on  the unit cube $[0,1]^d$。如果它是infinite,典型的模型是PPP(泊松点过程)。








todo todo todo todo
todo todo todo todo
todo todo todo todo
todo todo todo todo

\section*{练习题}

\begin{exercise}
    利用和a Galton–Watson 分支过程的比较,找到Gilbert圆盘图的渗流概率$\theta(r,\lambda)$的数值的界
\end{exercise}   

\begin{exercise}
    对于Gilbert随机图,说明$p=n^{-1-1/k}$是出现$k+1$个节点的树的阈值函数
\end{exercise}

\begin{exercise}
    在Gilber随机图中,说明,如果$np<1$,包括顶点1的成分的期望大小的界是
    \begin{equation*}
        \mathbb{E}|C(1)| \leq \frac{1}{1-n p}
    \end{equation*}
\end{exercise}

\begin{exercise}
    对于空时SINR图,找到一个顶点入度的上界,此上界仅依赖门限$\theta$
\end{exercise}

\begin{exercise}
    确定空时圆盘干扰快照图中一个接收机的平均入度。一个发射机的出度由引理11.19确定。使相似的计算
    来找到一条“捷径”,这样你需要很少的计算来得到答案。
\end{exercise}

\begin{exercise}
    对于Poisson隐私图,说明,入度$N_{in}$有矩生成函数mgf
    \begin{equation*}
        \mathbb{E} \mathrm{e}^{t N_{\text {in }}}=\mathbb{E}\left(\exp \left(V\left(\mathrm{e}^t-1\right) / \lambda\right)\right)
    \end{equation*}
    其中$V$是和强度为1的PPP典型点关联的Voronoi cell的体积。
    对于一维的例子,给出mgf的具体的表达式,并且确定$\mathbb{P}(N_{in}=k)$
\end{exercise}

\begin{exercise}
    对于Possion隐私图上的out-percolation,说明
    \begin{equation*}
        \theta(\lambda) \leq \max \{0,1-\lambda\}
    \end{equation*}
    
    其中$\theta(\lambda)$是$\vec{G}\lambda$中原点属于一个无穷out-component的概率。
\end{exercise}


\begin{exercise}
    对于无向Poission隐私图$G_\lambda$,找到临界强度的上界。利用一个face percolation argument。
    原点处的component是finite如果它被circuit of closed faces包围。
    定义a face 是 closed,如果这个configuration包括监听者,满足s it impossible for an edge to
    form from the inside of the circuit to the outside。
    如果那个configuration以至少$1/2$概率发生,$o$处component是finite a.s. 
    注意得到的界是相对弱的。
\end{exercise}
\chapter{连通性}

\section{导论}
在一些case中,percolation渗流并不是充分的,因为它仅仅保证无穷数量的节点在图或网络的某个地方连通。
事实上,连通节点的比例可能非常小;很容易会出现:大部分的节点和the infinite component不连通。
连通性是一个更加严格的条件;它要求所有节点都连接almost surely。

因为连通性参数,例如圆盘的半径,经常是无穷的(对于无穷图),我们首先聚焦在infinite graph,
比方说一个面积为$n$的方形区域,接着研究图在$n\rightarrow \infty$的连通性行为。

我们首先讨论的是随机梯格的连通性

\subsection{随机梯格的连通性}
我们也考虑网格$\mathbb{L}^2$上一个大小为$n\times n$个顶点的盒子$B(x)$。
正如bond percolation中的那样,每一条边以概率$p$发生。我们想要找到
$p$要满足什么条件,使得$B(n)$的所有顶点是联通的,在$n\rightarrow\infty$的渐进情况下。
我们建立了$\mathbb{L}^2$上bond percolation的临界概率$p_c=1/2$。
会议$\theta(p)$指代原点$o$属于the infinite component的概率。

如果$p>1/2$,在网格上存在一个唯一的infinite component,$B(n)$的每一个顶点以概率$\theta(p)$连接上它。
因此在$B(n)$存在一个component具备mean size$n^2\theta(p)$。
事实上,可以得到,在极限$n\rightarrow \infty$下,$\theta(p)$正好是
$B(n)$中连通节点的比例。

因此,对于full connectivity,必须$p\rightarrow 1$.问题是在$n$增长时,大概要多大的n就基本满足。
让$p$是$n$的函数,平均孤立节点数量的渐进是
\begin{equation*}
    \mathbb{E} N_{\text {isol }}(n) \sim n^2(1-p(n))^4, \quad n \rightarrow \infty
\end{equation*}

因为每个(内部的)节点是以概率$(1-p(n))^4$孤立的。边界节点是渐进可忽略的,因为他们的数量是
$\Theta(n) $.选择
\begin{equation*}
    p(n)=1-\frac{c(n)}{\sqrt{n}}
\end{equation*}

导致$\mathbb{E}N_{isol}(n)\sim c(n)^4$。所以对于连通性$c(n)\rightarrow 0$是必须的。
下面的定理给了一个更清晰的结果。

\begin{theorem}
    (随机梯格中的孤立节点) 让$\mathbb{E}N_{isol}(n)$是$B(n)$中孤立节点的数量。那么$\mathbb{E}N_{isol}(n)$收敛到
    一个均值为$M$的泊松随机变量,当且仅当
    \begin{equation*}
        n^2(1-p(n))^4 \rightarrow M, \quad n \rightarrow \infty
    \end{equation*}
\end{theorem}

\begin{proof}
    我们使用所谓的Chen-Stein方法来展示孤立节点的数量在给定条件下收敛到一个泊松分布。
    我们要应用的特定结果如下
\end{proof}

\begin{lemma}
    (总共variaiton distance的Chen-Stein界) 让$(A_i),i\in [n]$是
    指示随机变量的collection。让$S=\Sigma_{i=1}^n A_i$,其中$\mathbb{E}A_i=p_i$,以及
    \begin{equation*}
        M \triangleq \sum_{i=1}^n p_i
    \end{equation*}
    
    在$M<\infty$的假设下。因此$\mathbb{E}S=M$。那么,如果$A_i$是随机变量$(X_k)$的增函数
    \begin{equation*}
        \|S-\operatorname{Poi}(M)\|_{\mathrm{TV}} \leq \frac{1-\mathrm{e}^{-M}}{M}\left(\operatorname{var} S-M+2 \sum_{i=1}^n p_i^2\right)
    \end{equation*}
    其中$\operatorname{Poi}(M)$是均值为$M$的泊松随机变量,
    $\|\cdot\|_{\mathrm{TV}}$是total variation distance,定义是
    \begin{equation*}
        \|P-Q\|_{\mathrm{TV}} \triangleq \sup \{Y \in \mathcal{F}:|\mathbb{P}(Q \in Y)-\mathbb{P}(P \in Y)|\}
    \end{equation*}

    对2个随机变量$P$和$Q$定义相同的概率空间with $\sigma$- algebra $\mathcal{F}$
\end{lemma}

在我们的情况中,潜在的独立随机变量$(X_k)$是


todo todo 
todo todo
todo todo
todo todo


\section*{练习题}

\begin{exercise}
    不使用本章的任何结果,说明:对于圆盘图,
    \begin{equation*}
        \pi r^2(n)=5\pi \log n
    \end{equation*}
    是a.a.s. connectivity的充分条件。

    提示,面积为$n$的方形区域的分拆为一些面积为$\sqrt{\log n}$的方形区域。
    在每个方形上定义一个事件,说明所有事件发生的概率趋于1,在$n\rightarrow \infty$的条件下。
    接着证明这个联合事件表明:在给定的$r(n)$的条件下,可以得到连通性。
\end{exercise}

\begin{exercise}
    估计在定理12.8的下界的讨论中,"likely scenario for disconnectedness"事件的概率。
    说明是$k(n)>\log n/8$是必要的。
\end{exercise}

\begin{exercise}
    考虑1st最近邻图,顶点集为$\Phi$,其中$\Phi$是$\mathbb{R}^d$一个强度为1的PPP。
    对于一般的维度数$d$,描述它的成分的性质。
    接管关注$d=1$的case,陈述典型成分的大小的分布。
\end{exercise}

\begin{exercise}
    考虑一个$\sigma^2=1/2$的高斯PPP(见例子2.2),所以强度是$(n/\pi)e^{-r^2}$.在原点加上一个点,
    通过连接距离在$s$以内的2个点定义圆盘图。让$\rho$是最大的半径$u$使得所有的在$b(o,u)$内部的点和
    $o$连接。说明,假设$Rs<<1$,球$b(o,R)$内的孤立节点数量可以近似为
    \begin{equation*}
        \mathbb{E} N_{\mathrm{isol}}(R) \approx \frac{\exp \left(-n s^2 \mathrm{e}^{-R^2}\right)-\exp \left(-n s^2\right)}{s^2} .
    \end{equation*}
    接着,假设$n\rightarrow \infty$以及$n^{-1 / 2} \ll s \ll(\log n)^{-1 / 2}$,说明
    \begin{equation*}
        \rho \approx \log \left(n s^2 / \log \left(1 / s^2\right)\right)
    \end{equation*}
\end{exercise}


\chapter{覆盖}

\section{导论}
在几何的语境中和连通性相对的,是覆盖。
这里我们假设一个随机集被附属到点过程的每一个点上,
主要的问题是,是否所有这些随s机集的并覆盖一个特定的目标区域(可能$\mathbb{R}^d$)。


在无线网路的上下文中,这样的问题在蜂窝系统中自然而然地被提出,那里的目标是:
每个蜂窝基站或者传感器可以覆盖一定的区域(随机的),可能做到对我们感兴趣的区域的覆盖吗?

\section{Germ-grain模型和布尔模型}
\subsection{Germ-grain模型}
给出germ-grain model的定义。
点过程的点(germs)
iid的紧随机集(grains)
每个点(germ)都有一个附属的区域(grain)

\begin{definition}
    (Germ-grain模型)让$\Phi=\{ x_i \}$是$\mathbb{R}^d$上的点过程,称为germs.
    $(S_1,S_2,\cdots)$是一系列随机非空集的Collection,称为grains。那么并
    \begin{equation*}
        \Xi=\bigcup_{i \in \mathbb{N}} x_i+S_i
    \end{equation*}

    就是一个germ-grain model。$\Xi\subset \mathbb{R}^d$是占据区域,
    $\mathbb{R}^d\backslash\Xi$称为未占据区域,或者空区域。
\end{definition}

$(x_i+S_i)$被称为覆盖过程。一块区域$S$附属在一个点$x$上,所以下标也可以写$x$,像下面这样:
\begin{equation*}
    \Xi=\bigcup_{x \in \Phi} x+S_x \subset \mathbb{R}^d
\end{equation*}

一个点$y$被称为被覆盖,如果$y\in \Xi$.

如果集合$S_x$或者$S_i$是可数的,$\Xi$是聚簇点过程,正如3.4节描述的那样,
$|S|=v_d(S)=0$.希望覆盖一个不可数的$\mathbb{R}^d$的子集,我们需要
$|S|>0$来取得正概率。

\begin{definition}
    (Vacancy)vacancy$V(\mathcal{R})$是未被覆盖区域$\mathcal{R} \subseteq \mathbb{R}^d$的一部分的勒贝格测度
    \begin{equation*}
        V(\mathcal{R}) \triangleq|\mathcal{R} \backslash \Xi|=\int_{\mathcal{R}} \chi(y) \mathrm{d} y,
    \end{equation*}

    其他
    \begin{equation*}
        \chi(y)=1(y \notin \Xi)=\prod_i 1\left(y \notin S_i+x_i\right) .
    \end{equation*}

    每个$\Xi$的联通子集形成了component;在覆盖过程的语境中,components也可以说成clumps.

\end{definition}

\begin{definition}
    我们说,一个集$\mathcal{R}\subset \mathbb{R}^d$被区域$\Xi$覆盖,
    如果$\mathbb{P}(V=0)=1$
\end{definition}

\begin{note}
    条件$\mathbb{P}(V=0)=1$并不指示$\mathcal{R}\subset \Xi$,因为
    $\mathcal{R}$的一个低维度的子集可能被覆盖。所以,严格来说,$\mathbb{P}(V=0)=1$仅仅是$\mathcal{R}\subset \Xi$的充分条件。
    然而,如果grains $S$是闭集,区域$\mathcal{R}$是一个有界开集,这两者是等价的,即,
    $\mathbb{P}(V=0)=1\Leftrightarrow \mathcal{R}\backslash\Xi=\varnothing$
\end{note}

除了覆盖的问题,一个重要的问题是:是否要去覆盖的区域包括一个无穷大小的clump。
如果确实如此,覆盖过程被称为percolate。
connectivity,coverage和percolation的对应关系在Box13.1进行了总结。


大多数tractable的germ-grain模型是布尔模型,定义如下:
\begin{definition}
    (布尔模型)一个布尔模型首先得是一个germ-grain模型,germ点过程是uniform PPP,
    grains $S_i$是iid.的。
\end{definition}

有时,germ-grian模型中,germs是来自lattice的,那么我们称这个模型是
lattice Boolean model。一个典型布尔模型的case是,grains都是iid.的球。
对此,重要的结果是:每一个位置被覆盖的次数是泊松随机变量。

\begin{theorem}
    (布尔模型中的vacancy and coverage) 在PPP具有强度$\lambda$的布尔模型中,
    一个位置没有被覆盖的概率是$\exp (-\lambda \mathbb{E}|S|)$,它也是$\mathbb{R}^d$没有被占据的比例。
    紧接着,如果$\mathbb{E}|S|<\infty$,那么$\Xi\neq \mathbb{E}^d$. 进一步,每一个location覆盖的次数是均值为
    $\lambda\mathbb{E}|S|$的泊松随机变量。
\end{theorem}

\begin{proof}
    让$M$是原点被覆盖的次数
    \begin{equation}
        M=\sum_{x \in \Phi} 1\left(o \in x+S_x\right) .
    \end{equation}
    
    让$F(y)=\mathbb{P}(y\in S)$,有$\mathbb{E}|S|=\int_{\mathbb{R}^d} F(y) \mathrm{d} y .$。取期望,有
    \begin{equation*}
        \begin{aligned}
            \mathbb{E} M &=\mathbb{E}\left[\sum_{x \in \Phi} \mathbb{E}\left(1\left(o \in x+S_x\right) \mid \Phi\right)\right] \\
            &=\mathbb{E}\left[\sum_{x \in \Phi} F(-x)\right] \\
            & \stackrel{\text { (a) }}{=} \lambda \int_{\mathbb{R}^d} F(-y) \mathrm{d} y \\
            &=\lambda \mathbb{E}|S|
            \end{aligned}
    \end{equation*}

    其中(a)是因为Campbell定理,定理4.6;(b)是因为如下的公式:
    \begin{equation*}
        \mathrm{Ee}^{t 1\left(o \in x+S_x\right)}=1-F(-x)+e^t F(-x)
    \end{equation*}

    所以$M$的矩生成函数和泊松随机变量的矩生成函数形式一致,所以$M$服从泊松分布。
\end{proof}

这个定理的可以产生the 1st and 2nd moments of the vacancy.

\begin{corollary}
    (1st and 2nd moments of the vacancy) 对于布尔模型,vacancy的前2个矩可以由下面给出
    \begin{equation*}
        \mathbb{E}(V(\mathcal{R}))=|\mathcal{R}| \exp (-\lambda \mathbb{E}|S|)
    \end{equation*}
    
    以及
    \begin{equation*}
        \mathbf{E}\left(V(\mathcal{R})^2\right)=\iint_{\mathcal{R}^2} \exp \left(-2 \lambda \mathbb{E}|S|+\lambda \mathbb{E}\left(\left|\left(y_1-y_2+S\right) \cap S\right|\right)\right) \mathrm{d} y_1 \mathrm{~d} y_2 .
    \end{equation*}
\end{corollary}


\begin{proof}
    均值的推导是
    \begin{equation*}
        \mathbb{E} V=\int_{\mathcal{R}} \mathbb{P}(x \notin \Xi) \mathrm{d} x=\int_{\mathcal{R}} \mathbb{E} \chi(x) \mathrm{d} x=|\mathcal{R}| \exp (-\lambda \mathbb{E}|S|)
    \end{equation*}

    对于2nd moment,推导是
    \begin{equation*}
        \begin{aligned}
            \mathbb{E}\left(V^2\right) &=\iint_{\mathcal{R}^2} \mathbb{P}\left(y_1 \notin \Xi, y_2 \notin \Xi\right) \mathrm{d} y_1 \mathrm{~d} y_2 \\
            &=\iint_{\mathcal{R}^2} \mathbb{E}\left(\chi\left(y_1\right) \chi\left(y_2\right)\right) \mathrm{d} y_1 \mathrm{~d} y_2
            \end{aligned}
    \end{equation*}

    其中,
    \begin{equation}
        \begin{aligned}
            \mathbb{E}\left(\chi\left(y_1\right) \chi\left(y_2\right)\right) &=\mathbb{P}\left(\forall i: y_1 \notin x_i+S_i, y_2 \notin x_i+S_i\right) \\
            & \stackrel{(\mathrm{a})}{=} \mathbb{P}\left(\forall i: x_i \notin y_1-y_2+S_i, x_i \notin S_i\right) \\
            &=\mathbb{P}\left(\forall i: x_i \notin\left(y_1-y_2+S_i\right) \cup S_i\right) \\
            & \stackrel{\text { (b) }}{=} \exp \left(-\lambda \mathbb{E}\left(\left|\left(y_1-y_2+S\right) \cup S\right|\right)\right) \\
            &=\exp \left(-2 \lambda \mathbb{E}|S|+\lambda \mathbb{E}\left(\left|\left(y_1-y_2+S\right) \cap S\right|\right)\right)
            \end{aligned}
    \end{equation}

    其中(a)是因为平稳性,(b)是因为定理13.5
\end{proof}

如果$E(|S|)=\infty$,那么:如果$\lambda>0$,即,没有临界的密度。,那么每个component的germ数量是无穷的 a.s. 。
否则,主要问题是:是否$\Xi$中包括原点的component,无限的概率是正的。
(假设原点是PPP的一部分);以及,$\mathbb{R}^d$是否被覆盖。对于圆盘图来说,可以说明,the infinite component是唯一的a.s. 。

\section{具备固定圆盘的布尔模型}
\subsection{单个覆盖}
研究最多的是:把固定半径$r$的圆盘(或者球)当作grain的布尔模型。
在这个case中,我们写作$\Xi=\Phi \oplus S$,这个覆盖过程和Gilbert圆盘图等价。
事实上,从圆盘图,我们立马可以得到coverage percolation problem:如果$r>r_c/2$,其中$r_c$是
圆盘图上渗流的临界半径,布尔模型有一个infinite component,since then覆盖过程的2个圆盘重叠如果他们的顶点在圆盘图中连通。

这个模型的一个基础的应用是,传感器网络。
如果传感器能够检测半径$r$以内的事件,$\Xi$是传感器网络覆盖的区域。
Figure 13.1展示了这些布尔模型的2个例子。

我们关心于:在半径为$r$的圆盘上,面积为$a$的方形区域,在$n\rightarrow\infty$的时候被覆盖a.a.s. ,
需要的条件是什么。我们通过三步来解决这个问题。首先我们找到一个仅仅依赖mean vacancy的必需的条件。
第二步,我们使用vacancy的方差来在必需条件的基础上进行改善。第三步,我们展示这个第二必要条件也是必要的。

平面覆盖的第1必要条件

考虑2维的例子,最基础的问题是,$r$需要多大来为了保证整个平面被覆盖,因为我们可能总是放缩$\lambda$和$r$同时保持$\lambda r^2$为常数,
不失一般性,我们固定$\lambda=1$。

原点没被覆盖的概率是PPP中没有点在距离$r$以内,即,
\begin{equation*}
    \mathbb{E}_\chi(o)=\exp \left(-\pi r^2\right)
\end{equation*}

这对$\mathbb{R}^2$中的所有点都成立,所以面积$n$的方形区域的期望vacancy是
\begin{equation*}
    \mathbb{E} \mathrm{V}\left([0, \sqrt{n}]^2\right)=\mathbb{E}\left|[0, \sqrt{n}]^2 \backslash \Xi\right|=n \exp \left(-\pi r^2\right)
\end{equation*}

这也可以从Corollary 13.6得到。

覆盖的一个必要条件是:当$n\rightarrow\infty$的时候,$\mathbb{E}V([0,\sqrt{n}]^2)\rightarrow 0$. 这由下面这个式子保证:
\begin{equation}
    \pi r^2=\log n+\omega(1), \quad n \rightarrow \infty,
\end{equation}

它也是连通性的相同的条件。然而,$\mathbb{E}V\rightarrow 0$对于覆盖来说并不充分,因为
它并没有保证$\mathbb{P}(V=0)\rightarrow 1$. 
这个non-implication $\mathbb{E}V(n) \not \Rightarrow \mathbb{P}(V=0)\rightarrow 1$为这个事实提供了例子:
均值的收敛并不能表明a.s. 的收敛。

平面覆盖的第2必要条件

$\mathbb{P}(V=0)$的边界可以得到,如果vacancy的2nd moment(或者方差)也成为已知的。
写$V=V1(V>0)$,应用Cauchy-Schwartz不等式,我们有
\begin{equation*}
    \mathbb{E} V=\mathbb{E}(V 1(V>0)) \leq\left(\mathbb{E}\left(V^2\right) \mathbb{P}(V>0)\right)^{1 / 2},
\end{equation*}

因此
\begin{equation}
    \mathbb{P}(V>0) \geq \frac{(\mathbb{E} V)^2}{\mathbb{E}\left(V^2\right)}=1-\frac{\operatorname{var} V}{\mathbb{E}\left(V^2\right)} \quad \Longrightarrow \quad \mathbb{P}(V=0) \leq \frac{\operatorname{var} V}{\mathbb{E}\left(V^2\right)}
\end{equation}

因此我们需要去计算方差,我们可以使用Corollary 13.6 来计算方差,
vacancy的协方差可以从(13.2)算出,为
\begin{equation*}
    \operatorname{cov}\left(\chi\left(y_1\right) \chi\left(y_2\right)\right)=\exp (-2 \lambda \mathbb{E}|S|)\left(\exp \left(\lambda \mathbb{E}\left(\left|\left(y_1-y_2+S\right) \cap S\right|\right)\right)-1\right)
\end{equation*}

因此,
\begin{equation*}
    \begin{aligned}
        \operatorname{var} V(\mathcal{R}) &=\iint_{\mathcal{R}^2} \operatorname{cov}\left(\chi\left(x_1\right), \chi\left(x_2\right)\right) \mathrm{d} x_1 \mathrm{~d} x_2 \\
        &=\exp \left(-2 \pi r^2\right) \iint_{\mathcal{R}^2}\left(\exp \left(\left|b(o, r) \cap b\left(y_1-y_2, r\right)\right|\right)-1\right) \mathrm{d} y_1 \mathrm{~d} y_2 \\
        & \sim|\mathcal{R}| \mathrm{e}^{-2 \pi r^2} \int_{\mathbb{R}^2}(\exp (|b(o, r) \cap b(y, r)|)-1) \mathrm{d} y \quad \text { as }|\mathcal{R}| \rightarrow \infty
        \end{aligned}
\end{equation*}

因为渐进地,边界影响消失了。2个半径为$r$的圆盘在距离$h$的交区域,
\begin{equation*}
    A(h)=2 r^2 \arccos \left(\frac{h}{2 r}\right)-\frac{h}{2} \sqrt{4 r^2-h^2}, \quad h<2 r .
\end{equation*}

因此
\begin{equation*}
    \operatorname{var} V(\mathcal{R}) \sim|\mathcal{R}| \mathrm{e}^{-2 \pi r^2} 2 \pi \int_0^{2 r} h\left(\mathrm{e}^{A(h)}-1\right) \mathrm{d} h
\end{equation*}

因为$A(h)\geq \pi r^2-2rh$,积分是紧的下界,
\begin{equation*}
    \int_0^{2 r} h\left(e^{A(h)}-1\right) \mathrm{d} h>\int_0^{\pi r / 2} h\left(e^{\pi h^2-2 h r}-1\right) \mathrm{d} h
\end{equation*}

因此,
\begin{equation*}
    \begin{aligned}
        \operatorname{var} V(\mathcal{R}) & \gtrsim|\mathcal{R}| \mathrm{e}^{-2 \pi r^2} 2 \pi \int_0^{2 r} h\left(\mathrm{e}^{\pi h^2-2 h r}-1\right) \mathrm{d} h \\
        &=|\mathcal{R}| \pi \mathrm{e}^{-\pi r^2} \frac{2-\mathrm{e}^{-\pi r^2}\left(\pi^2 r^4+2 \pi r^2+2\right)}{4 r^2},
        \end{aligned}
\end{equation*}

其中$\gtrsim$表示渐进意义下的不等关系。
为了得到上界,我们注意到$A(h)\leq \pi r^2-\pi r h/2$(for $h\leq 2r$),这导致
\begin{equation*}
    \operatorname{var} V(\mathcal{R}) \lesssim|\mathcal{R}| 4 \mathrm{e}^{-\pi r^2} \frac{2-\mathrm{e}^{-\pi r^2}\left(\pi^2 r^4+2 \pi r^2+2\right)}{\pi r^2}
\end{equation*}

所以有一个常数因子$16/\pi^2$在两个渐进的界之间,我们有
\begin{equation*}
    \operatorname{var} V\left([0, \sqrt{n}]^2\right)=\Theta\left(n \frac{\mathrm{e}^{-\pi r^2}}{r^2}\right), \quad \text { as } n \rightarrow \infty
\end{equation*}

当$n\rightarrow\infty$的时候,必须地$r\rightarrow\infty$来期望得到覆盖,因此我们已经丢弃了
方差的界的所有小项。

结果是,(13.4)中$\mathbb{P}(V>0)$的下界需要为0,来保证收敛,即,
$(\mathbb{E} V)^2 / \mathbb{E}\left(V^2\right) \rightarrow 0$.
因为$(\mathbb{E} V)^2=n^2e^{-2\pi r^2}$,条件是
\begin{equation*}
    \frac{(\mathbb{E} V)^2}{\operatorname{var} V+(\mathbb{E} V)^2} \rightarrow 0 \quad \Longleftrightarrow \quad n r^2 \mathrm{e}^{-\pi r^2} \rightarrow 0 \quad \text { as } n \rightarrow \infty
\end{equation*}

由于因子$r^2$的存在,有一个比(13.3)更加限制性的必要的条件.
令$\pi r^2=\log n+\log\log n+c$,我们有$nr^2e^{-\pi r^2}\rightarrow e^{-c/\pi}$,
所以我们需要将常数$c$替换为一个满足$\omega(1)$的函数,即,对于平面被覆盖,我们的新的必须条件是
\begin{equation}
    \pi r^2=\log n+\log \log n+\omega(1)
\end{equation}

平面被覆盖的充要条件

为了展示这个更加紧的必要条件(1.35)也是充分的,我们使用Gilbert对open disk这个case的一个观察:
面积为$n$的方形区域被覆盖,如果圆盘边界的所有的交集被覆盖,
磁盘边界和面积$n$的正方形区域边界之间的所有交点也是如此.

如果圆盘是closed,结果是相同的,但是Gilbert条件的方程是很不优雅的,因为圆盘的所有交,现在
不得不被覆盖3次(他们已经被2个形成交的圆盘覆盖过了,但是还要被another disk覆盖),
而盘边界和边界之间的交点需要被覆盖两次(它们已经被引起交点的盘覆盖了一次,并且需要被另一个盘覆盖)。图13.2显示了购买力平价在单位正方形和半径为r的圆及其交点上的两种实现。


使用Gilbert的观察,我们得到结论如下:
\begin{itemize}
    \item 1. 对于一个半径为$r$的圆盘,交的密度是$4\pi r^2$,因为盘边界在距离2r内与每个盘边界相交
    \item 2. 在面积$n$的方形区域的交的期望次数是$4\pi r^2 n$
    \item 3. 为了覆盖,每一个交需要存在有一个PPP的点距离小于$r$
    \item 4. 交没有被覆盖的平均次数,是$4\pi r^2n e^{-\pi r^2}$,如果(13.5)成立,它将会变成0。所以这个条件也是充分的。
\end{itemize}

现在我们考虑多覆盖的case。

\subsection{多覆盖}
这里我们需要平面的每个位置都被覆盖至少$k$次。Gilbert的条件以一种直接的方式广义化到$k$-coverage:
每一个圆盘边界的交必须被覆盖$k$次。

正如单覆盖的case里讨论的那样,充要条件是:交被覆盖$k-1$(或者更少)次的个数的平均$I_{k-1}$,
变为0:
\begin{equation*}
    \mathbb{E}\left(I_{k-1}\right) \sim 4 \pi r^2 n \exp \left(-\pi r^2\right) \frac{\left(\pi r^2\right)^{k-1}}{\Gamma(k)} \rightarrow 0, \quad \text { as } n \rightarrow \infty
\end{equation*}

\begin{theorem}
    (固定圆盘们的k-converage的条件)  对于$\mathbb{P}\left(V\left([0, \sqrt{n}]^2\right)=0\right) \rightarrow 1 \text { as } n \rightarrow \infty$,即,k-coverage,它的充要条件是:
    \begin{equation*}
        \pi r^2=\log n+k \log \log n+\omega
    \end{equation*}
\end{theorem}

\begin{proof}
    对于$\pi r^2=\log n+k\log\log n+c,\mathbb{E}(I_{k-1})\rightarrow e^{-c}\pi^{-k}$

    正如连通性问题那样,k-coverage需要一个比Single coverage稍大的半径。

    通过更多的努力,可以得到,对于$\pi r^2=\log n+k\log \log n+c$
    \begin{equation*}
        \mathbb{P}\left(V\left([0, \sqrt{n}]^2\right)=0\right) \rightarrow \exp \left(-\frac{\mathrm{e}^{-c}}{\Gamma(k)}\right), \quad n \rightarrow \infty
    \end{equation*}
\end{proof}

这个结果的证明背后的想法是:观察到,
未覆盖区域基本上形成了它们自己的均值$e^{-c}/\Gamma(\kappa)$泊松过程。
为了使得这些启发性的结果(heuristics)更加严格,我们需要用到the Chen-Stein方法。

\section{应用}
覆盖过程在无线网络中的经典应用当然是传感器网络。
对于这个问题,之前节得到的结果是直接可应用的。
这里我们讨论远在布尔模型之上的3个应用。

\subsection{无线网络中的SINR 覆盖}
让$\Phi=\{ x_i \}\subset \mathbb{R}^2$是一个点过程,
用$I^{!x}(y)$表示位置$y$处的干扰,如果$x\in \Phi$是关心的发射机,即,
\begin{equation*}
    I^{\mid x}(y)=\sum_{z \in \Phi \backslash\{x\}} h_{z y} \ell(\|z-y\|)
\end{equation*}

其中$h_{zy}$是$z$和$y$之间的衰落。我们定义位置$y$处的干扰:
\begin{equation*}
    \operatorname{SINR}(x, y)=\frac{h_{x y} \ell(\|x-y\|)}{W+I^{1 x}(y)}
\end{equation*}

其中$W$是热力学噪声功率。发射机$x$的覆盖蜂窝包括所有这样的位置,
在其中$x$为关心发射机的SINR等于或
大于阈值$\theta$:
\begin{equation*}
    S_x=\left\{y \in \mathbb{R}^2: \operatorname{SINR}(x, y) \geq \theta\right\}
\end{equation*}

由于衰落,蜂窝非常可能是disconnected的。SINR覆盖过程是
\begin{equation*}
    \Xi_{\mathrm{SINR}}=\bigcup_{x \in \Phi} S_x
\end{equation*}

如果$\theta\geq 1$,每一个位置不能被超过一个发射机覆盖,当然并不能是最近的那个。

无噪声覆盖

如果$\Phi$是一个强度为$\lambda$的uniform PPP,衰落是iid.的瑞利衰落,路径损失指数是$\alpha$,
$W=0$,那么位置$y$被发射机$x$覆盖的概率$p_x(y)$,可以从
Section 5.2中成功概率推导出:
\begin{equation*}
    p_x(y)=\exp \left(-\lambda \pi \theta^\delta \Gamma(1+\delta) \Gamma(1-\delta)\|x-y\|^2\right)
\end{equation*}

其中$\delta=2/\alpha$.如果$\theta\geq 1$,$p_x(y)$是被覆盖的区域部分,我们得到了平均的覆盖蜂窝大小
\begin{equation*}
    \mathbb{E}|S|=\int_{\mathbb{R}^2} p_o(y) \mathrm{d} y=\frac{1}{\lambda \theta^\delta \Gamma(1+\delta) \Gamma(1-\delta)} .
\end{equation*}

没有干扰和噪声,当路径损失指数$\alpha$增加的时候,覆盖蜂窝收敛到Voronoi蜂窝图

无干扰的覆盖
没有interference,所以现在就是SNR 覆盖过程,
它是一个布尔模型(如果$\Phi$是一个PPP的话)。没有衰落,SNR cell仅仅是
一个半径为$(\theta W)^{-1/\alpha}$的圆盘,因为
\begin{equation*}
    \frac{\|x-y\|^{-\alpha}}{W} \geq \theta \quad \Longleftrightarrow \quad\|x-y\|<\frac{1}{(\theta W)^{1 / \alpha}} .
\end{equation*}

所以$|S|=\pi(\theta W)^{-\delta}$. 噪声功率和SNR门限应该被单个参数替代。
使用发射功率$P$和【将噪声功率从$W$改成$W^{\prime}=W/P$】等价。在这个case中,我们可以使用
Section 13.3的结果,来找到关于$P$的条件,来使一个大的区域被覆盖。由于瑞利衰落,我们有
\begin{equation*}
    \mathbb{P}(\mathrm{SNR} \geq \theta)=\mathbb{P}\left(\frac{h\|x-y\|^{-\alpha}}{W}\right)=\exp \left(-\theta W\|x-y\|^\alpha\right)
\end{equation*}

以及
\begin{equation*}
    \mathbb{E}|S|=\frac{\pi \Gamma(1+\delta)}{(\theta W)^\delta}
\end{equation*}

在非衰落情况下按小区大小归一化:
\begin{equation*}
    \frac{\mathbb{E}|S|}{\pi(\theta W)^{-\delta}}=\Gamma(1+\delta)
\end{equation*}

这表明了瑞利衰落会减少平均蜂窝大小,如果$\alpha>2$,因为:
$\Gamma(1+\delta)<1$如果$\delta<1$.

噪声和干扰都考虑,计算覆盖

瑞利衰落下,噪声和衰落都考虑,覆盖概率是
\begin{equation*}
    \left.p_o(y)=\exp \left(-\theta W\|y\|^\alpha-\lambda \pi \theta^\sigma \Gamma(1+\delta) \Gamma(1-\delta)\|y\|^2\right)\right) .
\end{equation*}

对于$\delta=1/2$这是可积的,但是得到的结果是很丑陋的unwieldy。

\subsection{最近邻发射机的SIR覆盖}
这里我们考虑一个之前模型的轻微改动版本,其中关心的发射机并没有确定性给出,而是选择为
最近的一个发射机,这里假设在距离$R$处。
假设接收机在原点$o$,因为现在圆盘$b(o,R)$被保证没有干扰器,这将导致干扰引起很大的改变。
所以,对于iid的随机变量$h_x$,干扰是:
\begin{equation*}
    I=\sum_{x \in \Phi \backslash b(o, R)} h_x \ell(x)
\end{equation*}

再一次我们关注这样的case,强度为$\lambda$的PPP的发射机在平面上,对于关心的信号和干扰,衰落都是瑞利衰落。
噪声被忽略。

为了找到干扰$R$的拉普拉斯变换,我们可以遵循Section 5.1.7的步骤,
只不过现在$b(o,R)$是空的。
\begin{equation*}
    \begin{aligned}
        \mathcal{L}_I(s \mid R) &=\mathbb{E} \prod_{x \in \Phi \backslash b(o, R)} \mathbb{E}_h\left(\mathrm{e}^{-s h \ell(x)}\right) \\
        &=\mathbb{E} \prod_{x \in \Phi \backslash b(o, R)} \frac{s \ell(x)}{1+s \ell(x)} \\
        & \stackrel{(\mathrm{a})}{=} \exp \left(-\lambda \int_{\mathbb{R}^2 \backslash b(o, R)} \frac{1}{1+s \ell(x)} \mathrm{d} x\right) \\
        &=\exp \left(-2 \pi \lambda \int_R^{\infty} \frac{r s}{s+r^\alpha} \mathrm{d} r\right) \\
        &=\exp \left(\pi s^\delta \Gamma(1+\delta) \Gamma(1-\delta)-\pi R^2 H_\delta\left(R^\alpha / s\right)\right),
        \end{aligned}
\end{equation*}

其中(a)是因为pgfl,$delta=2/\alpha$,$H_\delta$是高斯双曲函数
\begin{equation*}
    H_\delta(x) \triangleq{ }_2 F_1(1, \delta ; 1+\delta ;-x)
\end{equation*}

利用$H_\delta$,从$0$到$R$的 “complementary integral”可以表示成
\begin{equation*}
    \int_0^R \frac{r s}{s+r^\alpha} \mathrm{d} r=\frac{R^2}{2} H_\delta\left(R^\alpha / s\right) .
\end{equation*}

$o$被最近邻发射机(距离$R$处)覆盖的概率因此是
\begin{equation*}
    \begin{aligned}
        p_o(R) &=\mathbb{P}\left(h R^{-\alpha}>\theta I\right) \\
        &=\mathcal{L}_I\left(\theta R^\alpha\right) \\
        &=\exp \left(-\lambda \pi R^2\left(\theta^\delta \Gamma(1+\delta) \Gamma(1-\delta)-H_\delta(1 / \theta)\right)\right) .
        \end{aligned}
\end{equation*}

因为$R$是PPP中最近邻的距离,它的概率分布是$f_R(r)=2\lambda \pi r e^{-\lambda \pi r^2}$,
覆盖的概率是
\begin{equation*}
    \begin{aligned}
        p_o &=\mathbb{E}\left(p_o(R)\right) \\
        &=\frac{1}{1+\theta^\delta \Gamma(1+\delta) \Gamma(1-\delta)-H_\delta(1 / \theta)}
        \end{aligned}
\end{equation*}

对于$\alpha=4$,因为$H_{1/2}(1/\theta)=\sqrt{\theta}$,
我们得到覆盖概率$p_o$的简单的闭形式
\begin{equation}
    p_o=\frac{1}{1+\sqrt{\theta}(\pi / 2-\arctan (1 / \sqrt{\theta}))} .
\end{equation}



\subsection{隐私覆盖?隐秘覆盖?}
隐私图,(包括一些边,在那些边上安全通信是可行的),首次在11.4.3节提出。
这里我们研究对应的覆盖问题。

基站和窃听器在平面上随机分布,基站可以覆盖蜂窝区域,并且基站可以覆盖半径由到最近的窃听者的距离确定的圆形区域。
虽然这些假设导致了分析上的可处理性
模型,它们非常逼真。事实上,蜂窝网络正在经历
从精心规划的基站部署到不定期部署的重大转变
包括微型基站和毫微微小区的异构基础设施。即使
如果没有这样小的基站,可以认为从
随机模型的精度与基于晶格的模型相同或更好
基站模型,见图。7.1和7.2。

虽然这些假设导致了分析上的可处理性
模型,它们非常逼真。事实上,蜂窝网络正在经历
从精心规划的基站部署到不定期部署的重大转变
包括微型基站和毫微微小区的异构基础设施。即使
如果没有这样小的基站,可以认为从
随机模型的精度与基于晶格的模型相同或更好
基站模型,见图。7.1和7.2。

我们假定基站和窃听器来自独立的强度为$1$和$\lambda$的PPP,
对应地,在$\mathbb{R}^d$中,它们分别是$\Phi$和$\Psi $。
一个$\lambda=0.1$的例子在图13.3中展示。

该覆盖率模型的行为与相应的(具有相同分布的独立盘半径)的模型完全不同,
如果radii是
独立的,我们从式子(13.5)中知道
\begin{equation*}
    \mathbb{E}\left(\pi R^2\right)=(1+\in) \log n
\end{equation*}

是【渐进覆盖一个面积为$n$的方形】充分条件(对于任意$\in >0$)。
在隐私覆盖的case中,$\mathbb{E}(\pi R^2)=1/\lambda$,所以the radii是独立的,
\begin{equation*}
    \lambda=[(1+\in) \log n]^{-1}
\end{equation*}

是充分的,为了覆盖,$\lambda$随着$n$对数趋势减少。
由于依赖性(一个监听者可能确定一些近邻的覆盖圆盘的半径),
$\lambda$下降得如此之快,大概是一个$n^{-1/d}$的趋势。

一维中的隐私覆盖

我们确定线段的覆盖比例$\mathbb{P}(o \in \Xi)=1-\mathbb{E} \chi(o)$

\begin{theorem}
    
    (一维中的隐私覆盖)覆盖的线段的比例是
    \begin{equation*}
        \mathbb{P}(o \in \Xi)=\frac{1+4 \lambda}{(1+2 \lambda)^2}
    \end{equation*}
\end{theorem}

\begin{proof}
    让$L$是【$o$被一个$o$左侧的点覆盖】事件。让$R$是【$o$被一个$o$右侧的点覆盖】事件。
    事件$L$和$R$是独立的,
    \begin{equation*}
        \mathbb{P}(o \in \Xi)=1-(1-\mathbb{P}(L))(1-\mathbb{P}(R))=1-(1-\mathbb{P}(R))^2=2 \mathbb{P}(R)-\mathbb{P}(R)^2
    \end{equation*}

    具有对称性。事件$R$发生,如果,对于某个$t>0$,$\Phi$中最近的点
    (a legitimate node)在距离$t$,没有$\Psi $(监听者)中的点在$[0,2t]$。因此
    \begin{equation*}
        \mathbb{P}(R)=\int_0^{\infty} \mathrm{e}^{-t} \mathrm{e}^{-2 \lambda t} \mathrm{~d} t=\frac{1}{2 \lambda+1}
    \end{equation*}
\end{proof}

所以一个长度为$n$的区间的期望的vacancy,是
\begin{equation*}
    \mathbb{E} V([0, n])=n \mathbb{E} \chi(o)=n(1-\mathbb{P}(o \in \Xi))=\frac{4 \lambda^2 n}{(1+2 \lambda)^2}
\end{equation*}

为了达成对实线的覆盖($n\rightarrow \infty$),我们因此需要$\lambda^2 n\rightarrow 0$。
一个更加细节化的分析展示出,如果$n\rightarrow \infty$,$\mathbb{P}(V=0)\sim e^{-4 n \lambda^2}$。
所以,对于覆盖,尺度缩放条件事实上是$\lambda =\mathcal{o}(n^{-1/2})$。
如果$n\lambda^2\rightarrow \infty$,$\mathbb{P}(V=0) \rightarrow 0$;
如果$n\lambda^2\rightarrow 0$,$\mathbb{P}(V=0)\rightarrow 1$。

2维中的隐私覆盖

2维的case要更加复杂。自然的推论是:
关键缩放是$\lambda=\mathcal{o}(n^{-1/3})$.
事实上,可以发现,如果$n\lambda^3\rightarrow \infty$,
$\mathbb{P}(V=0)\rightarrow 0$。
传统上,最著名的结果是
\begin{equation*}
    n(\log n)^3 \lambda^3 \rightarrow \quad \Longrightarrow \quad \mathbb{P}(V=0) \rightarrow 1
\end{equation*}

和$\lambda=o\left(n^{-1 / 3}(\log n)^{-1}\right)$,
所以有一个对数gap.


\subsection{哨兵选择}
哨兵选择问题是从能量有效传感器网络中一个实用issue抽象出来的。
一个感兴趣的区域$\mathcal{R}$被认为是k-covered(每个$k\in \mathcal{R}$被至少$k$个传感器覆盖),
这$k$个传感器可以被$k$钟颜色以某种方式染色,使得
$\mathcal{R}$被指定颜色的传感器子集所覆盖吗?
如果是这种情况,那么可以将时间划分为帧,使得在帧i中,只有颜色1+mod(i,k)的节点处于活动状态,而其他节点可以休眠以节省能量,同时仍然保证始终覆盖。

因为k-coverage是可以以一些小代价达成的(传感器感知半径$r$的略微增长——看定理13.7),
可以预期显著的能量增益$\&$前提是提供k覆盖的传感器可以被划分为k个单个覆盖。

明显地,这一般是不可能的,很容易构造一个k-cover的反例,其中
k-cover并不是可分的。

\begin{example}
    让$n=4$,$k=2$.$\{1,2,3,4\}$的大小为2的子集构成的集合有6个元素,
    \begin{equation*}
        \mathcal{S}=\{\{1,2\},\{1,3\},\{1,4\},\{2,3\},\{2,4\},\{3,4\}\} .
    \end{equation*}

    那么让
    \begin{equation*}
        \begin{aligned}
            &S_1=\{\{1,2\},\{1,3\},\{1,4\}\}, \\
            &S_2=\{\{1,2\},\{2,3\},\{2,4\}\}, \\
            &S_3=\{\{1,3\},\{2,3\},\{3,4\}\}, \\
            &S_4=\{\{1,4\},\{2,4\},\{3,4\}\} .
            \end{aligned}
    \end{equation*}

    $S$的每一个元素在$S_1,S_2,S_3,S_4$只出现2次,
    但是不能将$S_i$分割成2个单独的覆盖。

\end{example}


具体的设置是:一个强度为1的传感器节点的PPP,一个感兴趣的区域$\mathcal{R}=[0,\sqrt{n}]^2$。
假设每个传感器可以覆盖一个半径为$r$的圆盘,让$\Xi$是得到的布尔模型。
让$C_k$标示$\mathcal{R}$被k-覆盖的事件。我们知道,
k-覆盖可以渐进得到,如果$\pi r^2=\log k\log \log n+\omega(1)$.
让$P_k$是$\Xi$是k-partitionable的事件。那么

\begin{equation*}
    \mathbb{P}\left(C_k \backslash P_k\right) \leq \frac{c_k}{\log n}
\end{equation*}

因此,如果$\mathbb{P}\left(C_k\right)=\Theta(1)$,那么
当$n\rightarrow \infty$的时候,条件概率$\mathbb{P}\left(P_k \mid C_k\right)=1-o(1)$。
严格的证明是很长的。
已经被知道:k-partitionability的“障碍”,正如图13.4中展示的那样,当$n\rightarrow \infty$的时候
以接近0的概率发生。所以渐进上,k-覆盖可以被分拆almost surely.提出的哨兵选择方案是可以work的。

一个更加简单的方式是:验证一个随机染色是否可以以正概率达到parition.
这是"概率方法"的一个应用。这个方法,和Lovasz局部定理一样,产生
想要的结果如果每处覆盖的等级是$3k \log \log n$的。

这个证明方法并不是构造的,即,如果找到这样的染色仍然是开放问题。事实上,
传感器节点应该通过使用单独局部知识来选择一个颜色。对于覆盖过程更具体的介绍可以
在Hall(1985)的著作中找到。在第4章Stoyan et al.(1995)的部分,以及第3章到第5章的Meester\&Roy(1996)的部分,我们介绍了他们是怎么研究布尔模型的。
概率方法是Alon\&Spencer(2008)的研究对象。

SINR覆盖在第二部分的Baccelli \& Blaszczyszyn (2009)被讨论。
最近邻发射机覆盖的case在Andrews et al. (2011)得到了详细分析。
拓展到multi-tier异构网络的case在Dhillon et al. (2012)被研究。

在Sarkar \& Haenggi
(2012)中,隐私覆盖的问题被引入和研究
关于哨兵选择的主要的结果由Balister et al.
(2010).给出。

\section*{练习题}

\begin{exercise}
    让$\Phi_1,\Phi_2,\cdots$是$\mathbb{R}$上的独立的强度为$\lambda$的均匀PPP,
    让$1\geq t_1\geq t_2,\cdots$是收敛到0的正数序列。
    \begin{equation*}
        \Xi=\bigcup_{i \in \mathbb{N}} \bigcup_{x \in \Phi_i}\left(x, x+t_i\right),
    \end{equation*}
    其中$(x,x+t_i)\subset \mathbb{R}$是从$x$到$x+t_i$的开区间,
    让$V=V(\mathbb{R})=|\mathbb{R}\backslash \Xi \mid $是这个模型的vacancy.
    未覆盖的集合$\mathbb{R}\backslash\Xi$被称作随机康托集。
    找到位置$y\in \mathbb{R}$没有被$\Xi$覆盖的概率,给出要使$\mathbb{P}(V=0)$成立,关于序列$(t_i)$的充分条件和必要条件。
\end{exercise}

\begin{exercise}
    对于具有半径$r$的固定圆盘的布尔模型的单个覆盖,找到关于$r(n)$需要满足的条件,可以使得,
    vacancy$V=V([0,\sqrt{n}]^2)$,平均趋于0,但是$V=0$的概率并不是1。即,
    \begin{equation*}
        \mathbb{E}(V) \rightarrow 0  \text{but}  \mathbb{P}(V=0) \nrightarrow 1
    \end{equation*}
    有一个$r(n)$可以使得$mathbb{E}(V)\rightarrow 0$并且$\mathbb{P}(V=0)=0$吗?
\end{exercise}

\begin{exercise}
    Stienen模型是一个二维的覆盖过程,其中germs来自均匀PPP,grains是半径是【最近邻距离一般】的圆盘。
    给出Stienen模型的一个在$[0,5]^2$的实现,其中PPP的强度是1。确定模型的覆盖面积比例。
\end{exercise}

\begin{exercise}
    在标准布尔模型(germs来自PPP)中,巨大成分是唯一的如果它存在。通过反例表明,在顶点形成固定晶格的模型中,这并不一定正确。
    
\end{exercise}

\begin{exercise}
    对于在13.4.2节讨论的最近邻发射机覆盖问题,找到$\alpha=4$时的覆盖概率。
    正如(13.6)那样,但是考虑噪声。
\end{exercise}

% 13.6
\begin{exercise}
    一维中的隐私覆盖。考虑两个窃听者之间的典型到达间隔。根据窃听者PPP的强度,找出该间隔被覆盖的概率。
    提示:首先说明间隔的覆盖范围仅取决于间隔中点附近两点的位置。
\end{exercise}




todo
todo
todo
todo
todo
todo
todo


\chapter{占位章节}

\section{我也不想写啊,只有个章节标题,目录会把这一章忽略掉啊}
为什么啊

为什么有时候我要XeLaTeX编译两次才能显示正确的目录???
\subsection{用作subsection的占位符}
爷的目录为什么显示不全?
\section{用作section的占位符1}
爷的目录为什么显示不全?
\subsection{用作subsection的占位符1}
爷的目录为什么显示不全?
你到底会不会做\LaTeX
模板啊,\lstinline{\section*}在章节里看没有序号,
但是在目录中是有序号的啊,书签中也是有序号的啊
????



%%%%%%%%%%%%%%%%%%%%%%%%%%%
测试代码环境

\begin{lstlisting}[language=Python]
## 练习使用collections模块的Counter

from collections import Counter 
class Solution:
    def findJudge(self, n: int, trust: List[List[int]]) -> int:
        tmp1=map(lambda x:x[0],trust)
        outDgree=Counter(tmp1)
        tmp2=map(lambda x:x[1],trust)
        inDgree=Counter(tmp2)
        for i in range(1,n+1,1):
            if inDgree[i]==n-1 and outDgree[i]==0:
                return i
        return -1        
        
    
\end{lstlisting}
%%%%%%%%%%%%%%%%%%%%%%%%%%%
\input{chapters/chapter_elegant.tex}




\end{document}
